% Лекции Сергея Борисовича Стечкина% ??? Внесены исправления В.А. Юдина (параграф 16.1), версия 27.01.2009% Внесены исправления В.В.Арестова (начиная с параграфа 16.2), версия 06.07.2009% Внесены исправления Н.И. Черныха, версия 29.07.2009% Внесена грамматическая и ТеХ-правка М.Дейкаловой, версия 05.08.09\chapter{Неравенство Джексона в \boldmath ${L^2}$ с точной константой. \\Нормы сумм Валле Пуссена}\section{Теорема Черныха}Выше доказано неравенство Джексона в $L^2_{2\pi}$$$E_n(f)_{L^2}\le C\omega\left(\frac{\pi}{n},f\right)_{L^2}.$$Заметим, что неравенство можно усилить, так как доказательство проходит для $E_{n-1}(f)_{L^2}$:$$E_{n-1}(f)_{L^2} \le C\omega \left(\frac{\pi}{n},f\right)_{L^2}.$$Далее рассматривается следующая\begin{task}Указать наилучшее возможное значение константы $C$,при котором неравенство Джексона остается справедливымдля любого $n$.\end{task}Итак, мы хотим найти$$\sup_{{f\in L^2}\atop {f\not\equiv \Const}} \frac{E_{n-1}(f)_{L^2}}{\omega \left(\frac{\pi}{n},f\right)_{L^2}}=C_n$$и$$\sup_n C_n=C.$$Для $f\equiv \Const$ имеем $E_{n-1}(f)_{L^2}=0$,~$\omega\left(\dfrac{\pi}{n},f\right)_{L^2}=0$ и неравенство верно для любой константы.Оказывается, что $C_n$ от $n$ не зависит, $C_n=C=\dfrac{1}{\sqrt{2}}$, иокончательная форма неравенства Джексона в $L^2_{2\pi}$ принимаетвид:\begin{teo}[{Н.\,И\,.Черных}]\label{t1-Chernykh}Для каждой $f\in L^2_{2\pi}$, $f\not\equiv \Const$, и для любого $n\in \mathbb N$$$E_{n-1}(f)_{L^2}{<} \frac{1}{\sqrt{2}}\, \omega\left( \frac{\pi}{n},f\right)_{L^2},$${константу здесь уменьшить нельзя, хотя} неравенство превращаетсяв равенство только {для} $f\equiv \Const$. {Константа$\dfrac{1}{\sqrt{2}}$ здесь точная\footnote{Позжебыло доказано, что $\dfrac{\pi}{n}$ здесь тоже уменьшить нельзя без увеличения константы$2^{-1/2}$ перед $\omega$ (В.\,В.\,Арестов, Н.\,И.\,Черных, 1979 г.).}.}\end{teo}Но оказывается, что можно построить другой функционал,который меньше $\omega\left( \dfrac{\pi}{n},f\right),$а оценка для $E_{n-1}(f)$через этот функционал остается справедливой. Это теоремаЧерныха, неравенство Джексона с точной константой будет из нее следовать.\begin{teo}[Н.\,И.\,Черных]\label{t-Chernykh}Для каждой $f\in L^2_{2\pi}$ для любого $n\in \mathbb N$\begin{equation}\label{l16-chernykh}E_{n-1}^2 (f)_{L^2}\le \frac{n}{4}\int_0^{\frac{\pi}{n}} \|\Delta_t f\|_{L^2}^2 \sin nt\,dt=J_n\end{equation}{и неравенство обращается в равенство только для функций из $L_{2\pi}^2$ вида}$${\alpha_0+\sum\limits_{k=1}^\infty(\alpha_k\cos(2k+1)nx+\beta_k\sin(2k+1)nx).}$$\end{teo}{Таким образом,} $J_n$~-- тоже структурная характеристика функции, как и$\omega\left( \dfrac{\pi}{n},f\right).$ Очевидно, что неравенствоДжексона $\Big($с константой $C=\dfrac{1}{\sqrt{2}}\Big)$ есть следствиетеоремы Черныха. Действительно,$$\omega(h,f)=\sup_{|t|\le h} \|\Delta_t f\|_{L^2}=\sup_{0\le t\le h}\|\Delta_t f\|_{L^2}\ge \|\Delta_t f\|_{L^2}.$$Следовательно,$$J_n\le \frac{n}{4}\omega^2 \left( \frac{\pi}{n},f\right)\int_0^{\frac{\pi}{n}}\sin nt\, dt=\frac12\, \omega^2\left( \frac{\pi}{n},f\right),$$т.\,е. {неравенство Джексона можно записать в форме}$$E_{n-1}(f)_{L^2}\le \frac{1}{\sqrt{2}}\, \omega\left( \frac{\pi}{n},f\right).$$Когда имеет место равенство, выясним позже.%\begin{proof}Д\;о\;к\;а\;з\;а\;т\;е\;л\;ь\;с\;т\;в\;о\quad {теоремы~\ref{t-Chernykh}}.Пусть, как обычно, $\rho_k^2=a_k^2+b_k^2$ {$(k=1,\ldots).$}Тогда\begin{equation}\label{l16-chernykh-proof-1}E_{n-1}^2(f)_{L^2}=\sum\limits_{k=n}^{\infty} \rho_k^2\end{equation}и$$\|\Delta_t f\|_{L^2}^2=4\sum\limits_{k=1}^{\infty} \sin^2\frac{kt}{2}\rho_k^2=2\sum\limits_{k=1}^{\infty}\rho_k^2 (1-\cos kt)\ge$$$$\ge 2\sum\limits_{k=n}^{\infty}\rho_k^2 (1-\cos kt)=2E_{n-1}^2(f)_{L^2}-2\sum\limits_{k=n}^{\infty}\rho_k^2 \cos kt.$$Отсюда\begin{equation}\label{l16-chernykh-proof-2}E_{n-1}^2(f)_{L^2}\le \frac12 \|\Delta_t f\|_{L^2}^2+\sum\limits_{k=n}^{\infty}\rho_k^2\cos kt.\end{equation}Умножим обе части неравенства~(\ref{l16-chernykh-proof-2}) на $\sin t$ и проинтегрируем в промежуткеот 0 до~$\dfrac{\pi}{n}$. Получим$$ E_{n-1}^2(f)_{L^2}\int_0^{\frac{\pi}{n}}\sin nt\, dt=\frac{2}{n} E_{n-1}^2(f)_{L^2}\le $$$$ \le \frac{1}{2} \int_{0}^{\frac{\pi}{n}} \|\Delta_t f\|_{L^2}^2 \sinnt\, dt+\int_0^{\frac{\pi}{n}} \sum\limits_{k=n}^{\infty}\rho_k^2 \cos kt\sin nt\, dt. $$ В последнем интеграле ряд сходится абсолютно иравномерно, значит, можно поменять местами интегрирование и суммирование, так что$$\frac{2}{n}E_{n-1}^2(f)_{L^2} \le \frac{1}{2}\int_0^{\frac{\pi}{n}}\|\Delta_t f\|_{L^2}^2 \sin nt\, dt+ \sum\limits_{k=n}^{\infty}\rho_k^2\int_0^{\frac{\pi}{n}} \cos kt\sin nt\, dt,\qquad k\ge n.$$Подсчитаем интегралы под знаком суммы.\begin{equation}\label{l16-chernykh-proof-3}\int_0^{\frac{\pi}{n}} \cos kt\sin nt\, dt= \left\{\begin{array}{ll}0, & k=n \\\dfrac{2n}{n^2-k^2}\cos^2 \dfrac{k\pi}{2n}, & k>n\end{array}\right\} \le 0\qquad (k\ge n).\end{equation}Значит,\begin{equation}\label{l16-chernykh-proof-4}\sum\limits_{k=n}^{\infty}\rho_k^2 \int_{0}^{\frac{\pi}{n}} \cos kt\sinnt \le 0\end{equation}и$$\frac{2}{n} E_{n-1}^2(f)_{L^2}\le \frac{1}{2} \int_0^{\frac{\pi}{n}} \|\Delta_t f\|_{L^2}^2\sin nt\, dt.$$Неравенство~(\ref{l16-chernykh}) доказано.%\end{proof}Посмотрим, когда в неравенстве~(\ref{l16-chernykh}) будет равенство.Для этого надо, чтобы во всех выкладках было равенство.Проверим все такие места.1) Условие$$\sum\limits_{k=1}^{n-1}\rho_k^2 \ds\int_0^{\frac{\pi}{n}}\cos kt \sin nt\, dt=0$$(см.~(\ref{l16-chernykh-proof-1}) и~(\ref{l16-chernykh-proof-3})), в силу свойств$$\ds\int_0^{\frac{\pi}{n}}\cos kt\sin nt\, dt>0,\qquad 1\le k<n,$$означает, что $\rho_k=0$ для всех $k,$~ $k<n,$ кроме $k=0.$2) В~(\ref{l16-chernykh-proof-3}) и~(\ref{l16-chernykh-proof-4}) равенство будет,когда $\rho_k=0$ для всех $k\ne (2m+1)n.$Таким образом, в теореме~\ref{t-Chernykh} равенство будет тогда и только тогда, когдафункция $f$ имеет ряд Фурье вида$$\frac{a_0}{2}+\sum\limits_{k=0}^{\infty} a_{(2k+1)n}\cos(2k+1)nx+b_{(2k+1)n}\sin(2k+1)nx.$$Период таких функций равен $\dfrac{2\pi}{n}.$Исследуем теперь, когда в неравенствах$$E_{n-1}^2(f)_{L^2}\le \frac{n}{4}\int_0^{\frac{\pi}{n}} \|\Delta_t f\|_{L^2}^2 \sin nt\, dt\le \frac{1}{2} \,\omega^2 \left(\frac{\pi}{n},f\right)$$будет равенство, т.\,е. когда будет равенство в неравенствеДжексона в форме теоремы~\ref{t1-Chernykh}. Равенство будет только в том случае,когда для всех$t,$~ $0\le t\le \dfrac{\pi}{n},$$$\|\Delta_t f\|_{L^2}^2=\omega\left( \frac{\pi}{n},f\right).$$Учитывая, что $\|\Delta_t f\|_{L^2}^2\to 0$~ $(t\to 0),$ получаем$$\|\Delta_t f\|_{L^2}^2=\omega\left( \frac{\pi}{n},f\right)=0,$$т.\,е. $ \|\Delta_t f\|_{L^2}^2=0.$Следовательно, $\rho_k=0\ \ \forall\ k>0,$а это значит, что $f\equiv \Const.$Итак, доказано, что$$\sup_{{f\in L^2}\atop {f\not\equiv \Const}} \frac{E_{n-1}(f)_{L^2}}{\omega\left( \frac{\pi}{n},f\right)}\le \frac{1}{\sqrt{2}}.$$Покажем, что здесь на самом деле равенство. Выберем {$2\pi$-периодическую функцию}$$f_\eps(x)=\begin{cases}1,& 0\le x\le\eps\\ 0,& \eps<x<2\pi\end{cases},\qquad \eps<\pi,$$ подсчитаем среднее значение $f_\eps(x)$ на $(0,2\pi)$:$${\frac{1}{2\pi}}\int_0^{2\pi}f_\eps(x)dx=\frac{\eps}{2\pi}.$$Подсчитаем $E_0^2(f_\eps)$. Имеем$$E_0^2(f_\eps)=\frac{1}{\pi}\int_0^{2\pi}\left(f_\eps(x)-\frac{\eps}{2\pi}\right)^{{2}}\,dx =\frac{\eps}{\pi}{\left(1-\frac{\eps}{2\pi}\right)}.$$Оценим сверху $\omega^2(\pi,f_\eps)$. Для любого $t$ находим$$\|\Delta_tf_\eps\|^2=\frac1\pi\int_0^{2\pi}[f_\eps(x+t)-f_\eps(x)]^2dx\le\frac1\pi\int_0^{2\pi}[f_\eps^2(x+t)+f_\eps^2(x)]dx=\frac{2\eps}{\pi}.$$ Следовательно,при $\eps\to0$\begin{equation}\label{l16-E_0}\frac{E_0^2(f_\eps)}{\omega^2(f_\eps,\pi)}\ge {\frac{1-\eps/2\pi}{2}}\to\frac12.\end{equation}Таким образом получаем, что точность константы в теореме~\ref{t1-Chernykh} доказана для $n=1$.Для произвольного $n\in\mathbb N$,~ {$n\ge 2$,} рассмотримпериодизацию$$f_{\eps,n}(x)=\frac1n\sum\limits_{k=0}^{n-1}f_\eps\left(x-\frac{2\pi {k}}n\right),\qquad\eps<\dfrac{\pi}n.$${Эта функция} $\dfrac{2\pi}n$-периодична, {носители слагаемых не пересекаются,$f_{\eps,n}=\dfrac{1}{n}f_{\eps}$ на} {$\left[0,\dfrac{2\pi}{n}\right)$,} поэтому$$\dfrac{1}{2}\,a_0(f_{\eps,n})=\dfrac{1}{2}\,a_0(f_{\eps})=\dfrac{\eps}{2\pi},$$$$E_{n-1}^2(f_{\eps,n})={\frac{n}{\pi}\int_0^{2\pi/n}\left(f_{\eps n}(x)-\frac{\eps}{2\pi}\right)^2dx}={\frac{\eps}{n\pi}\left(1-\frac{\eps n}{2\pi}\right)}.$$Аналогично оцениваем модуль непрерывности. Для любого $t$$$\|\Delta_tf_{\eps,n}\|_{L^2}^2\le{\frac{2\eps}{n\pi}},$$что снова приводит {к оценке вида}~(\ref{l16-E_0}) для $n\ge 2$.\begin{remark}\label{r16-1}{Полное} пространство $H$\ $2\pi$-периодических суммируемыхфункций~$f$, инвариантное относительно сдвига на любое $h\in R,$в котором норма обладает свойствами $$ \forall\  h\qquad \|f(x+h)\|_{H}=\|f(x)\|_{H}, $$ $$ \forall\  f\qquad \|\Delta_t f\|_{H}\to 0\qquad (t\to 0) $$будем называть однородным пространством.Нетрудно видеть,что такое пространство содержит множество тригонометрических полиномов(не обязательно всех), которое всюдуплотно в этом пространстве, и справедливо неравенство Джексона$$E_{n-1}(f)_{H}\le C\omega \left(\frac{\pi}{n},f\right)_{H},$$для наилучших приближений полиномами из подпространства ${\cal T}_n\cap H,$когда это подпространство не пусто. Утверждение отригонометрических полиномах в $H$ вытекает из первыхчетырех условий приведенного определения однородногопространства.\end{remark}\begin{task}Верна ли гипотеза (С.\,Б.\,Стечкин), состоящая в том, чтоесли здесь $C=\dfrac{1}{\sqrt{2}}$ и если однородное функциональное пространство $H$достаточно большой размерности $(\ge 3),$ то оно гильбертово.\end{task}\begin{task}Верно ли неравенство$$E_{n-1}^p(f)_{L^p}\le \int_0^{\frac{\pi}{n}} \|\Delta_t f\|^p_{L^p} \varphi_n(t)\,dt,\qquad p>1,$$для некоторых $\varphi_n(t)$,~ {$\ds\int_0^{\frac{\pi}{n}} \varphi_n(t) dt <C$\,?}\end{task}{Следствием} этого неравенства {было бы} неравенство Джексона {для $L_{2\pi}^p$}.Для дифференцируемых $2\pi$-периодических функций знаем неравенство$$E_{n-1}(f)_{L^2}\le \frac{1}{n^k} E_{n-1} (f^{(k)})_{L^2},$$тогда по неравенству Джексона в форме Черныха имеем$$E_{n-1}(f)_{L^2}\le \frac{1}{n^k} E_{n-1} (f^{(k)})_{L^2}\le \frac{1}{\sqrt{2}}\frac{\omega\left(\frac{\pi}{n},f^{(k)}\right)_{L^2}}{n^k}.$$Таким образом из теоремы~\ref{t1-Chernykh} вытекает\begin{Corollary}Если у функции $f(x)\in C_{2\pi}$ существует абсолютнонепрерывная производная порядка $k-1$ и $f^{(k)}\inL^2_{2\pi},$ то$$E_{n-1}(f)_{L^2}\le \frac{1}{\sqrt{2}}\cdot\frac{1}{n^k}\omega\left( \frac{\pi}{n},f^{(k)}\right)_{L^2}.$$Это утверждение можно переписать в виде$$\sup_{{f:\ f^{(k)}\in L^2}\atop {f^{(k)}\not\equiv \Const}}\frac{E_{n}(f)_{L^2} n^k}{\omega\left( \frac{\pi}{n},f^{(k)}\right)_{L^2}}\le \frac{1}{\sqrt{2}}\,.$$\end{Corollary}\begin{task} Здесь найти точную константу пока неудается.\end{task}\begin{task}В неравенстве$$E_{n-1}(f)_{L^2}\le C_k\omega_k\left(\frac{\pi}{n},f\right)_{L^2}$$известна точная оценка только для $k=1.$ Для остальных $k$ наилучшаяконстанта $C_k$ не вычислена.\end{task}\begin{task}Для функций $m$ переменных на торе $\bT^m$ не ясно, как сформулироватьсоответствующие задачи: так$$E_n(f)_{L^2(\mathbb T^m)}\le C_m \omega \left( \frac{\pi}{n},f\right)_{L^2(\mathbb T^m)}$$или, скорее всего, так$$E_n(f)_{L^2(\mathbb T^m)}\le C_m \omega \left( \frac{\gamma_m}{n},f\right)_{L^2(\mathbb T^m)},$$надо найти порядок роста $\gamma_m$ при $m\to\infty.$\footnote{В.\,А.\,Юдин доказал (1981 г.) этонеравенство с точной константой $\dfrac{1}{\sqrt{2}}$ длянаилучших приближений тригонометрическими полиномами соспектром в круге радиуса $n$ с наименьшей при$C_m=\dfrac{1}{\sqrt{2}}$ (как позже доказал Д.\,В.\,Горбачев)константой $\gamma_m$.}\end{task}Все эти задачи можно решать вариационным методом и это на самом деле вариационные задачи. \section{Нормы сумм Валле Пуссена. Теорема Никольского}Начиная с этого момента, основной нашей целью будет изучение  наилучших приближенийнепрерывных $2\pi$-периодических функций (т.\,е. функций из пространства$C=C_{2\pi}$) тригонометрическими полиномами относительно нормы$$ \|f\|_{C}=\max\{|f(x)|:\ x\in(-\infty,\infty)\}. $$Пусть   $f\in {C}=C_{2\pi}.$  Рассмотрим суммыВалле Пуссена\footnote{В обозначениях лекции~6 средние Валле Пуссена,определенные здесь и в лекции~17, обозначались бы как $\sigma_{n,n-p},$ чтоследует помнить при использовании результатов из лекции~6.}\begin{equation}\label{f16-1} \frac{1}{p+1}\sum\limits_{k=n-p}^n s_k(f)=\sigma_{n,p}(f); \end{equation} здесь {$p$~-- целое,}  $0\le p\le n.$   Это есть   линейный оператор из $C$ в $C;$ изучим его норму  $$ \|\sigma_{n,p}\|_C^C=L_{n,p}. $$ \begin{teo}[{С.\,М.\,Никольский, 1940 г.}] \label{t16-3} При  $0\le p\le n$ $$ L_{n,p}=\frac{4}{\pi^2}\ln \frac{n}{p+1}+O(1). $$ \end{teo} \begin{Remark} Если $p=0,$ то $\sigma_{n,0}=s_n$ и получаем известную асимптотическую формулу для констант Лебега сумм Фурье. Если $c\, n\le p\le n,$ то $L_{n,p}=O(1).$ Нормы  $L_{n,p}$ растут в том и только том случае, когда $p=o(n).$  Следовательно, формула является асимптотической при $p=o(n).$ \end{Remark}  \begin{lemma}\label{l16-1a} При $r\ge 1$  $$ \int_0^{\pi}\frac{|\sin rt|}{t}\, dt= \frac{2}{\pi} \ln r+O(1). $$ \end{lemma} \proof  Запишем $$ \int_0^{\pi} \frac{|\sin rt|}{t}\, dt= \sum\limits_{k=0}^{k_0} \int_{\frac{k\pi}{r}}^{\frac{(k+1)\pi}{r}} \frac{|\sin rt|}{t}\, dt+ \int_{\frac{(k_0+1)\pi}{r}}^{\pi} \frac{|\sin rt|}{t}\, dt, $$ где целое неотрицательное  число $k_0$ выбрано из условия $k_0+1\le {r}<k_0+2$ или, то же самое, $\dfrac{(k_0+1)\pi}{r}\le \pi <\dfrac{(k_0+2)\pi}{r}.$ Последний интеграл ограничен при $r\ge 1$: $$ \int_{\frac{(k_0+1)\pi}{r}}^{\pi} \frac{|\sin rt|}{t}\, dt=O(1). $$ Сделав нужную замену, получаем $$ \sum\limits_{k=0}^{k_0} \int_{\frac{k\pi}{r}}^{\frac{(k+1)\pi}{r}} \frac{|\sin rt|}{t}\, dt= \sum\limits_{k=0}^{k_0} \int_{0}^{\frac{\pi}{r}} \frac{|\sin rt|}{t+\frac{k\pi}{r}}\, dt = \int_{0}^{\frac{\pi}{r}}\sin rt\cdot \sum\limits_{k=0}^{k_0} \frac{1}{t+\frac{k\pi}{r}}\, dt. $$ Здесь слагаемое при $k=0$ не зависит от $r$: $$ \int_{0}^{\frac{\pi}{r}} \frac{\sin rt}{t}\, dt=\int_{0}^{\pi} \frac{\sin t}{t}\, dt=O(1). $$ Для остальных слагаемых имеем $$ \frac{r}{\pi} \sum\limits_{k=1}^{k_0} \frac{1}{k+1}\le \sum\limits_{k=1}^{k_0} \frac{1}{t+\frac{k\pi}{r}}\le \frac{r}{\pi}\sum\limits_{k=1}^{k_0} \frac{1}{k}. $$Как следствие, получаем $$\sum\limits_{k=1}^{k_0} \frac{1}{t+\frac{k\pi}{r}}= \frac{r}{\pi} \sum\limits_{k=1}^{k_0} \frac{1}{k}  +O(r),\qquad t\in\left[0,\frac{\pi}{r}\right]. $$ Поскольку $k_0=r+O(1),$ то $$ \int_0^{\frac{\pi}{r}} \sin rt \sum\limits_{k=0}^{k_0} \frac{1}{t+\frac{k\pi}{r}}\, dt=\frac{2}{\pi} \sum\limits_{k=1}^{k_0} \frac{1}{k}+O(1)=\frac{2}{\pi}\ln k_0+O(1)=\frac{2}{\pi}\ln r+O(1). $$ Итак, мы показали, что при $r\ge 1$ $$ \int_0^{\pi}\frac{|\sin rt|}{t}\, dt= \frac{2}{\pi} \ln r+O(1). $$ Лемма доказана. В следующей лекции используем ее для доказательства теоремы~\ref{t16-3}.