% Лекции Сергея Борисовича Стечкина
% Внесены исправления С.В. Конягина и И.Г. Царькова, версия 24.02.2009
% Внесены исправления Н.И. Черныха, версия 16.07.2009
% Внесена грамматическая и ТеХ-правка М.Дейкаловой, версия 05.08.09

\chapter{Аппроксимативная компактность. Приближение в \boldmath $C$}

\section{Непрерывность метрической проекции}

Пусть {$X$ -- метрическое пространство,} $M\subset X$,\  $x\in X$,
$Y(x)$~-- множество наилучших элементов для $x$ {в $M$}. Если $M$~--
чебышевское множество, т.\,е. для любого $x$ существует и притом единственный
наилучший элемент {$y(x)$}, и если $M$ еще и ограниченно компактное множество,
то метрическая проекция
$$
x\longmapsto Y(x)=\{y(x)\}
$$
непрерывна (было доказано).

Пусть $X$~-- банахово пространство, $M\subset X$. Множество $M$
называется \textit{аппроксимативно компактным} (а.\,к.), если для каждого $x\in
X$ любая минимизирующая последовательность элементов $\{y_n\}$,
т.\,е. такая, что $y_n\in M$ и $\|x-y_n\|\to E(x,M)_X$~ ($n\to\infty)$, содержит
подпоследовательность, сходящуюся к элементу из $M$.

Если множество $M$ а.\,к. и $Y(x)$ состоит из одной точки, то всякая
минимизирующая последовательность $\{y_n\}$ сходится к этой точке.
Отметим также, что любое аппроксимативно компактное множество
замкнуто.

\begin{Example}
Если $H$~-- бесконечномерное гильбертово пространство, то единичная
сфера $S_1=\{x\colon\|x\|=1\}$ не является аппроксимативно
компактной, а множество $M=S_1\cup\{0\}$ аппроксимативно компактно.
\begin{figure}[ht]
\begin{center}
\includegraphics[width=0.35\textwidth]{pict/pict11-1.eps}
\end{center}
 \bigskip
 \refstepcounter{ris}\label{r11-1}

 \centerline{Рис.~\theris}
\end{figure}
\end{Example}

Выясним, как устроено аппроксимативно компактное множество.

\begin{enumerate}
\item Если $M$~-- а.\,к., то $Y(x)\ne \varnothing$ для всех $x\in X$.
\item Для любого $x\in X$ множество $Y(x)$ в а.\,к. множестве $M$ является
компактным, так как всякая последовательность из $Y(x)$ является
минимизирующей и, следовательно, содержит сходящуюся подпоследовательность.
\end{enumerate}

\begin{teo}[И.\,Зингер]
Пусть $M$~-- а.\,к. множество и пусть для элемента $x_0$ существует единственный
наилучший элемент $y(x_0)$. Тогда {метрическая проекция
$Y(x)$} непрерывна в точке $x_0$, т.\,е. для любой последовательности
$\{x_n\}\to x_0$~ $(n\to\infty)$ и для любых $y_n\in Y(x_n)$ имеем
$y_n\to y(x_0)$~ $(n\to\infty)$.
\end{teo}

Действительно, пусть последовательность $\{x_n\}$
сходится к $x_0$. Возьмем любое $y_n\in Y(x_n)$. Тогда $\{y_n\}$
будет минимизирующей последовательностью для $x_0$, ибо
$$
\|x_0-y_n\| = \|x_0-x_n+x_n-y_n\| \le \|x_0-x_n\| + \|x_n-y_n\| =
\|x_0-x_n\| + E(x_n,M) \to E(x_0,M)
$$
при $n\to\infty$; следовательно, $y_n\to y(x_0)$~ ($n\to\infty)$.

\begin{Corollary}
Если $M$~-- чебышевское а.\,к. множество, то метрическая проекция
$y(x)$ непрерывна в любой точке пространства $X$.
\end{Corollary}

\begin{Remark}
Отметим еще один важный случай непрерывности отображения $Y(x)$. Пусть
$X$~-- банахово пространство, $M$~-- гиперплоскость. Без
ограничения общности можно считать, что $M$ содержит $0$.
{Пусть еще задан элемент $x_0 \in X \setminus M$.}
В этом
случае любой элемент $x\in X$ представим единственным образом в виде
$$
x = y+\alpha x_0,\qquad y\in M,\qquad \alpha\in \mathbb R.
$$
{Отсюда следует, что $Y(x)=y+\alpha Y(x_0)$. Поэтому если} для некоторого
{$x_0\in X,$} {$x_0\notin M$}, наилучший элемент
{$Y(x_0)$} существует и единственен, то и для любого {$x\in X$} элемент $Y(x)$
существует, единственен и непрерывно зависит от $x$. Здесь
непрерывность имеет место просто ввиду линейности операции проектирования:
$$
Y(\alpha x_1+\beta x_2)=\alpha Y(x_1)+\beta Y(x_2),\qquad \|Y(x)\| \le 2\|x\|.
$$
\end{Remark}

Существуют тривиальные чебышевские множества, являющиеся к тому же
аппроксимативно компактными. Это все пространство и одноточечные множества.

\begin{Remark}
Существуют такие бесконечномерные банаховы пространства, в которых нет
ни одного нетривиального чебышевского подпространства (пример
А.\,Л.\,Гаркави).
\end{Remark}

\begin{task}
Верно ли, что во всяком сепарабельном пространстве размерности больше 1
существуют нетривиальные чебышевские подпространства?
\end{task}

В $C[0,1]$ есть нетривиальные чебышевские подпространства, например,
подпространство констант.

В $L[0,1]$ тоже есть нетривиальные чебышевские подпространства.

\begin{Example}
Рассмотрим $L(0,1)$, $\|f\|=\ds\int_0^1 |f(t)|\,dt$,
$$
M=\left\{\varphi\in L(0,1)\colon \varphi(t)=0\quad\forall\, t\in [0,1/2]\right\}.
$$

 \bigskip
\begin{figure}[ht]
\begin{center}
\includegraphics{pict/pict11-2.eps}
\end{center}
 \bigskip
 \refstepcounter{ris}\label{r11-2}

 \centerline{Рис.~\theris}
 \bigskip
\end{figure}

\noindent На рис.~\ref{r11-2}\ \ $\varphi^*$~-- единственный наилучший элемент
для $f \in L(0,1).$
\end{Example}

\begin{task}
Докажите, что любое конечномерное пространство положительной
размерности содержит чебышевское гиперподпространство.
\end{task}

\begin{Remark}
В любом конечномерном пространстве существует чебышевское подпространство
любой меньшей размерности (В.\,А.\,Залгаллер).
\end{Remark}


\section{Приближения в пространстве $C_{2\pi}$}

Операция нахождения наилучшего элемента довольно сложная. Поэтому
встает вопрос об оценке наилучшего приближения $E(f,M)$. Обычно
верхнюю оценку $E(f,M)$ получают из неравенства
$E(f,M) \le \|f-\varphi\|$, выбрав подходящую функцию $\varphi\in
M$.

Приведем несколько примеров оценки для случая {$C_{2\pi}$}~--
пространства непрерывных $2\pi$-периодических функций и {$M=\mathcal T_n$}~--
множества тригонометрических полиномов степени не выше $n$.

\begin{enumerate}
\item Пусть функция {$f\in C_{2\pi}$} и $\sum\limits_{k=0}^\infty A_k(x)$~--
ее ряд Фурье, где
$$
A_0(x)=\frac{a_0}{2},\qquad A_k(x)=a_k\cos kx+b_k\sin kx.
$$
Известно, что наилучшее приближение в смысле среднего квадратичного
осуществляется частичной суммой $s_n$ ряда Фурье:
$$
\sum\limits_{k=n+1}^{\infty} (a_k^2+b_k^2) = \frac{1}{\pi} \int_{-\pi}^{\pi}
(f-s_n)^2\, dx\le \frac{1}{\pi} \int_{-\pi}^{\pi}(f-t_n)^2\, dx\le
2\|f-t_n\|^2_C,
$$
где $t_n\in \mathcal{T}_n$~-- произвольный тригонометрический полином. Следовательно,
$$
E(f,\mathcal T_n)_C\ge \frac{1}{\sqrt{2}}\left( \sum\limits_{k=n+1}^{\infty}
(a_k^2+b_k^2) \right)^{1/2}.
$$

\item Для сумм Валле Пуссена $\sigma_{2n,n}(x)$, используя неравенство Лебега, получаем
$$
E(f,\mathcal T_{2n})_C\le \|\sigma_{2n,n}(x)-f\|_C\le \frac{2(2n+1)}{2n-n+1}
E(f,\mathcal{T}_n)_C\le 4E(f,\mathcal{T}_n)_C.
$$
Так что, если знаем приближение функции с помощью сумм Валле Пуссена
(а это~-- задача простая, так как $\sigma_{2n,n}$~-- линейный оператор), то
можем оценить наилучшие приближения и сверху, и снизу.

\item Для сумм Фурье имеется оценка
$$
\|f-s_n\|_C\le \left\{ \frac{4}{\pi^2} \ln n+O(1)\right\} E(f,\mathcal{T}_n)_C.
$$
\end{enumerate}


\section{Приближение рациональными дробями}

Пусть $R_{m,n}=R_{m,n}[a,b]$~-- множество всех алгебраических дробей
вида $R(x)=P(x)/Q(x)$, $\deg P\le m$, $\deg Q\le n,$
определенных всюду на отрезке $[a,b]$.

Будем приближать функцию $f\in C[a,b]$ рациональными дробями $R\in
R_{m,n}$. Оценим величину наилучшего приближения
$$
\inf_{R\in R_{m,n}} \|f-R\|_C=\rho_{m,n}(f).
$$

Будем предполагать, что дробь несократима и пусть
$$
\deg P=m-\mu,\qquad \deg Q=n-\nu.
$$
Считаем, что $R(x)$ непрерывна на $[a,b]$, т.\,е. полюсы $R(x)$ расположены
вне $[a,b]$. Пусть $a\le x_1< x_2<\ldots <x_N \le b$ и
$f(x_k)-R(x_k)=\lambda_k\ (k=1,\ldots,N).$ Если $\sign\lambda_k
\cdot \sign\lambda_{k+1}=-1$,~ $k=1,\ldots,N-1$ (т.\,е. знаки $\lambda_k$ чередуются),
то будем говорить, что конечная последовательность $\{x_k\}$ образует
валле пуссеновский альтернанс длины $N$ {для разности $f-R$} (см. рис.~\ref{r11-3} с $N=5$).

\begin{teo}[Ш.\,Ж.\,Валле Пуссен]
Пусть $f-R$ имеет валле пуссеновский альтернанс длины $N=m+n-\min
\{\mu,\nu\}+2.$ Тогда $\rho=:\rho_{m,n}(f)\ge \min\limits_k|\lambda_k|.$
\end{teo}

 \bigskip
\begin{figure}[ht]
\begin{center}
\includegraphics{pict/pict11-3.eps}
\end{center}
 \bigskip
 \refstepcounter{ris}\label{r11-3}

 \centerline{Рис.~\theris}
 \bigskip
\end{figure}



\begin{proof}
Предположим, что утверждение неверно, т.\,е. существует рациональная функция
$r = p/q \in R_{m,n}$ такая, что
$$
\|r-f\|_C<\min_{k}|\lambda_k|.
$$

Рассмотрим разность $R(x)-r(x)$ в точках $x=x_k$:
$$
R(x_k)-r(x_k)=R(x_k)-f(x_k)+f(x_k)-r(x_k)=-\lambda_k+(f(x_k)-r(x_k)).
$$
Ясно, что тогда должно быть
$$
\sign (R(x_k)-r(x_k))=\sign (R(x_k)-f(x_k))=-\sign \lambda_k.
$$
Обозначим $\Delta(x)=R(x)-r(x).$ Тогда $\Delta(x)$ принимает в точках
$x_k$ значения с чередующимися знаками и, значит, имеет между ними по
крайней мере $N-1$ нулей. Но
$$
\Delta(x)=R(x)-r(x)=\frac{P(x)}{Q(x)}-\frac{p(x)}{q(x)}=
\frac{Pq-Qp}{Qq},
$$
$$
\deg Pq\le m-\mu +n,\qquad \deg Qp\le m+n-\nu.
$$
Так как по условию
$$
N-1=m+n-\min(\mu,\nu)+1>m+n-\min(\mu,\nu),
$$
то число $N-1$ нулей числителя больше степени числителя, противоречие. Теорема доказана.
\end{proof}


\section{Системы Чебышева}

Пусть $C[a,b]$~-- пространство вещественных непрерывных  функций.

Система $(\varphi)$ функций $\varphi_1(x),\ldots,\varphi_n(x)$ из $C[a,b]$
называется {\it системой Чебышева} или системой Чебышева порядка $n$
на отрезке $[a,b]$, если для любых различных точек $x_1,x_2,\ldots,x_n\in [a,b]$
определитель $\mathcal D(x_1,\ldots,x_n)=\det(\varphi_i(x_k))$
отличен от нуля.

В частности, при $n=1$ функция $\varphi_1$ не обращается в нуль на $[a,b]$, т.\,е. сохраняет знак.

Аналогично определяются системы Чебышева в пространстве $C(K)$ для
произвольного компакта $K$.

Отметим некоторые свойства систем Чебышева.

1. Системы Чебышева, и только они, являются интерполяционными системами,
т.\,е. они всегда имеют однозначное решение задачи Лагранжа.

Напомним, что задача Лагранжа заключается в том, чтобы для заданной системы
узлов $\{x_k\}$ и значений $\{y_k\}$ найти полином
$\varphi(x)=\sum\limits_{i=1}^n a_i \varphi_i(x)$ такой, что
$\sum\limits_{i=1}^n a_i \varphi_i(x_k)=y_k$ ($k=1,\ldots,n$).

2. Всякая система Чебышева на отрезке линейно независима, так как
любой нетривиальный полином по системе Чебышева порядка $n$ имеет не более
$n-1$ нулей.

\begin{ex}
Если дана система Чебышева на $[a,b]$, то она конечно будет системой Чебышева
и на любом собственном подотрезке. Покажите, что обратное неверно.
\end{ex}

3. Определитель
%Для произвольного множества $\Sigma=\{\xi_i\}_{i=1,\ldots,n-1}\subset [a,b]$
%можно построить
%нетривиальный полином $\varphi$ по системе Чебышева на этом отрезке,
%множество нулей которого совпадает с $\Sigma$.
%Для этого можно взять
$$
{\cal D}(\xi)={\cal D}(\xi_0,\xi_1,\ldots,\xi_{n-1})=\left|
\begin{array}{cccc}
\varphi_1(\xi) &  \varphi_2(\xi) & \ldots &  \varphi_n(\xi)\\
\varphi_1(\xi_1) & \ldots  & \ldots &  \varphi_n(\xi_1)\\
\ldots & \ldots & \ldots & \ldots\\
\varphi_1(\xi_{n-1}) &  \ldots & \ldots &  \varphi_n(\xi_{n-1})
\end{array}
\right|
$$
сохраняет определенный знак на множестве ${\cal M}=\{\xi:\ \xi_0<\ldots<\xi_{n-1}\}.$
Это следует из того, что функция ${\cal D}(\xi)$ непрерывна и отлична от нуля
на ${\cal M},$ и что для любых $\xi,\xi'\in {\cal M}$
набор $\xi$ можно непрерывным образом перевести в набор $\xi',$
оставаясь в ${\cal M}.$

4. Для произвольного множества $\Sigma=\{\xi_i\}_{i=1,\ldots,n-1}\subset [a,b],\
\xi_1<\ldots<\xi_{n-1},$ существует полином $\varphi$ по
системе Чебышева на этом отрезке, множество
нулей которого совпадает с $\Sigma$ и который при переходе
через нули меняет знак. Для этого можно взять $\varphi(x)=A{\cal
D}(x,\xi_1,\ldots,\xi_{n-1}).$ Действительно, пусть $A=1,\
\sigma=\sign {\cal D}(\xi),\ \xi\in {\cal M},\ i\in \{1,\ldots,n-1\},\
a<\xi_i<b,\ x'$ и $x''$ -- точки, достаточно близкие к $\xi_i,$ и такие,
что $x'<\xi_i<x''.$ Тогда $\varphi(x')=(-1)^{i-1}{\cal D}(\xi_1,\ldots,\xi_{i-1},x',\xi_i,
\ldots,\xi_{n-1}).$ Так как $(\xi_1,\ldots,\xi_{i-1},x',\xi_i,\ldots,\xi_{n-1})\in {\cal
M},$ то $\sign\varphi(x')=(-1)^{i-1}\sigma.$ Аналогично, $\sign\varphi(x'')=(-1)^{i}\sigma.$
Следовательно, $\varphi$ меняет знак при переходе через $x_i.$

5. Существует полином по чебышевской системе на $[a,b]$,
сохраняющий знак на этом отрезке, оставаясь строго положительным или отрицательным.

В отличие от алгебраических многочленов (по системе $1, x, \ldots, x^{n-1}$),
в общем случае этот факт далеко не тривиальный. Вначале построим
полином $P_0(x)$, неотрицательный на $[a,b]$.
 Это можно сделать, выбирая $P_0(x)$ как равномерный предел сходящейся
подпоследовательности полиномов $\pm\dfrac{\varphi(x)}{\|\varphi\|_{C[a,b]}}$ при условии
$\xi=(\xi_1,\ldots,\xi_{n-1})\to(a,a,\ldots,a)$.
При подходящем выбранном знаке допредельные полиномы будут положительными на
$(\xi_{n-1},b]$ и, значит, $P_0(x)$ будет неотрицательным на $[a,b]$.
Пусть $x_1,\ldots,x_r$~--- нули полинома $P_0$, тогда $r\le n-1$.
Возьмем
произвольным образом точки $x_{r+1},\ldots,x_{n}$, отличные от $x_1,\ldots,x_r$ и друг
от друга. В силу интерполяционного свойства системы $(\varphi)$ найдется полином $Q$ по
этой системе такой, что $Q(x_1)=\ldots=Q(x_{n})=1$. Обозначим $M=\|Q\|_{C[a,b]}$.
Выберем $\epsilon>0$ так, чтобы для всех $i=1,\ldots,n$ и для всех $x$ таких, что
$|x-x_i|<\epsilon$, выполнялось неравенство $Q(x)>0$. Определим множество
$$
E = \{x\in[a,b]\colon \exists\ i=1,\ldots,n-1\quad |x-x_i|<\epsilon\}.
$$
Тогда $\delta:=\min_{x\in [a,b]\backslash E}P_0(x) > 0$.
Положим $P(x) = P_0(x) + \dfrac\delta{2M} Q(x)$.
Утверждается, что $P(x)>0$ для всех $x\in[a,b]$. Действительно, если $x\in E$,
то $P_0(x)\ge 0$, $Q(x)>0$. Если же $x\in [a,b]\backslash E$, то $P_0(x)\ge\delta$,
$\dfrac\delta{2M}Q(x)\ge-\dfrac\delta2$.





%%%% Конец %%%%

\section{Приближение непрерывных функций посредством \\ полиномов по системе Чебышева}

\begin{teo}
Если система $(\varphi)$, состоящая из $n$ {непрерывных линейно}
{независимых} функций, не есть система Чебышева, то эта система порождает
нечебышевское подпространство, т.\,е. существует функция $f$ такая,
что для нее есть по крайней мере два наилучших полинома $\varphi_1^*$ и
$\varphi_2^*$ по этой системе.
\end{teo}

\begin{proof}
В cамом деле, существуют точки $x_1< x_2< \ldots< x_n$ на $[a,b]$ такие, что
$\det (\varphi_i(x_k))=0$, т.\,е. строки и столбцы этого определителя линейно
зависимы. Значит, найдутся $c_i$,~ $\sum\limits_{i=1}^n c_i^2\ne 0$, такие, что
 \begin{equation}\label{f11-1}
 \sum\limits_{i=1}^n c_i \varphi_i(x_k)= 0,\qquad k=1,\ldots,n,
 \end{equation}
и найдутся $d_k$,~ $\sum\limits_{k=1}^n d_k^2 \ne 0$, такие, что
 \begin{equation}\label{f11-2}
 \sum\limits_{k=1}^n d_k \varphi_i(x_k)=0,\qquad i=1,\ldots,n.
 \end{equation}
Значит, для любого полинома {$\varphi=\sum\limits_{i=1}^n a_i \varphi_i$} будет
 \begin{equation}\label{f11-3}
 \sum\limits_{k=1}^n d_k\varphi(x_k)=0,\qquad k=1,\ldots,n.
 \end{equation}
Построим функцию $f$ следующим образом: $f(x_k)=\sign d_k$ при $k=1,\ldots,n$
(мы считаем, что $\sign 0=0$), $f$ линейна на отрезках
$[x_k,x_{k+1}]$ и постоянна на отрезках $[a,x_1]$ и $[x_n,b]$. По построению {$\|f\|_C=1$}.

Из \eqref{f11-3} следует, что для любого полинома $\varphi$ найдется такое число
$k$, что $d_k\ne0$ и $d_k\varphi(x_k)\le0$. Отсюда
$\|f-\varphi\|_C \ge |f(x_k)-\varphi(x_k)| \ge 1 = \|f-0\|_C$, т.\,е.
тождественный нуль является полиномом наилучшего приближения по системе
($\varphi$) для построенной функции $f(x)$.

{В силу \eqref{f11-1} существует полином $\varphi_0(x) \not\equiv 0$ такой,
что $\varphi_0(x_k)=0$~ ($k=1,\ldots,n$).} Возьмем
$\varphi_0(x)$ {и положим}
$$
f_\eps(x) = f(x)\cdot(1-|\eps\varphi_0(x)|),
$$
где $\varepsilon>0$ выбрано так, что $\|\eps\varphi_0\|_C < 1$.
Для этой функции {$f_\eps(x_k)=\sign{d_k},$
$\|f_\eps\|_C=1$ и поэтому}, рассуждая как в случае $f(x),$ получаем, что
тождественный нуль является ее
полиномом наилучшего приближения. Кроме того, для любой точки $x\in [a,b]$ имеем
$$
|f_\eps(x)+\eps\varphi_0(x)| \le |f(x)|\cdot(1-|\eps\varphi_0(x)|) + |\eps\varphi_0(x)| \le
(1-|\eps\varphi_0(x)|) + |\eps\varphi_0(x)| = 1,
$$
поэтому $-\eps\varphi_0$ также будет полиномом наилучшего приближения для $f_\eps$.

Итак, $f_\eps$ имеет по крайней мере два наилучших полинома и, значит,
$(\varphi)$ порождает нечебышевское подпространство. Теорема доказана.
\end{proof}

\begin{Remark}
Аналогичное утверждение может быть доказано при замене отрезка
$[a,b]$ на произвольный компакт~$K$.
\end{Remark}
