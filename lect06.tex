% Лекции Сергея Борисовича Стечкина
% ??? Внесены исправления С.А. Теляковского, версия 1.04.2009
% Внесены исправления Н.И.Черныха, версия 29.07.2009
% Внесена грамматическая и ТеХ-правка М.Дейкаловой, версия 05.08.09


\chapter{Приближение непрерывных функций суммами Фурье. \\
Суммы Валле Пуссена} \label{ch6} % Лекция 6

\section{Приближение суммами Фурье и Фейера в $C_{2\pi}$}


Пусть $f$~-- $2\pi$-периодическая непрерывная функция, $s_n(f)$~-- ее сумма Фурье
порядка~$n$.

1) Согласно теореме~\ref{l5-ln} нормы $\|S_n\|_C$ имеют порядок $\ln n$ при
$n\to\infty$.

\begin{Remark}
На самом деле справедлива более точная оценка
$$
\|S_n\|_C=\frac{4}{\pi^2}\ln n+c+O\left( \frac{1}{n}\right), \qquad n\to\infty,
$$
где $c$~-- некоторая абсолютная постоянная.
\end{Remark}

2) Важное свойство приближения суммами Фурье выражает неравенство Лебега
$$
\|f-s_n(f)\|_C\le (\|S_n\|_C+1) E_n(f)_C,
$$
где $E_n(f)_C$ -- наилучшее приближение функции $f\in C_{2\pi}$
тригонометрическими полиномами порядка $n.$

Отметим другие аппроксимативные свойства сумм Фурье.

3а) Аппроксимативный признак равномерной сходимости рядов Фурье. Известно, что равномерная
сходимость рядов Фурье имеет место не для всех непрерывных
функций. Нас интересуют достаточные условия равномерной сходимости.
Из свойства~2) следует, что если
$$
E_n(f)_C\ln n \to 0\qquad (n\to \infty),
$$
т.\,е. если
$$
E_n(f)_C=o\left( \frac{1}{\ln n}\right) \qquad (n\to \infty),
$$
то
$$
\|f-s_n(f)\|_C\to 0 \qquad (n\to \infty).
$$
И вообще, если $E_n(f)_C=\varphi(n),$
то мы знаем скорость, с которой суммы Фурье приближают функцию:
$$
\|f-s_n(f)\|_C=O(\varphi(n)\ln n)\qquad (n\to \infty).
$$
Так что для функций, плохих в смысле скорости убывания наилучших приближений, суммы
Фурье плохо приближают функцию или вообще расходятся, а для хороших~-- хорошо приближают.
Сравнивая скорость приближения двух функций, видим, что
суммами Фурье, вообще говоря, лучше приближается та функция, у которой
наилучшие приближения убывают быстрее.

3б) Для любой как угодно быстро убывающей функции $\psi(n)>0$
найдется отличная от тригонометрического полинома функция
$f\in C_{2\pi}$ такая, что
$$
\|f-s_n(f)\|_C=O(\psi(n)).
$$

Докажем это. Зададим произвольную последовательность $\varepsilon_n \downarrow 0,\
\varepsilon_n=O(\psi(n))$ и положим
$a_n=\varepsilon_n-\varepsilon_{n+1}$. Тогда
$\sum\limits_{k=n}^{\infty} a_k=\varepsilon_n$. Построим функцию $f(x)=\sum\limits_{k=1}^{\infty}
a_k \cos kx.$ Для нее
$$
\|f-s_n(f)\|\le \sum\limits_{k=n+1}^{\infty} a_k = \varepsilon_{n+1}
$$
и наше утверждение доказано.




3в) Тригонометрическая система полна в {$C_{2\pi},$ следовательно, для любой функции
$f\in C_{2\pi}$ имеем $E_n(f)_C\to 0$~ ($n\to
\infty$). Поэтому
$$
\|s_n(f)\|_C\le \|f\|_C+(\|{S_n}\|_C+1) E_n(f)_C=O(1)+o(\ln n)=o(\ln n)\qquad (n\to \infty).
$$
Так что для каждой непрерывной функции частичные суммы Фурье растут медленнее,
чем $\ln n$, хотя верхняя грань норм частичных сумм Фурье, взятая
по всему классу непрерывных функций, имеет порядок $\ln n$.

Проведем аналогичный анализ для сумм Фейера.

1) $\|\sigma_n\|_C=1,$ так как оператор положительный.

2) Для любой функции $f\in C_{2\pi}$ суммы Фейера равномерно ее
аппроксимируют, т.\,е.
$$
\|f-\sigma_n(f)\|_C\to 0\qquad (n\to \infty).
$$
Этот факт обычно доказывается в курсе математического анализа.

3) Докажем, что в отличие от сумм Фурье суммы Фейера не могут
приближать непрерывные функции очень быстро.
Именно, имеет место следующее утверждение.

\begin{teo}
Для каждой функции $f\in C_{2\pi}$, отличной от постоянной, существует число
$c=c(f)>0$
такое, что
$$
\|f-\sigma_n(f)\|_C\ge \frac{c(f)}{n}\qquad \forall\ n,
$$
т.\,е. ни одна функция из $C_{2\pi}$, не являющаяся тождественной константой,
не может приближаться суммами Фейера по порядку лучше, чем $\dfrac{1}{n}.$
\end{teo}

В этом смысле суммы Фейера хуже сумм Фурье: хорошие функции они приближают
плохо.

\begin{proof}
Если $f(x)\not\equiv\mathrm{const}$ и
$$
f(x) \sim \frac{a_0}{2}+\sum\limits_{k=1}^{\infty} A_k(x),
$$
то найдется $k_0>0$ такое, что $A_{k_0}(x) \not\equiv 0,$ т.\,е. либо
$a_{k_0}(f)\ne 0,$ либо {$b_{k_0}(f)\ne 0.$}  Пусть для
определенности $a_{k_0}\ne 0.$ {Имеем}
\begin{multline*}
\|f-\sigma_n(f)\|_C=\max\limits_{x} |f(x)-\sigma_n(f,x)|\ge \\ \ge \frac{1}{2\pi} \int_{0}^{2\pi} |f(x)-\sigma_n(f,x)|\, dx \ge \\ \ge
\frac{1}{2\pi} \left| \int_{0}^{2\pi} \{ f(x)-\sigma_n(f,x)\} \cos k_0x\, dx \right|=J.
\end{multline*}
Но $\sigma_n(f,x)=\sum\limits_{k=0}^n \left(1-\dfrac{k}{n+1}\right) A_k$. Поэтому если $n\ge k_0$, то
$$
J=\frac{1}{2} \left| \frac{k_0}{n+1} a_{k_0}
\right|
\ge \frac{1}{2(n+1)} |a_{k_0}|\ge \frac{c(f)}{n}.
$$
Если $n<k_0,$ то $\cos k_0x$ в $\sigma_n$ не входит и $J=\dfrac{1}{2}\,|a_{k_0}(f)|$. Теорема доказана.
\end{proof}

\begin{teo}
Существуют функции $f\in C_{2\pi}$, для которых
$$
\|f-\sigma_n(f)\|_C \asymp \frac{1}{n}, \qquad n\to\infty.
$$
\end{teo}

\begin{proof}
Возьмем $f_1(x)=\sin x$. Тогда $f_1(x)-\sigma_n(x,f_1)=\dfrac{1}{n+1} \sin x$ и
$$
\|f_1-\sigma_n(f_1)\|_C=\frac{1}{n+1}.
$$

Так что существуют непрерывные функции, для которых порядок
приближения суммами Фейера в точности равен $\dfrac{1}{n}.$
\end{proof}

Из теорем~6.1 и 6.2 следует, что никакая непрерывная
функция, кроме констант, не может быть приближена суммами
Фейера порядка $n$ со скоростью большей, чем
$\dfrac{c}{n},$ и эту оценку нельзя улучшить.

Метод приближений, обладающий подобным свойством,
называется насыщенным. Максимально возможная скорость
приближения непрерывных функций насыщенным
методом называется порядком насыщения этого метода (в $C_{2\pi}$).

Таким образом, порядок насыщения сумм Фейера равен $\dfrac{1}{n}.$

Вместе с тем из свойства 3б) сумм Фурье следует, что
скорость приближения ими непрерывных функций может быть как
угодно большой. Следовательно, метод Фурье является не
насыщенным.

Для насыщенного метода приближения множество функций, для
которых скорость приближения этим методом совпадает с
порядком насыщения метода, называется классом насыщения
метода. Если метод не насыщенный, то говорят, что у него
нет класса насыщения. Класс насыщения метода Фейера будет
указан позднее.

\ \

\section{Суммы Валле Пуссена}

Суммы Валле Пуссена хороши и для хороших, и для плохих функций.

Пусть {$f\in C[0,2\pi]$},~ $0\le m\le n$. Суммами Валле Пуссена называются
полиномы
$$
\sigma_{n,m}(f)=\frac{1}{n-m+1} \sum\limits_{k=m}^{n}s_k(f).
$$
В частности, при $m=0$ имеем $\sigma_{n,0}(f)=\sigma_n(f)$~-- суммы Фейера, при $m=n$
имеем $\sigma_{n,n}(f)=s_n(f)$~-- суммы Фурье.

Наиболее интересны аппроксимативные свойства сумм Валле Пуссена когда $m\asymp n$ и $n-m
\asymp n$, т.\,е. $an\le m\le An,$ где $a>0$, $A<1$.

Сумму Валле Пуссена можно представить так:
$$
\sigma_{n,m}(f)=\sum\limits_{k=0}^n \lambda_k^{(n)} A_k(x) \qquad
{(A_k(x)=A_k(x,f)=a_k(f)\cos kx+b_k(f)\sin kx),}
$$
где $\lambda_k^{(n)}=1$ при $k\le m$ и $\lambda_k^{(n)}=\dfrac{n-k+1}{n-m+1}$
при $m\le k\le n.$ Для наглядности удобно коэффициенты $\lambda$
изобразить графически {(см. рис.~\ref{r6-1})}.
\vspace{7mm}

\begin{figure}[ht]
\begin{center}
\includegraphics{pict/pict06-1.eps}
\end{center}
 \bigskip
 \refstepcounter{ris}\label{r6-1}

 \centerline{Рис.~\theris. }
 \bigskip
\end{figure}
\vspace{3mm}   


%%%%%%%%%%%%%%%%%%%%%%%%%%%%%%%%%%%%%%%%%%%%%%%%%%%%%%%%%%
%%%%%%%%%%%%%%%%%%%%%%%%%%%%%%%%%%%%%%%%%%%%%%%%%%%%%%%%%%

Укажем связь между суммами Валле Пуссена и Фейера:
$$
\sigma_{n,m}(f)=\frac{1}{n-m+1}\sum\limits_{k=m}^n s_k(f)=
$$
$$
=\frac{1}{n-m+1} \left\{ \sum\limits_{k=0}^n s_k(f)-
\sum\limits_{k=0}^{m-1} s_k(f) \right\}
=\frac{n+1}{n-m+1}\sigma_n(f)- \frac{m}{n-m+1}\sigma_{m-1}(f).
$$

\newpage

\section{Свойства сумм Валле Пуссена}

%\subsection{Оценка нормы}
{\bf 1. Оценка нормы}
\vspace{3mm}

Имеем
$$
\|\sigma_{n,m}\|_C\le \frac{n+1}{n-m+1}+\frac{m}{n-m+1}=
\frac{n+m+1}{n-m+1},
$$
так как $\|\sigma_{n}\|_C=1.$

Назовем основным участком изменения $m$
$$
an\le m\le An,\qquad 0<a<A<1.
$$
При $m\le An,\ A<1,$ в частности на основном участке
$$
\|\sigma_{n,m}\|_C\le \frac{n+m+1}{n-m+1}
\le \frac{n+An+1}{n-An+1}
<
\frac{1+A}{1-A}.
$$
Следовательно, если нет вырождения сумм Валле Пуссена на суммы Фурье, т.\,е.
если $m\le An,$~ $A<1,$ то нормы сумм Валле Пуссена равномерно
ограничены.

\begin{Remark}
На самом деле
$$
\|\sigma_{n,m}\|_C=\frac{4}{\pi^2} \ln \frac{n+m+1}{n-m+1}+O(1)
$$
-- теорема С.\,М.\,Никольского\footnote{Доказательство дано в лекции 17.}.
\end{Remark}

\vspace{3mm}
{\bf 2. Регулярность}
\vspace{3mm}

Метод приближения $\sigma(f,n)$ называется \textit{регулярным}, если для
любой непрерывной функции $f$ уклонение стремится к нулю:
$$
\|f-\sigma(f,n)\|_C \to 0\qquad (n\to \infty).
$$
Для каких параметров сумм Валле Пуссена это верно?  Для
регулярности необходимо и достаточно, чтобы нормы были
ограничены и была сходимость на всюду плотном множестве. Но
$$
\sigma_{n,m}(\cos kx)=\lambda_k^{(n)}\cos kx \to \cos kx \qquad (n\to \infty),
$$
так как $\lambda_k^{(n)} \to 1$ при $n\to \infty$ и фиксированных $k$ и $m$.

Аналогично для синусов. Итак, на косинусах и синусах имеется равномерная сходимость
сумм Валле Пуссена, а нормы ограничены, если нет вырождения
на суммы Фурье, т.\,е. если $m\le An$,~ $A<1$.

Следовательно, метод Валле Пуссена для $m\le An,~ A<1,$ является регулярным методом.

\begin{Remark}
Это условие и необходимо. Именно, если для некоторых чисел $m_n$ таких, что
$m_n<n$, метод Валле Пуссена $\sigma_{n,m_n}$ регулярен,
то $m_n<An$, начиная с некоторого номера $n,$ где $A<1$.
\end{Remark}

\vspace{3mm}
{\bf 3. Инвариантность}
\vspace{3mm}

При приближении суммами Валле Пуссена $\sigma_{n,m}$ остаются
на месте те функции, у которых часть спектра, начинающаяся с номера,
большего $m$, равна нулю, т.\,е. функции, у которых $a_k=b_k=0$ для всех $k>m$.
Следовательно, инвариантное
подпространство {для $\sigma_{n,m}$}~-- множество тригонометрических полиномов
порядка не выше $m:\ \sigma_{n,m}(t)=t$ для $t\in {\mathcal{T}_m}.$

\vspace{3mm}
{\bf 4. Неравенство Лебега для сумм Валле Пуссена в $C_{2\pi}$}
\vspace{3mm}

В общем виде согласно неравенству Лебега, если для линейного оператора
$P:\ f\to p(f)$ имеем $p(t)=t\ \ \forall\ t\in {\mathcal{T}_m},$ то
$$
\|f-p(f)\|\le (\|P\|+1) E_m(f).
$$

Для сумм Валле Пуссена неравенство Лебега  имеет вид
\vspace{2mm}
$$
\|f-\sigma_{n,m}(f)\|\le (\|\sigma_{n,m}\|+1) E_m(f)
\le \left( \frac{n+m+1}{n-m+1}+1 \right) E_m(f) =\frac{2(n+1)}{n-m+1}
E_m(f).
$$
\vspace{3mm}

\noindent На основном участке изменения $m$, т.\,е. для $an\le m\le An$,~ $0<a<A<1,$ {имеем} {$n-m+1\ge
(1-A)(n+1)$} и
\vspace{3mm}
$$
\|f-\sigma_{n,m}(f)\|\le \frac{2(n+1)}{n-m+1} E_m(f)
\le \frac{2(n+1)}{(1-A)(n+1)} E_m(f)
\le
\frac{2}{1-A}E_{[an]}(f).
$$
Для функций, у которых $E_{[an]}(f)\le RE_n(f),$
если нет вырождения на суммы Фурье и Фейера, имеем
$$
\|f-\sigma_{n,m}(f)\|\le \frac{2}{1-A} E_{[an]} (f) \le CE_n(f).
$$
Таким образом, при наложенных условиях на $m$ и $f$ получен в точности
порядок наилучшего приближения.

\begin{task}
Доказать, что для $m$ таких, что $an\le m\le An$,~ $0<a<A<1$, суммы Валле
Пуссена представляют собой ненасыщенный метод приближения.
\end{task}

Для сумм Фейера нет инвариантных подпространств, кроме констант, поэтому
неравенство Лебега для них можно записать только с $E_0(f)$:
$$
\|f-\sigma_n(f)\|\le 2E_0(f).
$$
Однако имеет место следующее утверждение.

\begin{teo}[С.\,Б.\,Стечкин]
Справедлива оценка
$$
\|f-\sigma_n(f)\|\le \frac{c}{n+1}\sum\limits_{k=0}^n E_k(f).
$$
\end{teo}

Из этой теоремы, которая приводится без доказательства, следует, что если
{для} некоторой функции $f$ имеем $\sum\limits_{k=0}^\infty
E_k(f)<\infty,$ то
$$
\|f-\sigma_n(f)\| {\asymp \frac{1}{n},} \qquad
n\to\infty,
$$
поскольку для любой отличной от константы непрерывной функции $\|f-\sigma_n(f)\|\ge
\dfrac{c(f)}{n}$.
Так будет, например, если $E_n(f)=O\left( \dfrac{1}{n^\gamma}\right)$,~ $\gamma>1$.

Мы получили достаточное условие для того, чтобы функция принадлежала классу насыщения
для метода Фейера (напомним, что в этом случае класс
насыщения составляют функции, которые приближаются суммами Фейера со скоростью $\dfrac{1}{n}$).

Необходимое и достаточное условие принадлежности функции классу насыщения для
сумм Фейера можно выразить в терминах сопряженных функций.

Здесь нам понадобится определение класса {$\mathrm{Lip}\,\alpha$} функций,
удовлетворяющих условию Липшица. Это класс функций $f$ таких, что для любых точек $x'$ и
$x''$ справедливо неравенство
$$
|f(x')-f(x'')|\le M |x'-x''|^{\alpha},
$$
где $0<\alpha\le 1$ и $M$~-- некоторая постоянная.

\begin{teo}[Г.\,Алексич, М.\,Заманский]
Для того чтобы функция $f$ принадлежала классу насыщения для сумм Фейера,
необходимо и достаточно, чтобы выполнялось условие $\widetilde
f\in\mathrm{Lip}\,1,$ где $\widetilde{f}$ -- функция,
сопряженная с $f.$
\end{teo}

Эта теорема приводится без доказательства. Понятие
сопряженной функции $\widetilde{f}$ проще всего ввести в частном
случае, когда $f(x)$ является граничным значением
гармонической внутри единичного круга и непрерывной в замкнутом круге функции $u(z),\ z=re^{ix},\ 0\le r<1.$
В этом случае сопряженная функция $\widetilde{f}(x)$
является граничным значением функции $v(z),\ z=re^{ix},\ r\to 1,$
определяемой из условия
$$
u(z)+iv(z)=f(z),
$$
где $f(z)$ -- функция аналитическая внутри единичного
круга.

В общем случае для суммируемых $2\pi$-периодических функций $f(x)$ сопряженная функция
определяется как
$$
\widetilde{f}(x)=\ds\lim\limits_{\varepsilon\to +0}\dfrac{1}{2\pi}\int_{\varepsilon}^{\pi}
(f(x-t)-f(x+t))\ctg\dfrac{t}{2}\,dt.
$$

Функция $\widetilde{f}$ сопряженная к суммируемой функции
$f,$ может быть не суммируемой, если~(\ref{Fourier})
-- ряд Фурье функции $f,$ то при $\widetilde{f}\in
L^1_{2\pi}$ ее ряд Фурье такой:
$$
\widetilde{f}(x)\sim \sum_{k=1}^{\infty}(-b_k\cos kx+a_k\sin
kx).
$$
