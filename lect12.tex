% Лекции Сергея Борисовича Стечкина
% ??? Внесены исправления С.В. Конягина и И.Г. Царькова, версия 24.02.2009
% Внесены исправления Н.И. Черныха, версия 16.07.2009
% Внесена грамматическая и ТеХ-правка М.Дейкаловой, версия 05.08.09

\chapter{Системы Чебышева. Теорема Хаара}

\section{Чебышевские подпространства в $C(K)$}

Изучаем приближения действительных функций в метрике $C$ посредством
конечномерных подпространств.

Пусть $f\in C[a,b]$,~ $L_n\subset C[a,b]$ --
подпространство, порожденное системой линейно независимых функций
$(\varphi) = \{\varphi_1,\ldots,\varphi_n\}$;
ищем наилучший
полином $\varphi^*(f)$ по системе $(\varphi)$, т.\,е. полином,
на котором достигается $E(f,L_n)_C.$ Было доказано, что если $(\varphi)$
не является системой Чебышева, то найдется элемент из
$C[a,b]$, для которого наилучшее приближение не единственно.

\begin{task}
(Не исследована до конца). На каких компактных множествах $K$ существуют
векторнозначные системы Чебышева, принимающие значения в $\mathbb R^m$?
\end{task}

Для случая $m=1$ имеется теорема Мейерхьюбера: для того, чтобы на
компакте существовала система Чебышева длины больше 1, необходимо и достаточно,
чтобы компакт был гомеоморфен части окружности, а при $n$ четном --
ее собственной части.

\begin{teo}[А.\,Хаар]
Пусть $K$ -- компакт. Для того, чтобы линейно независимая система {$(\varphi)$}
порождала чебышевское подпространство
{в $C(K)$}, необходимо и достаточно, чтобы она была системой
Чебышева на $K$.
\end{teo}

\begin{proof}
Н\;е\;о\;б\;х\;о\;д\;и\;м\;о\;с\;т\;ь~ обсуждалась на прошлой лекции:
именно, было доказано (для $K=[a,b]$), что если система нечебышевская, то
найдется функция, для которой наилучший полином не единственен.

Д\;о\;с\;т\;а\;т\;о\;ч\;н\;о\;с\;т\;ь.~ Пусть $C(K)$~-- множество
непрерывных функций на компакте $K,$ и система функций $(\varphi) =
\{\varphi_1,\ldots,\varphi_n\}$~-- чебышевская. Покажем, что тогда
она является системой единственности. В силу линейной
независимости системы $(\varphi)$ для
мощности $K^\sharp$ компакта справедлива оценка $K^\sharp\ge
n.$ Если $K^{\sharp}=n,$ то пространство, порожденное
системой $(\varphi)$ совпадает с $C(K),$ и значит, является
чебышевским.

Пусть $f\in C(K)$,~ $\varphi(x)=\sum\limits_{k=1}^n a_k \varphi_k(x)$~
($x\in K$)~-- произвольный полином по заданной системе функций.
Рассмотрим \textit{множество точек максимального уклонения}
$$
M(f,\varphi) = \{x\in K\colon \|f-\varphi\|_C=|f(x)-\varphi(x)|\}.
$$
Так как $K$~-- компакт и $f$ и $\varphi$~-- непрерывные функции, то такое множество непусто.

Пусть теперь $\varphi^*$ -- наилучший полином для $f$, т.\,е.
$\|\varphi^*-f\|=E(f,{L_n})_C$. Будем изучать $M(f,\varphi^*)$. Докажем
вспомогательное

\begin{Proposition}\label{card_M}
Пусть $f\in C(K)$, и $\varphi^*$ -- ее наилучший полином по
чебышевской системе порядка $n.$ Тогда множество $M(f,\varphi^*)$ не может
быть слишком маленьким, а именно, должно быть
$$
\card M(f,\varphi^*)\ge n+1.
$$
\end{Proposition}

Докажем это утверждение от противного. Пусть для некоторых $f$ и
$\varphi$ число точек в $M(f,\varphi)$
не превышает $n$. В частности, тогда $f\notin L_n,\ E_n(f,L_n)_C>0.$
Покажем, что такой полином $\varphi$ не является наилучшим.

Построим понижающий полином $h\in (\varphi)$ такой, что $\varphi+\eps h$
дает меньшее уклонение от $f$:
$$ \|f-(\varphi+\eps h)\|<\|f-\varphi\| $$
для некоторого $\eps > 0$. Рассмотрим множество $M(f,\varphi)$. Оно состоит
из точек $x_1,\ldots,x_n$ (если точек $\{x_k\}$ меньше чем $n$,
доказательство такое же). {В силу свойства} {интерполяционности
системы $(\varphi)$ существует} полином $h$ такой, что {для всех $x_k$}
$$
h(x_k) = f(x_k)-\varphi(x_k){(\ne 0)}.
$$
Окружим точки $x_k$ окрестностями $U_k$, в которых функции $h$ и $f-\varphi$
сохраняют знак. Тогда $|f-\varphi-\eps h|<\|f-\varphi\|$ в $U_k$
при малых $\eps>0$. Вне объединения $U_k$ имеем $|f-\varphi|<\|f-\phi\|_C$.
{Значит,} $|f-\varphi-\eps h| < \|f-\varphi\|_C$ при достаточно
малых $\eps$. Утверждение доказано.

Вернемся к доказательству теоремы Хаара (от противного). Пусть есть функция
$f\in C(K)$ и для нее имеется два наилучших полинома $\varphi_1^*$ и
$\varphi_2^*$:
$$
\|f-\varphi_1^*\|=\|f-\varphi_2^*\|=E(f,{L_n})_C=E.
$$
Тогда, так как множество наилучших полиномов выпукло, то при
любом $t\in[0,1]$ для $\varphi = t\varphi_1^* + (1-t)\varphi_2^*$ справедливо
равенство $\|f-\varphi\|=E$. В частности, при $t=\dfrac12$
полином $\widetilde\varphi = \dfrac12\,\varphi_1^*+\dfrac12\,\varphi_2^*$~--
наилучший, и по доказанному утверждению найдутся точки $x_k$,~ $k=1,2,\ldots,n+1$, в которых
$$
\|f-\widetilde\varphi\|=|f(x_k)-\widetilde\varphi(x_k)|=E.
$$
В этих точках для $\varphi_1^*$ и $\varphi_2^*$ должно быть
$$
|f(x_k)-\varphi_1^*(x_k)|=E,\qquad |f(x_k)-\varphi_2^*(x_k)|=E,
$$
причем знаки разностей должны быть одинаковы, т.\,е.
$$
f(x_k)-\varphi_1^*(x_k) = f(x_k)-\varphi_2^*(x_k) = \pm E.
$$
Отсюда следует, что для полинома $h=\varphi_1^*-\varphi_2^*$, не равного
тождественно нулю по предположению,
$$
h(x_k) = \varphi_1^*(x_k)-\varphi_2^*(x_k) = (f(x_k)-\varphi_2^*(x_k))-
(f(x_k)-\varphi_1^*(x_k)) = 0,\qquad k=1,\ldots,n+1.
$$
Таким образом, ненулевой полином по системе Чебышева имеет $n+1$ нуль,
чего не может быть. Значит, $h\equiv 0$~-- противоречие. Теорема
доказана.
\end{proof}

\begin{Remark}
Условие $\card M(f,\varphi^*)\ge n+1$ для наилучшего полинома $\varphi^*$~-- необходимое, но не достаточное.
\end{Remark}

\begin{Example}
В $C[0,1]$ приближаем константами, $n=1$,~ $n+1=2$. Для наилучшей константы
$\varphi^*=\dfrac12$ (см. рис.~\ref{r12-1})
$$
\card M(f,\varphi)\ge n+1=2.
$$

 \bigskip
\begin{figure}[ht]
\begin{center}
\includegraphics{pict/pict12-1.eps}
\end{center}
 \bigskip
 \refstepcounter{ris}\label{r12-1}

 \centerline{Рис.~\theris}
 \bigskip
\end{figure}


%константы
Для любой постоянной функции $\varphi<\dfrac12$ имеется континуум точек
максимального уклонения, но она не есть наилучшая.
\end{Example}


\section{Теорема Чебышева}

Пусть $f$~-- непрерывная на $[a,b]$ функция, $(\varphi)$~-- система Чебышева
и $\varphi=\sum\limits_{k=1}^n a_k \varphi_k(x)$~-- полином по этой
системе.

Рассмотрим множество $M=M(f,\varphi)$ точек максимального уклонения.

Будем говорить, что $M(f,\varphi)$ имеет (чебышевский) альтернанс длины $n+1$,
если существуют точки $x_1, x_2,\ldots,x_{n+1}$ из $M$ такие, что

1) $a\le x_1<x_2<\ldots<x_{n+1}\le b$,

2) знаки разностей $f(x_k)-\varphi(x_k)$ чередуются
($k=1,2,\ldots,n+1$).

\begin{teo}[П.\,Л.\,Чебышев]
Для того, чтобы полином $\varphi$ по чебышевской системе порядка $n$ на отрезке был
наименее уклоняющимся от $f$, необходимо и достаточно, чтобы
множество точек максимального уклонения $M(f,\varphi)$ имело альтернанс длины
по крайней мере $n+1$.
\end{teo}

\begin{proof}
Н\;е\;о\;б\;х\;о\;д\;и\;м\;о\;с\;т\;ь.~ Пусть в $M(f,\varphi)$ не
содержится альтернанс длины $n+1$. Покажем, что тогда найдется полином с
меньшим уклонением.

Пусть $M(f,\varphi) = M_+\cup M_-$, где
$$
M_+=\{x\in M(f,\varphi)\colon f(x)-\varphi(x)>0\},\qquad
M_-=\{x\in M(f,\varphi)\colon f(x)-\varphi(x)<0\}.
$$
{Можно считать также,
что $M_+\ne \varnothing$ и $M_-\ne \varnothing$, так как в противном случае найдется}
{полином $\varphi(x)-\varepsilon P_+(x)$ с меньшим уклонением, где
$P_+(x)>0$ на $[a,b]$,~ $\varepsilon>0$ при} {$M_+\ne \varnothing$,
$\varepsilon<0$ при $M_-\ne \varnothing$ и $|\varepsilon|$ -- достаточно
малое число. Существование полинома} {$P_+(x)$ по чебышевской системе было
доказано ранее.} Пусть также $\eta_1=\min\{x\colon x\in M(f,\varphi)\}$;
без ограничения общности предполагаем, что $\eta_1\in M_+$. Обозначим
$$ \eta_2 = \min\{x>\eta_1\colon x\in M_-\}, $$
$$ \eta_3 = \min\{x>\eta_2\colon x\in M_+\} $$
и т.\,д. Мы получили систему точек $\eta_1<\ldots<\eta_k$; при этом по
{нашим предположениям} {$1<k\le n$}. Далее, положим
$$ \zeta_1 = \max\{x\in M_+\colon x<\eta_2\}, $$
$$ \zeta_2 = \max\{x\in M_-\colon x<\eta_3\}, $$
$$ {\ldots\ldots\ldots\ldots\ldots\ldots\ldots\ldots\ldots} $$
$$ {\zeta_k = \max\{x\in M\} \qquad (\zeta_k\ge\eta_k).} $$


Наконец, на каждом интервале $(\zeta_i,\eta_{i+1})$,~ $i=1,\ldots,k-1$,
возьмем произвольным образом точку $\xi_i$.

Окружим отрезки $[\eta_i,\zeta_i]$, {$i=1,\ldots,k$}, интервалами {$(a_i,b_i)$},
не содержащими точек~$\xi_i$ (см. рис.~\ref{r12-2}). В случае $\eta_1=a$
и/или $\zeta_k=b$ вместо $(a_1,b_1)$,~ $(a_k,b_k)$ возьмем промежутки $[a,b_1)$,
$(a_k,b]$. По доказанному ранее, если $k=n$, то существует
полином $h(x)=\pm{\cal D}(x;\xi_1,\ldots,\xi_{n-1})$,
который в точках $\xi_1,\ldots,\xi_{n-1}$ имеет нули и меняет знаки.

 \bigskip
\begin{figure}[ht]
\begin{center}
\includegraphics{pict/pict12-2.eps}
\end{center}
 \bigskip
 \refstepcounter{ris}\label{r12-2}

 \centerline{Рис.~\theris}
 \bigskip
\end{figure}


Правильно выбрав знак в формуле для $h(x),$ можно считать, что $h(x)>0$ на интервалах
{$(a_{2i-1},b_{2i-1})$} и $h(x)<0$ на интервалах $(a_{2i},b_{2i})$.
{Если же $k<n$, то
недостающие различные точки $\{\xi_i\}_{i=k}^{n-1}$ поместим в случае
четного их числа на интервале $(\xi_1,a_2)$, а в случае нечетности $n-k$
возьмем $\xi_{n-1}=a$ при $a<\eta_1$ или $\xi_{n-1}=b$ при $\zeta_k<b$,
а остальные точки $\{\xi_i\}_{i=k}^{n-2}$ разместим так же на интервале
$(\xi_1,a_2).$ Ясно, что построенный по этим точкам полином $h(x)$
сохраняет свойство $\sign h(x)=\sign(f(x)-\varphi(x))$ при $x\in(a_i,b_i)\cap M,\
(i=1,2,\ldots,k)$, уже доказанное в случае $k=n.$
Осталось построить такой же полином в оставшемся
случае, когда $n-k$ нечетно и $\eta_1=a,$~ $\zeta_k=b.$ В этом случае по уже
выбранным точкам $\{\xi_k\}_{k=1}^{n-2}$ построим два полинома
$h_a(x)$ и $h_b(x),$ зануляющихся в них и в точках $x=a,$~ $x=b,$ соответственно.
Полагая $h=h_a+h_b$, опять получим полином с нужными свойствами.}
Рассуждая далее как в конце доказательства утверждения~\ref{card_M}, видим, что
$\|f-\varphi-\eps h\|<\|f-\varphi\|$ при малых $\eps>0$.

Д\;о\;с\;т\;а\;т\;о\;ч\;н\;о\;с\;т\;ь.~ Предположим, что имеется альтернанс длины
$n+1$. Покажем, что тогда $\varphi$ есть полином, наименее
уклоняющийся от функции $f$. Если это не так и существует еще один полином
$\varphi_1$, который дает меньшее уклонение, то в точках альтернанса
{разность $\varphi_1-\varphi$ имеет чередующиеся знаки (см. рис.~\ref{r12-3}). Следовательно,}
$\varphi_1-\varphi$ имеет $n$ нулей, что
невозможно. Теорема доказана.

 \bigskip
\begin{figure}[ht]
\begin{center}
\includegraphics{pict/pict12-3.eps}
\end{center}
 \bigskip
 \refstepcounter{ris}\label{r12-3}

 \centerline{Рис.~\theris}
 \bigskip
\end{figure}


\end{proof}

\begin{Remark}
В случае, когда система Чебышева есть система многочленов степени не
выше $n$, {чтобы полином $p_n$ наименее уклонялся от функции $f$,}
необходимо и достаточно, чтобы {в $M(f,p_n)$} существовал альтернанс
длины $n+2$.
\end{Remark}


\section{Валле пуссеновский альтернанс}

Пусть $(\varphi)=\{\varphi_k(x)\}_{k=1}^n$}~-- система Чебышева непрерывных функций на отрезке
$[a,b]$, $f\in C[a,b]$ и $F\subset [a,b]$. Обозначим
$$
E(f,(\varphi),F)=\inf \|f-\varphi\|_{C(F)},
$$
где нижняя грань берется по полиномам $\varphi$ по системе
$(\varphi)$. Заметим, что для сужения $f$ на множество $F,$ содержащее не менее $n+1$ точек,
 теорема Чебышева тоже верна: для того чтобы полином $\varphi$
обеспечивал наилучшее приближение к $f$ на $F$, необходимо
и достаточно, чтобы {на этом множестве у $f-\varphi$} был чебышевский
альтернанс длины по крайней мере $n+1$.

Пусть далее $F=F_{n+1}$ -- валле пуссеновский альтернанс длины
$n+1$, т.\,е. $F_{n+1}$ -- конечная последовательность $\{x_k\}\subset [a,b],$\
$x_1<\ldots<x_{n+1},$ такая, что для некоторого полинома
$\varphi_F$
$$
f(x_k)-\varphi_F(x_k)=(-1)^k\sigma\lambda_k,
$$
где
$$
\sigma\in \{1,-1\},\qquad \lambda_k>0,\qquad
k=1,\ldots,n+1.
$$

Будем рассматривать уклонение полинома от функции на произвольном таком множестве
$F_{n+1}\subset [a,b]$. Ясно, что
$$
E(f,(\varphi),[a,b])\ge E(f,(\varphi),F_{n+1}).
$$

Так как $\varphi_F$ -- полином по системе $(\varphi)$, то можно считать, что
на $F_{n+1}$ приближаем функцию $f$,\ {$f(x_k)=(-1)^k\lambda_k$}.
Используя это, постараемся вычислить $E(f,(\varphi),F_{k+1})$ явным образом.

Рассмотрим $\varphi(x)=\sum\limits_{i=1}^n a_i \varphi_i(x)$ и
потребуем, чтобы {на $F_{n+1}$ было}
$$
f(x_k)-\varphi(x_k)=(-1)^k \rho, \qquad k=1,\ldots,n+1.
$$
Это есть линейная система уравнений относительно коэффициентов $a_i$~
($i=1,\ldots,n$) и неизвестного уклонения $\rho$, т.\,е. $n+1$ уравнение с
$n+1$ неизвестным:
$$
(-1)^k \lambda_k=(-1)^k \rho+\sum\limits_{i=1}^n a_i \varphi_i(x_k).
$$
Найдем $\rho.$ Определитель этой системы
$$
\left|
\begin{array}{cccc}
{-1} & \varphi_1(x_1) & \ldots & \varphi_n(x_1) \\
{1} & \ldots & \ldots & \ldots \\
\vdots & & & \\
{(-1)^{n+1}} & \varphi_1(x_{n+1}) & \ldots & \varphi_n(x_{n+1})
\end{array}
\right|=
{-}\sum\limits_{k=1}^{n+1} \mathcal D(x_1,\ldots,x_{k-1},x_{k+1},\ldots,x_{n+1})
$$
{отличен от нуля}, так как считаем, что $x_1< x_2<\ldots< x_{n+1}$
{и, следовательно,} все определители под знаком суммы имеют один знак.
(Действительно, от одного определителя к другому можно перейти
непрерывным изменением набора узлов $\{x_k\}$, причем в этом процессе
определитель не будет обращаться в нуль). Уклонение
$E(f,(\varphi),F_{n+1})=\rho$ определяется по формуле
$$
\rho=\frac{\sum\limits_{k=1}^{n+1}\lambda_k
\mathcal D(x_1,\ldots,x_{k-1},x_{k+1},\ldots,x_{n+1})}
{\sum\limits_{k=1}^{n+1}
\mathcal D(x_1,\ldots,x_{k-1},x_{k+1},\ldots,x_{n+1})}.
$$
Меняя, если нужно, знак у числителя и знаменателя, видим, что $\rho$
есть некое среднее из $\lambda_k$ с положительными весами.
Следовательно, для валле пуссеновского альтернанса $F_{n+1}$ имеем оценку
$$
E(f,(\varphi),[a,b]) \ge \rho\ge \min_{k}\lambda_k,
$$
где $\lambda_k=|f(x_k)-\varphi_F(x_k)|,\ x_k\in F_{n+1},\ k=1,2,\ldots,n. $

\begin{teo}[об очистке]
$$
E(f,(\varphi),[a,b])=\sup_{F_{n+1}} E(f,(\varphi),F_{n+1})=
E(f,(\varphi),F_{n+1}^*),
$$
где $F_{n+1}^*$ -- чебышевский альтернанс для функции $f$
на $[a,b].$
\end{teo}

\begin{proof}
Ясно, что первая величина не меньше второй.
Докажем, что вторая величина не меньше первой.
В качестве $F_{n+1}$ возьмем $F_{n+1}^*=\{x_1^*,\ldots,x_{n+1}^*\}$ -- чебышевский альтернанс
для наилучшего полинома на всем отрезке. Тогда на этом множестве
функция не приближается лучше, чем на всем отрезке, так как в силу теоремы
Чебышева наилучшие полиномы на $[a,b]$ и на $F_{n+1}^*$ совпадают. Теорема доказана.
\end{proof}
