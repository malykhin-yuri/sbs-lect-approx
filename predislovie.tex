\begin{center}

{\Large \bf ПРЕДИСЛОВИЕ}

\end{center}

\ \

Настоящее издание возникло в результате обработки конспекта
лекций профессора С.\,Б.\,Стечкина по теории приближений, прочитанных им как спецкурс на
механико-математическом факультете МГУ им.~М.\,В.\,Ломоносова в
1970--1971 учебном году. Лекции были записаны студенткой
Т.\,В.\,Деминой.

Естественно, что такие студенческие записи
нуждались в доработке, которая была выполнена в последние
годы учениками С.\,Б.\,Стечкина. Лекции 1--4 к печати
готовил Ю.\,Н.\,Субботин, лекции 5--7 готовил С.\,А.\,Теляковский,
лекции 8--10 готовили В.\,И.\,Бердышев, Н.\,Н.\,Холщевникова и И.\,Г.\,Царьков,
лекции 11-13 готовили С.\,В.\,Конягин и И.\,Г.\,Царьков, лекции
14, 15 и частично 16 готовил В.\,А.\,Юдин, лекции 16--20
готовил В.\,В.\,Арестов.

Первоначальную обработку
оригинального конспекта лекций (расшифровка скорописи рукописного
текста, набор формул, исправление явных опечаток,
оформление рисунков) и последующую многократную перепечатку
материала лекций в процессе редактирования и согласования
осуществляли: в Москве -- А.\,И.\,Козко, Ю.\,В.\,Малыхин,
Т.\,В.\,Радославова, Н.\,Н.\,Холщевникова\footnote{Без инициативы и настойчивости
Натальи Николаевны эта книга вряд ли могла бы быть (прим. отв. редактора)};
в Екатеринбурге~--
П.\,Ю.\,Глазырина, М.\,В.\,Дейкалова, А.\,А.\,Кошелев,
Н.\,А.\,Куклин, К.\,С.\,Тихановцева, В.\,В.\,Шевченко.
Особенно большая работа легла на плечи М.\,В.\,Дейкаловой,
Ю.\,В.\,Малыхина, В.\,В.\,Шевченко.

Общее редактирование текста провел Н.\,И.\,Черных при большой поддержке и
помощи С.\,А.\,Теляковского, Н.\,Н.\,Холщевниковой и
Ю.\,В.\,Малыхина.

Различные варианты спецкурса по теории приближений
С.\,Б.\,Стечкин читал систематически, почти каждый учебный
год своей многолетней педагогической деятельности в
Московском и Уральском государственных университетах.
Спецкурсы С.\,Б.\,Стечкина, как и этот предлагаемый сейчас
читателю, отличались оригинальностью в выборе материала и
его изложении. Студенты, слушавшие лекции С.\,Б.\,Стечкина,
и обязательные и специальные курсы, помнят и высоко ценят
исключительное мастерство и артистизм, с которыми они
читались. Лекции С.\,Б.\,Стечкина производили на слушателя
завораживающее впечатление.

К сожалению, передать манеру чтения лекций С.\,Б.\,Стечкина
в изложении конспекта его лекций невозможно, и мы с самого
начала не стремились к этому, понимая безнадежность такой
попытки.

Тем не менее мы считаем, что издание этого спецкурса будет
полезным и интересным как для студентов, так и для
преподавателей.

Все подстрочные замечания сделаны при подготовке этого
издания.

\ \

Июль, 2010 г.

\ \

В.\,В.\,Арестов, В.\,И.\,Бердышев, С.\,В.\,Конягин,
Ю.\,Н.\,Субботин,

 С.\,А.\,Теляковский, Н.\,Н.\,Холщевникова, И.\,Г.\,Царьков,
Н.\,И.\,Черных, В.\,А.\,Юдин.
