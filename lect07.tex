% Лекции Сергея Борисовича Стечкина
% ??? Внесены исправления С.А. Теляковского, версия 1.04.2009
% Внесены исправления Н.И.Черныха, версия 29.07.2009
% Внесена грамматическая и ТеХ-правка М.Дейкаловой, версия 05.08.09

\chapter{Линейные методы суммирования рядов Фурье в $\boldsymbol C_{2\pi}$} % Лекция 7

\section{Определение линейных методов суммирования}

Методы приближения суммами Фурье $s_n$, Фейера $\sigma_n$, Валле Пуссена
$\sigma_{n,m}$~-- это частные случаи линейных методов суммирования рядов Фурье.

{Пусть $f\in L(-\pi,\pi)$. Продолжим функцию $f$ на всю числовую ось $2\pi$-периодически,
считая что $f(x+2\pi)=f(x)$ для всех~$x$; в
результате получаем функцию $f\in L_{2\pi}$.} Каждой такой функции можно сопоставить ряд Фурье
$$
f(x)\sim \sum\limits_{m=0}^{\infty} A_m(x).
$$

%\begin{Definition}
\begin{defi}
Пусть задан ряд из элементов~$A_m$ банахова~пространства
\begin{equation}\label{l7-A}
\sum\limits_{m=0}^{\infty} A_m
\end{equation}
(никаких предположений о сходимости ряда не делается). Пусть дана бесконечная матрица чисел
$$
T=(\lambda_m^{(n)})\qquad (m=0,1,\ldots\,;\ \  n=0,1,\ldots).
$$
При помощи этой матрицы для ряда~(\ref{l7-A}) строится последовательность рядов
$$
\tau_n=\sum\limits_{m=0}^{\infty} \lambda_m^{(n)} A_m.
$$
Если все эти ряды сходятся, то говорят, что для ряда~(\ref{l7-A}) определен
(линейный)
метод суммирования $T$, который переводит ряд~(\ref{l7-A}) в последовательность
$\{\tau_n\}$:
$$
A \stackrel{(T)}{\longmapsto} \{\tau_n\}.
$$
\end{defi}
%\end{Definition}

Суммы Фурье $s_n$~-- линейный метод суммирования ряда Фурье, когда матрица $T$ имеет вид
$$
\left(
\begin{array}{ccccc}
1 & 0 & 0 & 0 & \cdots \\
1 & 1 & 0 & 0 & \cdots \\
1 & 1 & 1 & 0 & \cdots \\
\cdots & \cdots & \cdots & \cdots & \cdots
\end{array}
\right);
$$
для сумм Фейера $\sigma_n$  матрица $T$ имеет вид
$$
\left(
\begin{array}{ccccc}
1 & 0 & 0 & 0 & \cdots \\
1 & {1}/{2} & 0 & 0 & \cdots \\
1 & {2}/{3} & {1}/{3} & 0 & \cdots \\
\cdots & \cdots & \cdots & \cdots & \cdots
\end{array}
\right) .
$$
Нетрудно записать матрицу для сумм Валле Пуссена $\sigma_{n,m}.$ В этих трех случаях
$\lambda_m^{(n)}=0,$ начиная с некоторого $m$, т.\,е. $\tau_n$~-- конечные суммы. Такие
матрицы (и соответствующие методы суммирования) называют конечнострочными. В этих
случаях условие сходимости рядов $\tau_n$ выполняются.

Мы будем изучать конечнострочные методы суммирования для рядов Фурье
$$
f\sim \sum\limits_{m=0}^{\infty} A_m(x)=\frac{a_0}{2}+\sum\limits_{m=1}^{\infty}
{(a_m\cos mx+b_m\sin mx)},
$$
$$
T=(\lambda_m^{(n)}) \qquad (n=0,1\ldots;\quad m=0,1,\ldots, M(n)).
$$
Тогда каждой функции $f$ ставится в соответствие последовательность
тригонометрических полиномов
$$
\tau_n(f,x)=\sum\limits_{m=0}^{M(n)} \lambda_m^{(n)} A_m(x).
$$
В теории приближения рассматривают уклонение $f$ от $\tau_n(f,x)$ и исследуют
поведение этого уклонения при $n\to \infty.$

Запишем аналог интеграла Дирихле для произвольного метода суммирования
$$
\tau_n(f,x)=\frac{1}{\pi} \int_{-\pi}^{\pi} K_n(t) f(x+t)\, dt,
$$
где
$$
K_n(t)=\frac{\lambda_0^{(n)}}{2}+\sum\limits_{m=1}^{M(n)} \lambda_m^{(n)}
\cos mt
$$
-- соответствующее ядро метода суммирования. Эта последовательность ядер определяет
метод суммирования. Будем рассматривать в пространстве
$C_{2\pi}$ линейный оператор
$$
{\mathfrak{T}}_n:\ f(x)\longmapsto \tau_n(f,x),
$$
для нормы которого из {$C_{2\pi}$} в {$C_{2\pi}$} имеем
$$
\|{\mathfrak{T}}_n\|_C=\sup_{\|f\|_C\le 1} \max_x
|{\tau}_n(f,x)|=\frac{1}{\pi}\int_{-\pi}^{\pi} |K_n(t)|\, dt.
$$
Действительно, неравенство
$$
\|{\mathfrak{T}}_n\|_C=\sup_{\|f\|_C\le 1} \max_x
\left| \frac{1}{\pi} \int_{-\pi}^{\pi} K_n(t) f(x+t)\, dt\right|
\le  \frac{1}{\pi} \int_{-\pi}^{\pi} |K_n(t)|\, dt
$$
очевидно. Для доказательства равенства достаточно взять в качестве $f$ непрерывные функции, близкие
в $L_{2\pi}$ к $\mathrm{sign}\,K_n(t)$ ($K_n$~--
тригонометрический полином, поэтому такие функции $f$ легко построить (см. рис.~\ref{r7-1})).

%%%%%%%%%%%%%%%%%%%%%%%%%%%%%%%%%%%%%%%%%%%%%%%%%%%%%%%%%%
%%%%%%%%%%%%%%%%%%%%%%%%%%%%%%%%%%%%%%%%%%%%%%%%%%%%%%%%%%

\begin{figure}[ht]
\begin{center}
\includegraphics{pict/pict07-1.eps}
\end{center}
 \bigskip
 \refstepcounter{ris}\label{r7-1}

 \centerline{Рис.~\theris}
 \bigskip
\end{figure}



%%%%%%%%%%%%%%%%%%%%%%%%%%%%%%%%%%%%%%%%%%%%%%%%%%%%%%%%%%
%%%%%%%%%%%%%%%%%%%%%%%%%%%%%%%%%%%%%%%%%%%%%%%%%%%%%%%%%%

{Рассмотрим задачу, когда линейный метод суммирования рядов Фурье является} {регулярным.}
Другими словами, когда
$$
\forall\ f\in C\qquad \|\tau_n(f,x)-f(x)\|_C\to 0\qquad (n\to
\infty)?
$$
Если это верно, то говорят, что метод регулярен или Фурье-регулярен.

Приведем критерий регулярности линейных методов суммирования рядов Фурье.

\begin{teo}
Пусть $T$~-- конечнострочная матрица. Линейный метод суммирования рядов Фурье,
задаваемый матрицей $T$, регулярен тогда и только тогда, когда

$1)$ для некоторого числа $M$ имеет место оценка
$$
\|{\mathfrak{T}}_n\|_C\le M
$$
для всех $n;$

$2)$ ${\mathfrak{T}}_n(\cos kx)\stackrel{C}{\longrightarrow} \cos kx$,
${\mathfrak{T}}_n (\sin kx)
\stackrel{C}{\longrightarrow} \sin kx$ {при $n\to \infty$} для
всех $k$ равномерно по~$x.$
\end{teo}

Действительно, это есть критерий сходимости для линейных операторов: ограниченность
норм в совокупности и сходимость на плотном множестве
(здесь -- на множестве всех тригонометрических полиномов).

Условие 2 можно переписать так:
$$
{\mathfrak{T}}_n(\cos mx) = \lambda_m^{(n)} \cos mx \to \cos
mx\qquad (n\to \infty),
$$
$$
{\mathfrak{T}}_n(\sin mx) = \lambda_m^{(n)} \sin mx \to \sin mx \qquad (n\to \infty),
$$
так что для выполнения условия~2 необходимо и достаточно, чтобы $\lambda_m^{(n)}\to 1$~
($n\to \infty$) для каждого фиксированного $m$. Условие
2 всегда легко проверить.

Что касается условия 1, то, так как
$$
\|{\mathfrak{T}}_n\|_C=\frac{1}{\pi} \|K_n\|_L=
\frac{1}{\pi} \int_{-\pi}^{\pi} |K_n(t)|\, dt,
$$
надо уметь оценивать $\|K_n\|_L$.

Предположим, что коэффициенты $\lambda_m^{(n)}$ <<снимаются>> с функции
$\varphi(u)$, т.\,е. $\lambda_m^{(n)}=\varphi\left( \dfrac{m}{n}
\right)$ (см. {рис.~7.2}).


%%%%%%%%%%%%%%%%%%%%%%%%%%%%%%%%%%%%%%%%%%%%%%%%%%%%%%%%%%
%%%%%%%%%%%%%%%%%%%%%%%%%%%%%%%%%%%%%%%%%%%%%%%%%%%%%%%%%%

\begin{figure}[ht]
\begin{center}
\includegraphics{pict/pict07-2.eps}
\end{center}
 \bigskip
 \refstepcounter{ris}\label{r7-2}

 \centerline{Рис.~\theris}
 \bigskip
\end{figure}




%%%%%%%%%%%%%%%%%%%%%%%%%%%%%%%%%%%%%%%%%%%%%%%%%%%%%%%%%%
%%%%%%%%%%%%%%%%%%%%%%%%%%%%%%%%%%%%%%%%%%%%%%%%%%%%%%%%%%

В этом случае {при $M(n)=n$}
$$
K_n(t)=\frac{\varphi(0)}{2}+\sum\limits_{m=1}^n \varphi\left( \frac{m}{n}
\right) \cos mt=n\left\{
\frac{\varphi(0)}{2}\cdot\frac{1}{n}+\sum\limits_{m=1}^n \varphi\left(
\frac{m}{n} \right) \cos\left(n \cdot \frac{mt}{n}\right)\cdot \frac{1}{n}
\right\}.
$$
Выражение {в фигурных скобках} есть интегральная сумма Римана для интеграла
$$
\int_{0}^1\varphi(u)\cos(n\cdot ut)\, du,
$$
и можно показать, что если функция $\varphi$ достаточно гладкая, то для такой
квадратурной формулы имеет место сходимость. Тогда получим, положив
$nt=y$,
$$
\|{\mathfrak{T}}_n\|_C\approx \frac{2}{\pi} \int_0^{\pi} \left|
\int_0^1 \varphi(u) \cos(u\cdot nt)\ du \right|n\, dt=
\frac{2}{\pi} \int_0^{n\pi} \left|
\int_0^1 \varphi(u) \cos uy\ du \right|\, dy.
$$
Если интеграл
$$
\frac{2}{\pi} \int_0^{\infty} \left|
\int_0^1 \varphi(u) \cos uy \ {du}\right|\, dy
$$
сходится, то нормы $\|{\mathfrak{T}}_n\|_C$ ограничены.
Можно доказать, что если интеграл расходится, то метод {не регулярен}.

\begin{Remark}
Полученная формула является {довольно} точной приближенной формулой для
$\|{\mathfrak{T}}_n\|_C$.
\end{Remark}

Представляют интерес не только регулярные методы (так, метод сумм Фурье {не регулярен}).
Если нет регулярности, то надо исследовать, как
растут нормы $\|{\mathfrak{T}}_n\|_C$.


\section{Приближение линейными методами суммирования\\ рядов Фурье на классах
функций}

Пусть $K$~-- компактный класс в {$C_{2\pi}$} или класс, который становится
компактным после введения нормировки (например, класс функций, у
которых $|f'(x)|\le 1$, -- {не компактный}, но становится компактным
после введения нормировки $f(0)=0$ и замыкания).

Зададим конечнострочный метод суммирования ряда Фурье $\mathfrak{T}_n:\
f(x)\to \tau_n(x,f)$ и рассмотрим верхние
грани
$$
\sup_{f\in K}\|f-{\tau}_n(f)\|_C.
$$
Пусть $K=W^r$~-- класс $2\pi$-периодических функций с непрерывной $r$-й производной,
$|f^{(r)}(x)|\le 1$ и пусть $\tau_n=s_n$.

{Имеет место}
\begin{teo}[А.\,Н.\,Колмогоров]
Для любого $r$
$$
\sup_{f\in W^r}\|f-s_n(f)\|_C=n^{-r}\left\{ \frac{4}{\pi^2}\ln n+O(1)\right\},
\qquad n\to\infty.
$$
\end{teo}

{Эту теорему приводим} без доказательства.

В качестве компакта $K$ можно брать классы $\mathrm{Lip}\,\alpha$
($0<\alpha\le 1$), $H[\omega]$, $W^r$, $A(q)$. Здесь класс $H[\omega]$ для заданного
модуля непрерывности $\omega$ определяется как класс функций $f$,
для модуля непрерывности которых справедлива оценка $\omega (f,\delta)\le
M\omega (\delta)$ с некоторой абсолютной постоянной $M$; $A(q)$~--
класс $2\pi$-периодических функций $f$, аналитических на действительной
прямой и в симметричной относительно действительной прямой полосе
шириной $2q,$ причем $|f(x\pm iq)|\le 1$.

{Задача о погрешности аппроксимации класса $K$ линейным методом $\tau_n$ несколько}
{проще в случае, если $K$ -- класс истокообразно представимых функций, т.\,е.}
{представимых с помощью некоторого ядра $K(t)$ по формуле}
$$
{f(x)=\frac{a_0}{2}+\frac{1}{\pi}\int_{-\pi}^{\pi} K(t) \varphi(x+t)\, dt
\qquad (\varphi \ \bot \ 1, \ \ \text{т.\,е.} \ \ a_0(\varphi)=0).}
$$

Классы $W^r$, $r>0$, хороши {тем, что их функции истокообразно представимы. В этом} {случае в
качестве $K(t)$ можно брать ядро Фавара}
$${\mathfrak{D}_{r}(t)=\sum\limits_{k=1}^{\infty} k^{-r}\cos \left(
kt+\frac{r\pi}{2}\right),}
$$
$$
{f(x)=\frac{a_0}{2}+\frac{1}{\pi}\int_{-\pi}^{\pi} \mathfrak{D}_{r}(t)
f^{(r)}(x+t)\, dt.}
$$
{Для представления всего класса $W^r$ (в том числе и для нецелых $r$) производную
$f^{(r)}$} {можно заменить на произвольную
непрерывную функцию $\varphi$ и получить класс}
$${W^r=\left\{f(x)=
\frac{a_0}{2}+\frac{1}{\pi}\int_{-\pi}^{\pi}\mathfrak{D}_{r}(t) \varphi(x+t)\,
dt:\ |\varphi(t)|\le 1,\quad \varphi\perp 1\right\}.}$$

Подобное представление {(со своим ядром $K(t)$)} имеет место {также} для функций класса
$A(q)$ {и ряда других классов}.

{Для таких классов имеем}
$$
{f(x)=\frac{a_0}{2}+\frac{1}{\pi}\int_{-\pi}^{\pi} K(t)\varphi(x+t)\, dt,}
$$
поэтому
$$
\tau_n(f,x)= \frac{a_0 \lambda_0^{(n)}}{2}+\frac{1}{\pi} \int_{-\pi}^{\pi}
{\tau_n(K,t)}\varphi(x+t)\,
dt,
$$
и если $\lambda_0^{(n)}=1$, то
$$
f(x)-\tau_n(f,x)=\frac{1}{\pi} \int_{-\pi}^{\pi}
\{K(t)-\tau_n(K,t)\} \varphi(x+t)\, dt.
$$
Так как $\varphi \perp 1,$ т.\,е. $\dfrac{1}{\pi}\ds\int_{-\pi}^{\pi} \varphi(x-t)\,
dt=0,$ то из ядра этой {\it свертки} можно вычесть любую
константу:
$$
f(x)-\tau_n(f,x)=\frac{1}{\pi}\int_{-\pi}^\pi
\{ {K(t)-\tau_n(K,t)}-c\}\varphi(x+t)\, dt.
$$
Тогда получим
$$
\sup \|f(x)-\tau_n(f,x)\|_C\le \inf_c \sup_{|\varphi|\le 1}
\left| \frac{1}{\pi} \int_{-\pi}^\pi \{ {K(t)-\tau_n(K,t)}-c\}
\phi(x+t)\, dt \right| \le
$$
$$
\le \inf_c \left\{ \frac{1}{\pi} \int_{-\pi}^\pi
|{K(t)-\tau_n(K,t)}-c|\, dt\right\}=E_0({K-\tau_n(K)})_{{L_{2\pi}}}.
$$
На самом деле для широкого класса ядер $K$
здесь имеет место равенство.

Для классов $\mathrm{Lip}\,\alpha$, {$H[\omega]$} задача труднее, {так как эти
классы не истокообразно представимые.}

Задача изучена для широкого класса методов суммирования $\tau_n$ и для
широкого набора классов $K$ {(не только истокообразно представимых).}


\section{Интерполяционные процессы}

Пусть на основном отрезке $[a,b]$ задана матрица узлов $(x_k^{(n)})$
{($k=0,1,\ldots,n$;~ $n\in \bN$)}. Для каждой функции $f{\in
C[a,b]}$ и любого $n$ можно построить многочлен Лагранжа $p_n(f,x,(x_k^{(n)}))$,
интерполирующий $f$ в узлах $x_k^{(n)}$. Тем самым введен линейный оператор
$P_n\colon f \mapsto p_n(f,x,(x_k^{(n)}))$. В этом случае говорят, что задан интерполяционный процесс.

Рассмотрим задачу: {существует ли регулярный
интерполяционный процесс, т.\,е.} существует ли такая матрица
узлов, что
$$
\forall\ f\in C\qquad \|f-p_n(f)\|_C \to 0\qquad (n\to \infty)?
$$

\begin{teo}[Г.\,Фабер]
Не существует матрицы узлов, для которой  процесс регулярен, именно
$$
\forall\ (x_k^{(n)})\qquad \|P_n\|_C \to \infty\qquad (n\to \infty).
$$
\end{teo}

Докажем более сильное утверждение.

\begin{teo}
 Для любой матрицы узлов {$(x_k^{(n)})_{k=0}^n$~ $(n \in \bN)$}
$$
\|P_n\|_C\ge C\ln n,
$$
где {$P_n\colon f \mapsto p_n(f,x,(x_k^n)_{k=0}^n)$.}
\end{teo}

\begin{lemma}[о тригонометрических полиномах]
Для любых $n$ точек $\theta_k$,~ $0\le \theta_1<\theta_2<\ldots<\theta_n\le \pi$,
существует четный полином
$$
{t_{n-1}}{(\theta)}=\frac{a_0}{2}+\sum\limits_{k=1}^{n-1} a_k\cos k\theta,
$$
обладающий свойствами:
$$
|{t_{n-1}}(\theta_k)|\le 1\qquad (k=1,\ldots,n),
$$
$$
\|{t_{n-1}}\|_C\ge a\ln n,
$$
где $a$~-- некоторая константа.
\end{lemma}

\begin{proof}
Построим такой полином. Для полиномов Фейера
$$
A_n(\theta)=\frac{\cos \theta }{n-1}+\ldots+\frac{\cos(n-1)\theta}{1},
$$
$$
B_n(\theta)=\frac{\cos(n+1) \theta}{1}+\ldots+\frac{\cos
\theta(2n-1)}{n-1},
$$
было доказано {(см. доказательство теоремы~\ref{l5-ln})}, что при любом $n$
$$
\| A_n(\theta)-B_n(\theta)\|_C\le M
$$
и $|A_n(0)|\asymp \ln n$ при $n\to \infty.$

Фиксируем заданные $\theta_1,\ldots,\theta_n$ и рассмотрим фундаментальные
полиномы Лагранжа $C_k(\theta)$ порядка $n-1$, соответствующие
тригонометрической интерполяции:
$$
C_k(\theta)=\frac{\prod\limits_{i\ne k} (\cos \theta - \cos\theta_i)}
 {\prod\limits_{i\ne k} (\cos \theta_k - \cos\theta_i)}.
$$
Тогда
$$
C_k(\theta_i)=
\begin{cases}
0,& i\ne k,\\
1,& i=k.
\end{cases}
$$

Положим
$$
u(\theta)=A_n(2\theta)-\sum\limits_{k=1}^n \{ B_n(\theta_k+\theta)+
B_n(\theta_k-\theta)\} C_k(\theta).
$$
Это~-- тригонометрический полином порядка не выше $3n$.

Легко убедиться, что $a_0(u)=\dfrac{1}{\pi}\ds\int_{-\pi}^{\pi} u(\theta)\, d\theta=0$
{и,} значит, существует такая точка $\alpha,$ что $u(\alpha)=0.$ Зафиксируем эту
точку $\alpha$ и построим четный тригонометрический полином порядка не выше $n-1$
$$
{t_{n-1}}(\theta)=\{A_n(\theta+\alpha)+A_n(\theta-\alpha)\}-\sum\limits_{k=1}^n\{B_n(\theta_k+
\alpha)+B_n(\theta_k-\alpha)\} C_k(\theta).
$$
Имеем
$$
{t_{n-1}}(\theta_k)=\{A_n(\theta_k+\alpha)+A_n(\theta_k-\alpha)\}-\{B_n
(\theta_k+\alpha)+B_n(\theta_k-\alpha)\}.
$$
Следовательно, $|{t_{n-1}}(\theta_k)|\le 2\|A_n-B_n\|_C{\le 2 M},$ {а}
$$
{t_{n-1}}(\alpha)=u(\alpha)+A_n(0)=A_n(0)\asymp \ln n \qquad (n\to\infty),
$$
т.\,е. для {$t_{n-1}$} при некотором положительном $a$ имеем
$$
\|t_{n-1}\|_C\ge a\ln n.
$$
Осталось поделить $t_n(\theta)$ на $2M$, чтобы выполнялось условие $|t_{n-1}(\theta_k)|\le 1.$
Тем самым лемма доказана.
\end{proof}

Переходим к доказательству теоремы~7.4. Оценим норму оператора
$P_{n-1}.$

Поскольку в теореме имеется в виду интерполяционный алгебраический многочлен
с узлами $\{x_k\}$ на отрезке $[a,b]$, <<пересадим>> построенный в
лемме четный тригонометрический полином {$\tau_n(\theta)$} на
отрезок $[a,b]$ {с помощью замены $x=\dfrac{a+b}{2}+\dfrac{b-a}{2}\cos
\theta$}. Тогда точкам $\{x_k\}$ на $[a,b]$ будут соответствовать
точки $\{ y_k\}$,~ $-1\le y_k\le 1$:
$$
x_k=\frac{a+b}{2}+\frac{b-a}{2} y_k
$$
и точки $\theta_k$,~ $0\le \theta_k\le \pi$, {такие, что}
$$
y_k=\cos \theta_k.
$$
При этом получим алгебраический многочлен $p_{n-1}^*(x),$
для которого
$$
{|p_{n-1}^*(x_k)|\le 1 \qquad (k=1,2,\ldots,n), \qquad a\le x_n<x_{n-1}<\ldots<x_1\le b,}
$$
$$
\|p_{n-1}^*\|_{C[a,b]}\ge a\ln n,
$$
Так как существуют непрерывные на
$[a,b]$ функции $f_n$ такие, что $f_n(x_k)=p_{n-1}(x_k),$
{$\|f_n\|_{C[a,b]}\le 1$, то $\|P_{n-1}\|_C^C=\sup\limits_{\|f\|_C\le
1}\|p_{n-1}(x,f,\{x_k\}_{k=0}^n)\|_C\ge \|p_{n-1}^*\|_C,$} и теорема доказана.

\begin{Remark}
Гауссовский квадратурный процесс сходится для любой {$f\in C[0, 2\pi]$}:
$$
\int_a^b f(x)\, dx-\sum\limits_{k=1}^n A_kf(x_k) \to 0\qquad (n\to \infty).
$$
Если по этим же узлам построить интерполяционный процесс, то для некоторой
функции $f$ он не будет сходится, но тем не менее
$$
\int_a^b p_{n-1} (f,x)\, dx \to \int_a^b f(x)\, dx \qquad (n\to \infty).
$$


%%%%%%%%%%%%%%%%%%%%%%%%%%%%%%%%%%%%%%%%%%%%%%%%%%%%%%%
%%%%%%%%%%%%%%%%%%%%%%%%%%%%%%%%%%%%%%%%%%%%%%%%%%%%%%%

\begin{figure}[ht]
\begin{center}
\includegraphics{pict/pict07-3.eps}
\end{center}
 \bigskip
 \refstepcounter{ris}\label{r7-3}

 \centerline{Рис.~\theris. }
 \bigskip
\end{figure}


%%%%%%%%%%%%%%%%%%%%%%%%%%%%%%%%%%%%%%%%%%%%%%%%%%%%%%%
%%%%%%%%%%%%%%%%%%%%%%%%%%%%%%%%%%%%%%%%%%%%%%%%%%%%%%%

Сходящихся интерполяционных процессов нет, а сходящиеся квадратурные процессы
есть (гауссовский). Это поясняет рис.~\ref{r7-3}.
\end{Remark}
