% Лекции Сергея Борисовича Стечкина
% Внесены исправления С.А. Теляковского, версия 1.04.2009
% Внесены исправления Н.И.Черныха, версия 29.07.2009
% Внесена грамматическая и ТеХ-правка М.Дейкаловой, версия 05.08.09

\chapter{Ряды Фурье, суммы Фейера} % Лекция5

\section{Первые сведения о рядах Фурье}

Будем рассматривать периодические функции с периодом $\omega=2\pi$. Пусть {$H_{2\pi}$}~--
некоторый класс $2\pi$-периодических функций,
заданных на всей числовой оси. {Любой} измеримой{,} интегрируемой в смысле Лебега
на основном периоде функции {$f\in
L_{2\pi}=L[0,2\pi)$} можно {поставить в соответствие ее ряд} Фурье
\begin{equation}
\label{Fourier}
f(x)\sim \frac{a_0}{2}+\sum\limits_{k=1}^{\infty}(a_k \cos kx+b_k\sin kx)
= \sum\limits_{k=0}^{\infty}A_k(x),
\end{equation}
где
$$ a_k=\frac{1}{\pi} \int_{-\pi}^{\pi} f(x)\cos kx\,dx, $$
$$ b_k=\frac{1}{\pi} \int_{-\pi}^{\pi} f(x)\sin kx\,dx$$
-- коэффициенты Фурье функции $f$.

Это соответствие формальное, так как для {$f\in L_{2\pi}$} ряд может не сходиться к $f$ ни почти всюду
(пример Колмогорова), ни в смысле сходимости в $L$.

Рассмотрим последовательность частичных сумм ряда Фурье
$$ s_n(f)=s_n(f,x)=\sum\limits_{k=0}^{n} A_k(x)\in {\mathcal{T}_n}. $$
Мы видим, что $s_n(f)$~-- {это} значения линейного оператора {$S_n$}, определенного на
{$L_{2\pi}$}, который ставит функции {$f$} в
соответствие {элемент пространства ${\mathcal{T}_n}$} тригонометрических полиномов порядка $n$:
$$
{ S_n:\ L_{2\pi} \longrightarrow {\mathcal{T}_n},\qquad S_nf=s_n(f). }
$$
Пусть $G,H$ -- линейные нормированные пространства с нормами
$\|\cdot\|_G,\,\|\cdot\|_H,$ содержащиеся в $L_{2\pi}.$
Как обычно, будем обозначать
$$ \|S_n\|_{H}^{G}=\sup_{\|f\|_{H}\le 1} \|s_n(f)\|_{G} $$
норму оператора $S_n$ как оператора из $H$ в $G.$ Если $G=H$, то пишем $\|S_n\|_H^H=\|S_n\|_{H}$
или просто $\|S_n\|,$ если ясно, какое пространство $H$ имеется в виду.

Согласно неравенству Бесселя
$$ \|s_n(f)\|_{L^2}\le \|f\|_{L^2}\qquad \forall\ f\in L^2_{2\pi}. $$
Так как $s_n(f)=f$ для $f\in {\mathcal{T}_n},$ то
$$ \|S_n\|_{L^2} = 1. $$

Пространства {$H=H_{2\pi}$}, в {которых} нормы {$\|S_n\|$} ограничены, обладают
следующим свойством:
$$ \forall\ f\in H\qquad s_n(f) \stackrel{H}{\longrightarrow} f\qquad {(n\rightarrow\infty)} , $$
т.\,е.
$$ \|f-s_n(f)\|_H \to 0\qquad (n\to \infty). $$

Для доказательства достаточно воспользоваться теоремой о поточечной сходимости
последовательности {линейных} операторов {в полном линейном} {нормированном
пространстве $X$:} если

1)~нормы операторов ограничены в совокупности,

2)~последовательность операторов сходится на множестве $\Gamma$, плотном в {$X$ (т.\,е.}
\mbox{$\overline{\Gamma}=X$)},\\
то последовательность
операторов сходится на {$X$}.

В самом деле, в качестве $\Gamma$ можно взять $\mathcal{T}=\bigcup\limits_n {\mathcal{T}_n}$.
Следовательно, справедлива

\begin{teo}
Если множество тригонометрических полиномов плотно в $H$ и константы Лебега
$\|S_n\|_H$ ограничены в совокупности, то имеет место сходимость
$$
s_n(f) \stackrel{H}{\longrightarrow} f\qquad (n\rightarrow\infty) \qquad \forall\ f\in H,
$$
т.\,е.
$$
\|f-s_n(f)\|_H \to 0 \qquad (n\to \infty) \qquad \forall\ f\in H.
$$
\end{teo}

\begin{Remark}
Константы Лебега ограничены в любом пространстве $L^p_{2\pi}$ при $1<p<\infty$, т.\,е.
$$
\|S_n\|_{L^p}\le A_p.
$$
Утверждение этого замечания приводится без доказательства.
\end{Remark}

Рассмотрим пространство $C_{2\pi}$ непрерывных
$2\pi$-периодических функций и обозначим через $L_n$ константы Лебега в
этом пространстве: $\|S_n\|_C=L_n.$

\begin{teo}\label{l5-ln}
При $n\rightarrow \infty$ справедливо порядковое равенство
$$
L_n \asymp \ln n,
$$
т.\,е. существуют такие константы $a$ и $A$, $0<a\le A<\infty$,
и такое натуральное число $n_0$, что
$$
\forall\ n\ge n_0 \qquad a\le \frac{L_n}{\ln n}\le A.
$$
\end{teo}

Будем говорить, что $L_n$ имеет порядок $\ln n$. Отношение <<$\asymp$>> (символ Харди) обладает свойствами симметричности, транзитивности и
рефлексивности.

\begin{proof}
1) Докажем сначала, что $L_n=O(\ln n),$ т.\,е. что найдется такое $A,$ что при некотором
$n_0$ для любого $n\ge n_0$ имеем $L_n\le A\ln n.$ Воспользуемся формулой Дирихле для
частичной суммы ряда Фурье
$$
s_n(f,x)=\frac{1}{\pi}\int_{-\pi}^{\pi} f(x+t) \mathcal D_n(t)\, dt,
$$
\begin{figure}[ht]
\begin{center}
\includegraphics{pict/pict05-1.eps}
\end{center}
 \bigskip
 \refstepcounter{ris}\label{r5-1}

 \centerline{Рис.~\theris. }
 \bigskip
\end{figure}
где
$$
\mathcal D_n(t)=\frac12+\sum\limits_{k=1}^n \cos kt=\frac{\sin\left( n+\frac12
\right)t}{2\sin \frac{t}{2}}
$$
-- ядро Дирихле. График $D_n(t)$ схематично изображен на рис.~\ref{r5-1}. Если $\|f\|_C\le 1,$ т.\,е. $|f(x)|\le 1$ для
всех $x,$ то
$$
|s_n(f,x)|\le \frac{1}{\pi} \int_{-\pi}^{\pi} |\mathcal D_n(t)|\, dt.
$$
Так как для $\mathcal D_n(t)$ при некотором {$C_1>0$} имеют
место оценки
$$
|\mathcal D_n(t)|\le \frac{1}{2\left|\sin \frac{t}{2}\right|}\le \frac{C_{{1}}}{|t|}
\qquad \forall\ t,~ |t|\le \pi,
$$
и
$$
|\mathcal D_n(t)|\le n+\frac{1}{2} \le C_{{1}}n\qquad \forall\ t{,~ n\ge1},
$$
то
\begin{multline*}
|s_n(f,x)|\le \frac{1}{\pi} \int_{-\pi}^{\pi} |\mathcal
D_n(t)|\, dt
{=\frac{2}{\pi}\int_{0}^{\pi} |\mathcal D_n(t)|\,dt}\le \\
\le {\frac{2}{\pi}C_1}\left\{ \int_{0}^{\frac{1}{n}} n\, dt +
\int_{\frac{1}{n}}^{\pi} \frac{dt}{t}\right\}\le C_2\{ 1+\ln n\}\le {A} \ln n\quad
\forall\ n\ge {2},
\end{multline*}
где $A$~-- некоторая абсолютная постоянная.

2) Докажем теперь, что $L_n\ge c\ln n$ для некоторого $c>0,$
т.\,е.
$$
\|{S}_n\|_C=\sup_{\|f\|_C\le 1} \|s_n(f)\|_C\ge c\ln n,
$$
или
$$
\forall\ n{\in\bN}\quad \exists\ f_n\in C,\quad \|f_n\|_C\le 1:\ \|s_n(f_n)\|_C\ge c\ln n.
$$
Для этого достаточно показать, что существуют точка $x_0\in [0, 2\pi)$ и функции {$f_n \in C_{2\pi}$}
 такие, что
$$
\|f_n\|_C\le 1,\qquad |s_n(f_n,x_0)|\ge c\ln n.
$$

Предварительно докажем, что полиномы
$$
d_n(x)=\sum\limits_{k=1}^n \frac{\sin kx}{k}
$$
равномерно ограничены, т.\,е.
$$
\exists\ {B}:\quad \forall\ n\quad \forall\ x\quad |d_n(x)|\le {B}.
$$

Выберем {по $x>0$} натуральное $m=m(x)\asymp \dfrac{1}{x},$ т.\,е. так, что
при $0< x\le \pi$ выполняются неравенства
$$
\frac{A_1}{x}\le m(x)\le\frac{A_2}{x}
$$
с некоторыми числами $0<A_1<A_2<\infty$, не зависящими от $x$.  Тогда для $n>m$
$$
|d_n(x)|=\left| \sum\limits_{k=1}^m \frac{\sin kx}{k}+
\sum\limits_{k=m+1}^n \frac{\sin kx}{k}\right|=|s_1+s_2|\le |s_1|+|s_2|.
$$
Имеем в силу выбора $m$
$$
|s_1|=\left| \sum\limits_{k=1}^m \frac{\sin kx}{k} \right|\le \sum\limits_{k=1}^m
\frac{kx}{k}=mx\le {A_2}.
$$


Для оценки $s_2$ воспользуемся оценкой Абеля $\Big($если $\Big| \sum\limits_1^{p} a_k\Big| \le
A$ при всех $p,\ b_k\ge 0,$ $b_k\downarrow,$ то $\Big|
\sum\limits_{{k=m}}^n a_k b_k \Big|\le 2{Ab_m }\Big)$ {и формулой }
$$
 \widetilde{\Cal D}_n(x)=
\frac{\sin\frac{nx}{2}\cdot\sin\frac{(n+1)x}{2}}{\sin\frac{x}{2}}
$$
для сопряженного ядра Дирихле $\widetilde{\Cal D}_n=\sum\limits_{k=1}^n\sin kx.$
В результате получим
$$
\left| \sum\limits_{k=1}^p \sin kx \right|\le \dfrac{C_1}{|x|},
$$
и
$$
|s_2|\le \frac{C_{{1}}}{|x|}\cdot\frac{1}{m}{\le \frac{C_1}{A_1}}.
$$
Значит, величины $|s_2|$ ограничены абсолютной постоянной. {Если же} {$n\le m(x)\,
\Big(\le \dfrac{A_2}{x} \Big),$ то $s_2=0$ $\Big($как обычно, считаем, что
$\sum\limits_{k=m}^n{\alpha_k}=0$ при $n<m\Big)$
и $d_n(x)$ оценивается так же, как $s_1$:}
$$
{|d_n(x)|\le nx< A_2.}
$$
Поэтому для всех $n$ и всех $x$
$$
|d_n(x)|\le {B},
$$
где $B$ -- некоторая абсолютная постоянная.

Зафиксируем $n\in \bN$ и рассмотрим полиномы Фейера
$$
A_n(x)=\frac{\cos x}{n-1}+\frac{\cos 2x}{n-2}+\cdots+ \frac{\cos(n-1)x}{1}
$$
и
$$
B_n(x)=\frac{\cos (n+1)x}{1}+\frac{\cos (n+2)x}{2}+\cdots+
\frac{\cos(2n-1)x}{n-1}.
$$
Построим функцию\label{l5-A_n-B_n}
$$
f_n(x)=A_n(x)-B_n(x).
$$
Для нее
$$
s_n(f_n)=A_n(x),\qquad A_n(0)=\sum\limits_{k=1}^{n-1} \frac{1}{k} \asymp \ln
n.
$$
Далее,
$$
f_n(x)=\sum\limits_{k=1}^{n-1} \left\{ \frac{\cos kx}{n-k}-\frac{\cos
(n+n-k)x}{n-k} \right\}
=2\sin nx \sum\limits_{k=1}^{n-1} \frac{\sin (n-k)x}{n-k}=2\sin nxd_{n-1}(x),
$$
следовательно, {$|f_n(x)|\le 2B.$} Итак, мы построили
равномерно ограниченную последовательность функций $f_n(x),$ такую, что
$s_n(f_n)(0)\asymp \ln n.$
Значит, если ${f_n^*}(x)=\dfrac{f_n(x)}{2B},$ то $\|{f_n^*}\| \le 1$ и
$$
L_n\ge \|s_n(f_n^*)\|_C\ge c\ln n.
$$
Следовательно, константы Лебега имеют порядок $L_n \asymp \ln n.$ Теорема доказана.
\end{proof}

Мы показали, что для любого {$n\in\bN$} существует {$2\pi$-периодическая}
функция {$f_n\in C_{2\pi},$~ $\|f_n\|_C\le 1,$} такая, что
$\|s_n(f_n)\|\ge a\ln n.$ Поставим вопрос: можно ли здесь взять функцию
$f$, не зависящую от $n$, т.\,е. верно ли, что
$$
\exists\ f\in C_{2\pi}\qquad \exists\ a>0\qquad \forall\ n>1\qquad \|s_n(f)\|\ge a\ln n\,?
$$


Оказывается, что такое утверждение неверно: для каждой функции {$f\in C[0,2\pi]$}
имеем {$\|s_n(f)\|_C=o(\ln n)\ \ (n\rightarrow\infty).$ Эту оценку на всем классе
$C[0,2\pi]$ тоже нельзя} {улучшить в смысле порядка, так как} имеет место
{следующая} теорема (которая приводится без доказательства).
\begin{teo}[Д.\,Е.\,Меньшов]
Для каждой функции $\varphi(n)=o(\ln n)$, $n\rightarrow \infty$, существует непрерывная
$2\pi$-периодическая
функция $f=f_{\varphi}$ такая, что для всех достаточно больших~$n$
$$
\|s_n(f)\|_C\ge \varphi(n).
$$
\end{teo}

В пространстве {$L_{2\pi}$} имеет место теорема, аналогичная {теореме~\ref{l5-ln}}:
$$ \|{S}_n\|_L \asymp \ln n,\qquad n\to \infty.$$

Так что константы Лебега в {$L^p_{2\pi}$} ограничены при каждом
$p\in (1,\infty)$ и имеют порядок $\ln n$ для {$L_{2\pi}$} (т.\,е. при
$p=1$) и для {$C_{2\pi}$} (т.\,е. при $p=\infty$).

Нормы оператора $S_n$  в пространствах $L^p_{2\pi}$ как
функции от $p$ схематично изображены на рис.~5.2.

\begin{figure}[ht]
\begin{center}
\includegraphics{pict/pict05-2.eps}
\end{center}
 \bigskip
 \refstepcounter{ris}\label{r5-2}

 \centerline{Рис.~\theris. }
 \bigskip
\end{figure}




Вообще, из всех {$L^p_{2\pi}$} ($1\le p\le \infty$) в смысле
приближения суммами Фурье пространство
{$L^2_{2\pi}$} лучше всего, $C_{2\pi}$ хуже
всего; {$L_{2\pi}$} примерно как $C_{2\pi},$ хотя немного лучше; {пространства $L^p_{2\pi},$~ $1< p< \infty, $}
похожи на {$L^2_{2\pi},
\ L_{2\pi}$} похоже на {$C_{2\pi}.$} В {$L_{2\pi}$} и в {$C_{2\pi}$}
нормы операторов {$S_n$} неограничены, следовательно,
существуют функции, для которых ряды Фурье (в {$L_{2\pi}$} и {$C_{2\pi}$})
не будут сходиться, и даже частичные суммы не будут ограничены.
{ В отличие от этого, для любой функции $f\in L^p_{2\pi}\ (1<p<\infty)$ ее ряд}
{Фурье~(\ref{Fourier}) сходится в пространстве
$L^p_{2\pi}$ к $f$: $\|f-S_n(f)\|_{L^p}\to0\ \ (n\to\infty).$ Имея это в виду,
для $f\in L^p_{2\pi}\ (1<p<\infty)$ в~(\ref{Fourier}) вместо знака
соответствия $(\sim)$ пишут также знак равенства $(=)$\footnote{Знаменитый результат
Л.\,Карлесона ($p=2,$ 1966~г.) и Р.\,Ханта ($1<p<\infty$, 1967~г.):
$f\in L_{2\pi}^p\ (1<p<\infty)\Rightarrow s_n(f,x)\to f(x)\ (n\to\infty)$ почти всюду.}.

\section{Суммы Фейера}\label{s5-2}

Суммами Фейера называются полиномы
\begin{equation}
\label{Fieyer}
\sigma_n(f,x)=\frac{1}{n+1} \sum\limits_{k=0}^{n} s_k(f,x)=\frac{1}{\pi}
\int_{-\pi}^{\pi} f(x+t) K_n(t)\,
dt,
\end{equation}
{заданные на пространстве $2\pi$-периодических функций} $L_{2\pi},$ где $K_n(t)$ -- среднее
арифметическое ядер Дирихле.

Перечислим свойства сумм Фейера.


1. Основное свойство. Из представления~(\ref{Fieyer})
следует, что если $f(x)\ge 0$ для всех $x,$ то $\sigma_n(f,x)\ge
0$ для всех $x$ и $n,$ так как неотрицательны ядра Фейера:
$$
K_n(t)=\frac{1}{n+1} \sum\limits_{k=0}^{n}
\mathcal D_k(t)=\frac{\sin^2(n+1)\frac{t}{2}}{2(n+1)\sin^2 \frac{t}{2}}\ge
0.
$$
Операторы,
обладающие таким свойством, называются \textit{положительными}.

Пусть $K^+=\{ f(x)\ge 0\}$~-- конус положительных функций в
$C_{2\pi}$ (класс $K$ считается конусом, если $\{ f\in K,\ \lambda>0\}
\Rightarrow \lambda f\in
K$).
Внутренние точки {в $K^+$}~-- строго положительные функции. {Говоря о конусе
$K^+$ в $L^p_{2\pi},$ будем предполагать дополнительно,}
{что $K^+\subset L^p$ и что он наделен топологией $L^p_{2\pi}.$
Тогда в $L^p_{2\pi}$ при} {$1\le p<\infty$} конус $K^+$ не
имеет внутренних точек. В отличие от этих пространств, конус
{$K^+$ в $C_{2\pi}$} имеет внутренние точки.

2. Суммы Фейера переводят конус положительных функций в
    себя:
$$
\sigma_n(K^+) \subset K^+.
$$

3. Суммы Фейера оставляют константы на месте:
$$
\sigma_n(A,x)=A,
$$
если $A$ не зависит от $x.$

4. Нормы сумм Фейера в $C_{2\pi}$ равны 1:
$$
\|\sigma_n\|_C=1.
$$

Действительно, если $\|f(x)\|_C\le 1,$ то
$$
\left| \frac{1}{\pi} \int_{-\pi}^{\pi} f(x+t) K_n(t)\, dt \right|\le \frac{1}{\pi}
\int_{-\pi}^{\pi} K_n(t)\, dt=1,
$$
{так как $K_n(t)=\dfrac{1}{2}+\ds\sum\limits_{k=1}^n \left(1-\frac{k}{n+1} \right)\cos kt.$ }
С другой стороны, $\sigma_n(1,x){\equiv}1.$
Следовательно,
$$
\|\sigma_n\|_C=\sup_{\|f\|_C\le 1} \|\sigma_n(f,x)\|_C=1.
$$
Аналогичное утверждение имеет место и для всех $L_p$:
$$
\|\sigma_n\|_p=1 \qquad (1\le p<\infty).
$$
Доказательство в основном такое же.

В курсе математического анализа доказывается следующее утверждение.
\begin{teo}
Для любой непрерывной функции $f$
$$
\|\sigma_n(f)-f\|_C \to 0,\qquad n\to\infty,
$$
т.\,е. суммы Фейера равномерно сходятся к функции.
\end{teo}
