% Лекции Сергея Борисовича Стечкина
% Внесены исправления В.А. Юдина, версия 27.01.2009
% Внесены исправления Н.И. Черныха, версия 19.07.2009
% Внесена грамматическая и ТеХ-правка М.Дейкаловой, версия 05.08.09

\chapter{Теорема Джексона}

\section{Приближение в $L_2(a,b)$}

%Пусть $X$~-- сепарабельное банахово пространство, $x_1,x_2,\ldots,x_n,\ldots$~--
%бесконечная система элементов из $X$,~ $L_n$~--
%подпространство, натянутое на $n$ первых элементов $\{x_1,\ldots,x_n\}$.
%Для любого $x\in X$ рассмотрим {наилучшее приближение}
%$$
%E_n(x)=E(x,L_n)_X\qquad (n=1,2,\ldots)
%$$
%{и отображение}
%$$
%X\longrightarrow \{ E_n(x)\}.
%$$

Полагаем дальше $X=L^2_{2\pi},$ $E_n(x)_{L^2}$ -- наилучшее
приближение в $L^2_{2\pi}$ функции $x(t)\in L^2_{2\pi}$
подпространством ${\cal T}_n$ (размерности $2n+1$)
тригонометрических полиномов $t_n(t)$ по системе
$$
1,\cos t,\sin t,\cos 2t,\sin 2t,\ldots.
$$
Эта система полна в {$L^2_{2\pi}$}, т.\,е.
$$
\forall\ x:\qquad E_n(x) \longrightarrow 0\qquad (n\to \infty).
$$
Рассмотрим для функции $x\in L^2_{2\pi}$ ее ряд Фурье
$$
x(t)\sim \frac{a_0}{2}+\sum\limits_{k=1}^{\infty} {(a_k \cos kt+b_k\sin kt)}
$$
и пусть
$$
s_n(t)\sim \frac{a_0}{2}+\sum\limits_{k=1}^{n} {(a_k \cos kt+b_k\sin kt)}.
$$
Тогда
$$
E_n^2(x)_{L^2}=\|x(t)-s_n(t)\|_{{L^2}}^2=\sum\limits_{k=n+1}^{\infty}(a_k^2+b_k^2)
$$
(по равенству Парсеваля).

\section{Неравенство Бернштейна}

Для любого тригонометрического полинома порядка $n$ выполняется неравенство
$$\|t_n'\|_{{L^2}}\le n\|t_n\|_{{L^2}}.$$


Действительно,
$$
\|t_n'\|_{{L^2}}^2 ={\sum\limits_{k=1}^n k^2}(a_k^2+b_k^2),
$$
$$
(n\|t_n\|_{{L^2}})^2=n^2\|t_n\|_{{L^2}}^2=n^2\left(
\frac{a_0^2}{2}+\sum\limits_{k=1}^n (a_k^2+b_k^2)\right).
$$
Отсюда ясно, что
$$
\|t'\|_{{L^2}}^2\le n^2\|t_n\|^2_{{L^2}}.
$$

\section{Модуль колебания, модуль непрерывности}

Пусть $\Delta_h x(t)=x\left( t+\dfrac{h}{2}\right)-x\left( t-\dfrac{h}{2}\right).$ Тогда величина
$$
\|\Delta_h x(t)\|_{{L^2}}=\ae(h,x)
$$
называется модулем колебания функции с шагом $h$ (в $L^2_{2\pi}$), а
величина
$$
\sup_{|h|\le \delta} \ae(h,x)=\omega(\delta,x)_{{L^2}}
$$
-- модулем непрерывности функции (в $L^2_{2\pi}$).

\begin{Remark} %%%Замечание.
Модуль непрерывности не может убывать
слишком быстро. Если $\ae(h,x)=o(h)$ при $h\to 0,$ т.\,е.
$$
\left\| \frac{\Delta_h x}{h}\right\|_{{L^2}} \to 0\qquad (h\to 0),
$$
то отсюда вытекает, что производная в метрике $L_2$
равна нулю почти всюду, т.\,е. \linebreak $x=\Const.$
\end{Remark}

Обозначим
$$
{\Delta_h^k x(t)=\Delta_h^{k-1}(\Delta_h x(t)),}
$$
$$
\left\| \Delta_h^k(x,t)\right\|_{{L^2}}=\ae_k(h,x).
$$
Тогда
$$
\sup_{|h|\le \delta} \ae_k(h,x)=\omega_k(\delta,x)_{{L^2}}
$$
называется модулем непрерывности $k$-го порядка.

\section{Теорема Джексона в {$L^2_{2\pi}$}}

\begin{teo}[неравенство Джексона]
Для каждой функции $x(t)\in L^2_{2\pi}$

$1)$ $E_n(x)_{{L^2}}\le C\omega\left( \dfrac{\pi}{n},x\right)_{{L^2}};$

$2)$ $E_n(x)_{{L^2}}\le C_k \omega_k \left( \dfrac{\pi}{n},x\right)_{{L^2}},$~ $k\in \bN;$

\noindent {\it где} $C$ {\it и} $C_k$ -- {\it абсолютные константы.}
\end{teo}

\begin{proof}%%%   Доказательство.
Имеем
$$
E_n^2(x)_{L^2}=\sum\limits_{k=n+1}^{\infty}(a_k^2+b_k^2),
$$
$$
{\Delta_h x(t)\sim\sum\limits_{k=1}^\infty\left(2\sin \frac{kh}{2}\right)(-a_k\sin kt+ b_k\cos kt),}
$$
{откуда}
$$
\ae^2(h,x)=4 \sum\limits_{k=1}^{\infty}\sin^2 k\frac{h}{2} (a_k^2+b_k^2).
$$
Но
$$
\frac{1}{\delta_n}\int_0^{\delta_n} \sin^2 k\frac{h}{2}\ {dh} \ge c>0\qquad
\left( k\ge n,\ \delta_n=\frac{\pi}{n}\right),
$$
поэтому
$$
\frac{1}{\delta_n}\int_0^{\delta_n} \sum\limits_{k=1}^{\infty}
\sin^2 k\frac{h}{2}(a_k^2+b_k^2)\ dh \ge
c \sum\limits_{k=n+1}^{\infty} (a_k^2+b_k^2),
$$
откуда
$$
\sum\limits_{k=n+1}^{\infty} (a_k^2+b_k^2)\le c_1 \frac{1}{\delta_n}
\int_0^{\delta_n} \sum\limits_{k=1}^{\infty}
\sin^2 k\frac{h}{2}(a_k^2+b_k^2)\ dh=
$$
$$
=c_2\frac{1}{\delta_n} \int_0^{\delta_n}\ae^2 (h,x)\ dh \le
c_2\omega(\delta_n,x)_{{L^2}}.
$$
\end{proof}

Доказательство второго утверждения теоремы аналогично, {если учесть, что}
$$
{\Delta_h^k x(t)\sim\sum_{l=1}^\infty\left(2\sin \frac{lh}{2}\right)^k
\left(a_l\cos \left(lx+\frac{k\pi}{2}\right)+ b_l\sin \left(lx+\frac{k\pi}{2}\right)\right).}
$$

\begin{Corollary}
Пусть функция $x$ абсолютно непрерывна и ее производная принадлежит {$L^2_{2\pi}.$} В этом случае
$$
\left\| \frac{\Delta_h x(t)}{h}\right\|_{{L^2}}\le K
$$
и тогда $E_n(x)=O\left( \dfrac{1}{n}\right),$ так как $\ae(h,x)\le Kh$ и, следовательно, $\omega(\delta,x)\le Kh.$
\end{Corollary}

\section{Обратная теорема}

\begin{teo} Для каждого $n\in\mathbb N$ и любой функции $x$ из $L^2_{2\pi}$
справедливо неравенство
$$
\omega^2\left( \frac{\pi}{n},x\right)_{{L^2}}\le \frac{C}{n^2}
\sum\limits_{{k=1}}^nkE_{{k-1}}^2(x)_{L_2},
$$
где $C=2\pi^2$.
\end{teo}

\begin{proof}
Разобьем $\|\Delta_h x\|^2$ на два слагаемых
$$\|\Delta_hx(t)\|_{L^2}^2 =
\sum\limits_{k=1}^{n-1}\left(2\sin\frac{kh}{2}\right)^2(a_k^2+b_k^2) +
\sum\limits_{k=n}^\infty\left(2\sin\frac{kh}{2}\right)^2(a_k^2+b_k^2) = I_1 + I_2.
$$
Оценим эти слагаемые, используя неравенства $|\sin x|\le|x|$,~ $|\sin x|\le1$,~
$x\in\mathbb R$. Имеем
$$I_1\le h^2\sum\limits_{k=1}^{n-1}k^2(a_k^2+b_k^2),
$$
$$
I_2\le 4\sum\limits_{k=n}^\infty (a_k^2+b_k^2),$$
откуда для
$$
\omega^2\left(\frac{\pi}{n},x\right)_{{L^2}} =
\sup_{|h|\le\frac{\pi}{n}}\|\Delta_hx\|^2_{{L^2}}
$$
имеем
$$
\omega^2\left(\frac{\pi}{n},x\right)_{{L^2}} \le \frac{\pi^2}{n^2}
\sum\limits_{k=1}^{n-1}k^2(a_k^2+b_k^2)+4E_{n-1}^2(x).
$$
Применяя
преобразование Абеля, получим
$$
\sum\limits_{k=1}^{n-1}k^2(a_k^2+b_k^2)\le
\sum\limits_{k=1}^{n-1}(2k-1)E_{k-1}^2(x)
{-(n-1)^2E_{n-1}^2(x)}.
$$
Следовательно,
$$
\omega^2\left(\frac{\pi}{n},x\right)_{{L^2}}\le
\frac{\pi^2}{n^2}\sum\limits_{k=1}^{n-1}(2k-1)E_{k-1}^2(x)
{-\frac{\pi^2}{n^2}(n-1)^2E_{n-1}^2(x)}+4E_{n-1}^2(x),
$$
{откуда легко выводится утверждение теоремы.}
Теорема доказана.
\end{proof}
