% Лекции Сергея Борисовича Стечкина
% ??? Внесены исправления В.И.Бердышева
% Внесены исправления Н.И.Черныха, версия 23.07.2009
% Внесена грамматическая и ТеХ-правка М.Дейкаловой, версия 05.08.09

%%%%%%%%%%%%%%%%%%%%%%%%%%%%%
\chapter{Критерий наилучшего \\
 приближения в $\boldsymbol L^p$. Корректность}
%%% Лекция 10.


 \section{Критерий наилучшего элемента в $L^p$}

  %%% Замечание.
 Пусть $H$ -- гильбертово пространство. В курсе анализа
 доказывается, что для того чтобы $y^*$ был наилучшим элементом в
 подпространстве
 $M\subset H$ для элемента $x,$ необходимо и достаточно, чтобы
 $$
 (x-y^*,y)=0\qquad \forall\  y\in M.
 $$
 Если $H=L_2,$ то это условие можно переписать
 $$
 \int_Q (x-y^*)y\, dt=0\qquad \forall\  y\in M.
 $$


 Оказывается, это утверждение есть частный
 случай более общей теоремы, которая дает необходимое
 и достаточное условие для наилучшего элемента в $L_p,$~ $p>1.$

 \begin{teo} %%% Теорема.
 Пусть $p>1,$~ $M\subset L_p$ -- подпространство, $ \ x\in L_p.$
 Для того, чтобы $y^*$ был наилучшим элементом из $M$ для $x$
 в $L_p$ необходимо и достаточно, чтобы
 \begin{equation}\label{f9-1}
 {\int_Q |x-y^*|^{p-1} \sign (x-y^*)y\, dt=0} {\qquad \forall\  y\in M}.
 \end{equation}
 \end{teo}

 \begin{proof} %%% Доказательство.
 Н\;е\;о\;б\;х\;о\;д\;и\;м\;о\;с\;т\;ь.~ Пусть условие \eqref{f9-1}
 не выполняется, т.\,е. существует $y\in M$ такой, что
 $$
 {\int_Q |x-y^*|^{p-1} \sign (x-y^*)y\, dt\ne 0.}
 $$
 Покажем, что тогда $y^*$ не есть наилучший элемент.

 Рассмотрим
 $$
 \Phi(\alpha)=\|x-y^*-\alpha y\|^p={\int_Q |x-y^*-\alpha y|^p\, dt.}
 $$
 Так как $p>1,$ то это есть дифференцируемая функция от $\alpha,$
 и по теореме о дифференцировании по параметру под
 знаком интеграла
 $$
 \Phi'(\alpha)=-p {\int_Q |x-y^*-\alpha y|^{p-1}\sign(x-y^*-\alpha y)y\, dt.}
 $$
 При $\alpha=0$ имеем $\Phi'(\alpha)|_{\alpha=0}\ne 0,$
 следовательно, при $\alpha=0$ нет минимума. Значит, при некотором
 $\alpha$ можно сделать уклонение $\|x-y^*-\alpha y\|$ еще меньше, чем
 $\|x-y^*\|,$ т.\,е. $y^*$ не является наилучшим элементом. Противоречие.

 Д\;о\;с\;т\;а\;т\;о\;ч\;н\;о\;с\;т\;ь.~ В силу \eqref{f9-1}, применяя неравенство
 Гельдера,  {для любого} {$y \in M$}
 имеем
 $$
 \int_Q |x-y^*|^{p}\, dt=\int_Q |x-y^*|^{p-1}(x-y^*)\sign(x-y^*)\, dt=
 $$
 $$
 =\int_Q |x-y^*|^{p-1}(x-y)\sign(x-y^*)\, dt \le \int_Q |x-y^*|^{p-1}|x-y|\
 dt\le
 $$
 $$
 \le \left\{ \int_Q |x-y^*|^{p}\, dt \right\}^{1/q}\cdot
 \left\{ \int_Q |x-y|^{p}\, dt \right\}^{1/p},\qquad
 {\frac{1}{p}+\frac{1}{q}=1}.
 $$
 Можно считать, что $\ds\int_Q |x-y^*|^{p}\, dt\ne 0$ (в противном случае
 $y^*$ -- наилучший элемент и все доказано). Тогда
 $$
 \left\{ \int_Q |x-y^*|^{p}\, dt \right\}^{1/p}\le
 \left\{ \int_Q |x-y|^{p}\, dt \right\}^{1/p},
 $$
 т.\,е. для любого $y\in M$ выполняется неравенство $\|x-y^*\|\le \|x-y\|,$ значит,
 $\|x-y^*\|=E(x,M).$
 Теорема доказана.
 \end{proof}

 \begin{Remark} %%% Замечание.
 При $p=1$ {условие} \eqref{f9-1} имеет вид
\begin{equation}\label{f9-2}
 \int_Q \sign(x-y^*)y\, dt=0,\qquad \forall\  y\in M,
\end{equation}
 и это есть достаточное условие для того, чтобы $y^*$
 был наилучшим элементом для $x$ (доказательство то же самое).
 Это условие будет и необходимым, если априори известно, что
 $x(t)-y^*(t)\ne 0$ почти всюду на $Q$ (так как тогда при $\alpha=0$ почти всюду существует
 производная $\dfrac{d}{d\alpha}|x-y^*-\alpha y|\Big|_{\alpha=0}=-y
 \, \sign(x-y^*)$). В общем случае это условие только
 достаточное. Необходимое и достаточное условие можно найти
 в работе\footnote{Kripke~B.R., Rivlin~T.J. Approximations in the metric of $L_1(X,\mu)$ //
 Trans. Amer. Math. Soc. 1965. Vol.~119, \No~1, iss.~7.
 P.~101--122.}.
% {[Kripke, Rivlin]}.}
 \end{Remark}

%\vspace{2mm}

%{\bf 1. Наилучшее приближение.}
\section{Наилучшее приближение в $L^1$}

Мы рассмотрели приближение в {$L^p$ подпространствами $M$} и
 доказали, что при $p>1$ {элемент} $y^*\in M$
 будет наилучшим элементом для $x$ тогда и только тогда, когда
 \begin{equation}\label{f9-1-10}
 \int |x-y^*|^{p-1} \sign (x-y^*)y\, dt=0\qquad \forall\  y\in M.
 \end{equation}

 \begin{Example} %%% Пример.
 Рассмотрим частный случай $M=\{c\}$ --  одномерное подпространство
 констант.

 Пусть $p=2$, {$L^p=L^2[a,b].$} В этом случае условие \eqref{f9-1-10} запишется так:
 $$
 \int_{a}^b\{x(t)-c^*\}\, dt=0,
 $$
 т.\,е. (см. рис.~10.1) $S_+=S_-.$

 %\vspace{2cm}
 %%%%%%%%%%%%%%%%%%%%%%%%%%%%%%%%%%%%%%%%%%%%%%%%%%%%%%
 %\hbox to 0.5cm {}{\special{em:graph pict1.pcx}}
 %\vspace{6cm}
 %%%%%%%%%%%%%%%%%%%%%%%%%%%%%%%%%%%%%%%%%%%%%%%%%%%%%%%%%%
 %\noindent \hskip3.0cm {рис.}

  \bigskip
\begin{figure}[ht]
\begin{center}
\includegraphics{pict/pict10-1.eps}
\end{center}
 \bigskip
 \refstepcounter{ris}\label{r10-1}

 \centerline{Рис.~\theris}
 \bigskip
\end{figure}



 \noindent Если $p=1,$ то условие \eqref{f9-1-10} перепишется как
 $$
 \int_a^b\sign \{x(t)-c^*\}\, dt=0
 $$
 или, { в случае, когда $L^1[a,b]$ есть пространство с мерой $\mu,$ как}
 \begin{equation}\label{f10-1}
 \mu(E_+)-\mu(E_-)=0,
 \end{equation}
 где $E_+\ (E_-)$ -- множество, на котором разность $x(t)-c^*>0\
 (<0),$~ $\mu$ -- мера.
 \end{Example}

 Легко построить пример, когда нет единственности
 наилучшего элемента в {$L^1$} \linebreak {{(см. рис.~10.2)}:}

\begin{figure}[ht]
\begin{center}
\includegraphics[width=0.5\textwidth]{pict/pict10-2.eps}
\end{center}
 \bigskip
 \refstepcounter{ris}\label{r10-2}

 \centerline{Рис.~\theris\ \  $(c^*\in [0,1])$}
\end{figure}



 \noindent Любая константа $c^*\in [0,1]$ -- {здесь} наилучшая.

 Как отмечалось, при $p=1$ {условие} \eqref{f9-1-10} -- {только} достаточное для наилучшего
 элемента. Можно построить пример, когда для наилучшей
 константы в {пространстве $L^1$ с мерой} условие \eqref{f10-1} {тоже} не выполняется
 (см. рис.~10.3 для меры Лебега).

 %\vspace{2cm}
 %%%%%%%%%%%%%%%%%%%%%%%%%%%%%%%%%%%%%%%%%%%%%%%%%%%%%%
 %\hbox to 0.5cm {}{\special{em:graph pict1.pcx}}
 %\vspace{6cm}
 %%%%%%%%%%%%%%%%%%%%%%%%%%%%%%%%%%%%%%%%%%%%%%%%%%%%%%%%%%
 %\noindent \hskip3.0cm {рис.}
 \bigskip
\begin{figure}[ht]
\begin{center}
\includegraphics{pict/pict10-3.eps}
\end{center}
 \bigskip
 \refstepcounter{ris}\label{r10-3}

 \centerline{Рис.~\theris\ (здесь видно, что $c^*=1/2$)}
 \bigskip
\end{figure}




 Если {же $\mu\{t\in[a,b] : x(t)-y^*(t)=0\}=0,$} то дифференцирование под знаком интеграла
 законно, и нарушение условия \eqref{f10-1} означает, что $y^*$ не является наилучшим
 элементов в $M$ для $x$.

 Пусть $X$ -- банахово пространство, $M\subset X $ --
 подпространство.
 В задаче о наилучшем приближении для любого $x\in X$
 мы ищем $y^*\in M$ такой, что
 $$
 \|x-y^*\|_X\le \|x-y^*-h\|\qquad \forall\  h\in M;
 $$
 т.\,е. $y^*$ будет наилучшим элементом, если  не найдется понижающий
 элемент $h.$ Значит, функционал
 $$
 \Phi(h)=\|x-y^*-h\|
 $$
 должен достигать при $h=\theta$ своего минимума. Если $\Phi(h)$ --
 дифференцируемый функционал, то для минимума необходимо, чтобы
 дифференциал от $\Phi(h)$ обращался в 0 при $h=\theta.$
 Или пусть фиксировано $h.$ Тогда функционал
 $$
 F_h(t)=\|x-y^*-th\|,\qquad t\in (-1,1),
 $$
 должен иметь минимум в точке $t=0$ при каждом $h\in M$; т.\,е. $\cD\|x-y^*-h\|
 |_{h=\theta}=0,$ так что условие \eqref{f9-1-10} просто означает, что дифференциал равен
 нулю.

 Так как $\Phi(h)$ -- выпуклый функционал, то
если $\Phi(h)$~-- дифференцируемый функционал,
 условие обращения дифференциала для $\Phi(h)=\|x-y^*-h\|$ на пространстве
 $M$ в 0 при $h=\theta$ есть необходимое и достаточное для того, чтобы $y^*$
 был наилучшим.

\section{Корректность}

Рассмотрим вопрос о непрерывной зависимости решения задачи наилучшего приближения
от задаваемых условий.

 Пусть $X$ --~банахово пространство, и пусть далее $M\subset X$ --~множество
 существования, элементы $x\in X,$ $y(x)$ -- элемент
 наилучшего приближения в $M$ для $x$, $E(x,M)_X$ --~наилучшее приближение
 элемента $x.$ Очевидно, что $E(x,M)_X=\Phi(x)$ --
 функционал~от~$x.$

 Так как
 $$
 E(x,M)-E(x',M)=\|x-y(x)\|-\|x'-y(x')\|\le \|x-y(x')\|-\|x'-y(x')\|\le
 \|x-x'\|,
 $$
 то получаем, что {\it наилучшее приближение {$E(x,M)$} зависит от} $x$
 {\it {непрерывно и} {даже равномерно непрерывно.}}

\ex Доказать полученную оценку без предположения $M\in (E).$

 Пусть теперь в $M$ и для $x$, и для $x'$ существуют единственные
 наилучшие элементы $y(x)$ и $y(x').$ Если $x$ и $x'$ близки,
 следует ли тогда, что $y(x)$ и $y(x')$ близки? Вообще говоря, это не
 так, $y(x),$ вообще говоря, не является непрерывной функцией от $x.$
 Но непрерывность $y(x)$ имеет место в одном важном случае.

 Пусть $M$ ограниченно компактно, т.\,е. пересечение $M$
 с любым шаром {есть} компакт. Ограниченно компактное
 множество всегда есть множество существования, т.\,е.
 для любого $x$ множество наилучших элементов
 $Y(x)\subset M$ не пусто. {Нас интересует} {непрерывность отображения}
 $$
   x \longmapsto Y(x)\subset M,\qquad{ x \in X}.
 $$
 Будем рассматривать {$Y_{\varepsilon}$~--} $\varepsilon$-расширение $Y(x)$ в $M$,~
 {$Y_{\varepsilon}=\{y\in M:\rho(y,Y(x))<\varepsilon\}$}
 Зафиксируем какой-нибудь элемент $x\in X.$
 Выясним, как связаны $Y_{\varepsilon}$ и $Y.$

 %\vspace{2cm}
 %%%%%%%%%%%%%%%%%%%%%%%%%%%%%%%%%%%%%%%%%%%%%%%%%%%%%%
 %\hbox to 0.5cm {}{\special{em:graph pict1.pcx}}
 %\vspace{6cm}
 %%%%%%%%%%%%%%%%%%%%%%%%%%%%%%%%%%%%%%%%%%%%%%%%%%%%%%%%%%
 %\noindent \hskip3.0cm {рис.}

  \bigskip
\begin{figure}[ht]
\begin{center}
\includegraphics{pict/pict10-4.eps}
\end{center}
 \bigskip
 \refstepcounter{ris}\label{r10-4}

 \centerline{Рис.~\theris}
 \bigskip
\end{figure}



 \noindent Ясно, что $Y=\bigcap\limits_{\varepsilon >0} Y_{\varepsilon}.$
 Теперь возьмем $E(x,M)=d$ и рассмотрим множество\linebreak (см. рис.~10.4)
 $$
 Z(\varepsilon)=Z(\varepsilon,x)=\{ z\in M:\ \|x-z\|\le d+\varepsilon\}.
 $$
 Тогда для ограниченно компактных множеств $M$ справедливо

 \begin{Proposition} %%% Утверждение.
 Для любого $\varepsilon>0$ найдется $\varepsilon_1>0$ такое, что
 $Z(\varepsilon_1)\subset Y_\varepsilon.$
 \end{Proposition}

 Действительно, {$\{Z(\varepsilon_1)\}_{\varepsilon_1}$} --
 убывающая по вложению при $\varepsilon_1\downarrow 0$
 система компактных множеств и $\bigcap\limits_{\varepsilon_1>0} Z(\varepsilon_1)=Y{(x)}.$
 Тогда по свойству компактных множеств для любой окрестности
 $Y_{\varepsilon}$  {множества $Y(x)$}
 все $Z(\varepsilon_1),$ начиная
 с некоторого $\varepsilon_0,$ т.\,е. при $\varepsilon_1\le \varepsilon_0,$
 лежат внутри $Y_{\varepsilon}.$

 В общем случае, если нет ограниченной компактности, это утверждение
 не имеет места.

 В частности получаем, что если
 {в ограниченно компактном множестве $M$}
 есть единственный
 наилучший элемент
{для $x$}, то все <<хорошие>> точки, т.\,е. такие $z\in M,$
 расстояние от которых до $x$
 мало отличается от наилучшего приближения, лежат
 в некоторой малой окрестности наилучшего элемента.

 \begin{teo}[о корректности] %%% Теорема
 Пусть $X$ -- банахово пространство, $M$ -- ограниченно компактное
 подмножество из $X,$~ $x\in X$ и существует единственный элемент
 $y^*\in M,$ ближайший к {$x.$} Тогда если $\{x_n\}$ -- любая сходящаяся
 к $x$ последовательность из $X$ и $\{y_n\}$ -- последовательность из
 $M$ такая, что $\|x_n-y_n\|\to \|x-y^*\|,$ то $y_n\to y^*.$
 \end{teo}

 Действительно, для любого $\delta>0$ для достаточно
 больших $n$
 $$
 \|x-y_n\|\le \|x-x_n\|+\|x_n-y_n\|\le \|x-y^*\|+\delta,
 $$
 т.\,е. $y_n\in Z(\delta).$ Мы уже отмечали, что для любого
 $\varepsilon>0$ множество $Z(\delta)$ принадлежит $Y_{\varepsilon}$ для
 достаточно малых $\delta.$ Таким образом, $\|y_n-y^*\|\le \varepsilon$
 для всех достаточно больших $n,$ т.\,е. $y_n\to y^*.$

 Учитывая непрерывность $E(x,M)$ {получаем}
 \begin{Corollary}
 Пусть $M$ -- ограниченно компактное множество из $X$
 и для любого $x\in X$ наилучший элемент $y(x)$ -- единственный.
 Тогда $y(x)$ {есть} непрерывная функция от $x$
 на всем банаховом пространстве. {При этом} $y(x)$ будет равномерно непрерывной
 функцией, если $x\in K,$ где $K$ -- компакт.
 \end{Corollary}

 \task %%% Задача.
 В пространстве $C{=C[0,1]}$ приближаем функциями из класса
 $$
 \{ x\in C:\quad \|x'\|\le 1,\quad x(0)=0\}.
 $$
 Будет ли $y(x)$ непрерывна? В каких банаховых пространствах для
 любого подпространства {$M$ метрическая проекция $y(x)$}
 равномерно непрерывна?

 Пусть $H$ -- гильбертово пространство, $M\subset H $ -- подпространство,
 $  x\in X,$ $y(x)$~{-- ближайший к $x$ элемент в $M.$}
 Из критерия наилучшего приближения
 $$
 (x-y(x),y)=0\qquad \forall\  y\in M
 $$
 следует, что
 $$
 \|x-y(x)\|^{2}+\|y(x)\|^2=\|x\|^2
 $$
 и
 $$
 \|y(x)\|\le \|x\|.
 $$
 Тогда {в силу линейности метрической проекции в $H$ на подпространство}
 $$
 \|y(x)-y(x')\|=\|y(x-x')\|\le \|x-x'\|,
 $$
 т.\,е. метрическая проекция в гильбертовом
 пространстве равномерно непрерывна (и является
 ограниченным линейным оператором).

 \begin{Remark} %% Замечание.
 Вообще говоря (и как правило, если пространство не гильбертово), линейность наилучших
 элементов не имеет места. Например, пусть в $C[-1,1]$
 приближаем константами {(см. рис.~10.5)} функции
 $$
     {f_1(t)=\begin{cases}
     0, & -1 \le t \le 0 \\
     t, & 0 < t \le 1
   \end{cases},}
   $$
   $$
   f_2(t)=f_1(-t),
   \qquad (f_1+f_2)(t)=|t|.
 $$

 %\vspace{2cm}
 %%%%%%%%%%%%%%%%%%%%%%%%%%%%%%%%%%%%%%%%%%%%%%%%%%%%%%
 %\hbox to 0.5cm {}{\special{em:graph pict1.pcx}}
 %\vspace{6cm}
 %%%%%%%%%%%%%%%%%%%%%%%%%%%%%%%%%%%%%%%%%%%%%%%%%%%%%%%%%%
 %\noindent \hskip3.0cm {рис.}

 \bigskip
\begin{figure}[ht]
\begin{center}
\includegraphics{pict/pict10-5.eps}
\end{center}
 \bigskip
 \refstepcounter{ris}\label{r10-5}

 \centerline{Рис.~\theris}
 \bigskip
\end{figure}

 \noindent Здесь имеем: $\dfrac12=c^*(f_1)=c^*(f_2)=c^*(f_1+f_2)\ne c^*(f_1)+
 c^*(f_2)=1.$

 Линейность имеет место только в гильбертовом
 и некоторых вырожденных пространствах.
 \end{Remark}

 \begin{teo} %%% Теорема.
 Пусть $X$ -- равномерно выпуклое пространство, $M\subset X $ -- подпространство. Тогда $M$
 -- подпространство единственности, и метрическая проекция $y(x)$ {на $M$}
 равномерно непрерывно зависит от $x$ {на любом ограниченном} {множестве}.
 \end{teo}

 \begin{proof} %%% Доказательство.
 Первое утверждение теоремы вытекает из теоремы~9.3.
 Применяя неравенство треугольника
 и используя свойства наилучшего приближения, для произвольных элементов $x$ и $x',$
 обладающих наилучшими элементами в $M,$ получаем
 $$
 \|x-y(x{'})\|\le \|x'-y(x')\|+\|x-x'\|\le \|x'-y(x)\|+\|x-x'\|\le
 $$
 $$
 \le \|x-y(x)\|+\|x-x'\|+\|x-x'\|=\|x-y(x)\|+2\|x-x'\|,
 $$
 т.\,е. если $x'$ близок к $x,$ то равномерно относительно $x$ и $x'$ элемент
 $y(x')$ имеет отклонение от $x$, {близкое к наилучшему.} Получили, что
 $y(x')\in Z(2\|x-x'\|,x).$
 Ввиду равномерной выпуклости пространства отсюда
 следует, что расстояние между $y(x)$ и $y(x')$
 равномерно уменьшается с уменьшением расстояния между $x$ и $x'$
при условии, что величины $\|x\|$ ограничены.

 Пространства $L_p,~ p>1,$ равномерно выпуклы, следовательно, в $L_p,~
 p>1,$
 {на} {ограниченном множестве элементов метрическая проекция}
 $y(x)$ равномерно непрерывна.

 В гильбертовом пространстве, как уже отмечали,
 метрическая проекция равномерно непрерывна {на всем пространстве}.

 В пространстве $C{[a,b]}$ метрическая проекция не является равномерно
 непрерывной.
 \end{proof}

 \begin{Example} %%% Пример.
 В $C[0,1]$ приближаем функциями $a+bx=p(x).$ Построим для любого
 $\varepsilon>0$ непрерывные на $[0,1]$ функции $f$ и $\widetilde f$
 такие, что $\|f-{\widetilde f}\|_C\le \varepsilon,$ но
 $\|p^*(f)-p^*({\widetilde f})\| {> 1}.$
 Это и будет означать, что для метрической проекции $p(f)$
 нет равномерной непрерывности. Пример называется <<молния>> (см. рис.~10.6).
 Здесь $\widetilde f(\varepsilon)=1+\varepsilon,~ \widetilde f(-\varepsilon)=1-\varepsilon,$
 $f(-1)=\widetilde{f}(-1)=0,\ f(1)=\widetilde{f}(1)=0,\ f(0)=\widetilde{f}(0)=-1,\
 f(\varepsilon)=f(-\varepsilon)=1,\ \widetilde{f}(-\varepsilon)=1-\varepsilon,\
 \widetilde{f}(\varepsilon)=1+\varepsilon$~-- вершины
 ломанных -- графиков функций $f(x)$ и $\widetilde{f}(x).$

 %\vspace{2cm}
 %%%%%%%%%%%%%%%%%%%%%%%%%%%%%%%%%%%%%%%%%%%%%%%%%%%%%%
 %\hbox to 0.5cm {}{\special{em:graph pict1.pcx}}
 %\vspace{6cm}
 %%%%%%%%%%%%%%%%%%%%%%%%%%%%%%%%%%%%%%%%%%%%%%%%%%%%%%%%%%
 %\noindent \hskip3.0cm {рис.}

 \bigskip
\begin{figure}[ht]
\begin{center}
\includegraphics{pict/pict10-6.eps}
\end{center}
 \bigskip
 \refstepcounter{ris}\label{r10-6}

 \centerline{Рис.~\theris}
 \bigskip
\end{figure}


 \noindent Здесь наилучшим для $f$ является $p^*(f)\equiv 0,$
 наилучшим для $\widetilde f$ является $p^*(\widetilde f\,)=x,$
 это следует из теоремы Чебышева об альтернансе
 (будет доказана). В примере есть чебышевские
 альтернансы необходимой длины.
 \end{Example}

 \begin{Example} %%% Пример.
В $C[0,1]$ будем приближать функции рациональными дробями из множества
 $$
 M=\left\{ \frac{a}{b+cx}\right\}
 $$
 Можно показать, что оператор наилучшего приближения в этом случае не является непрерывным.

 Действительно, используя критерий Чебышева, для каждого $r>0$ легко построить
 непрерывную функцию $f_r,$ для которой $\dfrac{1}{1+rx}$
 будет наилучшей рациональной дробью (по чебышевскому
 альтернансу в трех точках)
 и такую, что $f_r(x) \rightrightarrows f,$ причем
 для $f$ наилучшая рациональная дробь тождественно равна $0,$ а $\dfrac{1}{1+rx}
 \not\rightrightarrows 0$~ $(r\to \infty),$ так как
 функция $R(x),$ предельная для последней дроби, является разрывной на
 отрезке $[0,1]$: $R(0)=1,\ R(x)=0,\ 0< x\le 1.$

 Таким образом, оператор {наилучшего} проектирования в $C[0,1]$ на {множество} гипербол терпит
 разрыв (см. рис.~\ref{r10-7}).
 \end{Example}

 %\vspace{-5cm}
\begin{figure}[h]
\begin{center}
\includegraphics[width=0.5\textwidth]{pict/pict10-7.eps}
\end{center}
\bigskip
\refstepcounter{ris}\label{r10-7}

\centerline{Рис.~\theris}
\end{figure}

\begin{ex}
 Придумать метрику в трехмерном пространстве (<<шар>>, определяющий
 метрику) такую, чтобы метрическая проекция на некоторое
 одномерное подпространство была разрывной.
\end{ex}
