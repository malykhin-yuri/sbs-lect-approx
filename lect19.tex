% Лекции Сергея Борисовича Стечкина
% Внесены исправления В.В.Арестова, версия 06.07.2009
% Внесены исправления Н.И.Черныха, версия 24.07.2009
% Внесена грамматическая и ТеХ-правка М.Дейкаловой, версия 05.08.09

 %%%%%%%%%%%%%%%%%%%%%%%%%%%%%
 \chapter{Теорема Фавара и ее приложения}
 %%{Лекция 19.}

 \section{Теорема Фавара\\ о приближении дифференцируемых
 функций}


 На прошлой лекции мы  приближали функции, истокообразно
 представимые с помощью ядра $K\in L_{2\pi},$
 тригонометрическими полиномами; в связи с этим  искали
 наилучшее приближение ядра
 $$
 E_{n-1}(K)_L=\min_{t_{n-1}}\|K-t_{n-1}\|_L=\|K-t_{n-1}^*\|_L
 $$
 тригонометрическими полиномами в $L_{2\pi}.$ Наилучшим
 полиномом часто является полином, интерполирующий функцию $K$ в
 равноотстоящих узлах с  расстоянием $\dfrac{\pi}{n}$ между соседними узлами.
 Это, в частности, справедливо для ядра Бернулли $K_r,$ которое дает
 интегральное представление $r$ раз дифференцируемых функций, точнее, функций из класса
  $W^{(r)}$:
 $$
 f(x)=\frac{a_0}{2}+\frac{1}{\pi}\int_0^{2\pi} K_r(t)\varphi(x+t)\, dt,
 $$
 здесь $\varphi=f^{(r)}.$ В этом случае, как мы показали в
 теореме~\ref{t18-2},
 $$
  E_{n-1}(K)_L=\| K_r-t_{n-1}^*\|_L=\frac{M_r}{n^r},
 $$
 где $M_r$ --  константы, определенные  формулой \eqref{f18-15} предыдущей лекции.
 %Можно показать, что  $1<M_r<\frac{\pi}{2}.$

\ \


Цель настоящей лекции состоит в том, чтобы

 1) для функции $f\in C_{2\pi}^{(r)}$  оценить  наилучшее приближение $E_{n-1}(f)_C$
 тригонометрическими полиномами порядка $n-1;$

 2) вычислить
 $$
 \sup_{{W_1^{(r)}}} E_{n-1}(f)_C=E_{n-1}(W_1^{(r)})_C,
 $$
 где $W_1^{(r)}$ -- класс функций $f\in W^{(r)},$ для которых
 $\|f^{(r)}\|_{L^\infty}\le 1.$

 Ясно, что если последняя задача решена, то для любой функции $f\in C^{(r)}$
 имеет место неравенство
 $$
 E_{n-1}(f)_C\le E_{n-1}(W_1^{(r)})_C\cdot \|f^{(r)}\|_C.
 $$
 Пусть $t_{n-1}^*$ -- полином наилучшего приближения  в среднем ядра $K_r.$
 Тогда
 \begin{equation}\label{f19-1}
 \int_0^{2\pi} \{ K_r(\theta)-t_{n-1}^{*}(\theta)\}\varphi(x+\theta)
 \,d\theta=f(x)-t_{n-1}(x),%\eqno(1)
 \end{equation}
 где  $t_{n-1}$ -- некоторый тригонометрический полином порядка $n-1.$ А, значит,
 \begin{equation}\label{f19-2}
 E_{n-1}(f)_C\le \|f-t_{n-1}\|_C\le \int_0^{2\pi}
 |K_r(t)-t_{n-1}^{*}(t)|\, dt\cdot \|f^{(r)}\|_C. %\eqno(2)
 \end{equation}
 Отсюда следует оценка
 \begin{equation}\label{f19-3}
 E_{n-1}(W_1^{(r)})_C\le E_{n-1}(K)_L. %\eqno(3)
 \end{equation}
 На самом деле здесь имеет место равенство, т.\,е. справедливо следующее утверждение.

 \begin{teo}[Фавар] Для любых $n\ge 1,$~ $r\ge 1$
 $$
 E_{n-1}(W_1^{(r)})_C= E_{n-1}(K_r)_L.
 $$
 \end{teo}

Ввиду \eqref{f19-3}, для доказательства теоремы  достаточно доказать,
 что имеет место неравенство
 $$
 E_{n-1}(W_1^{(r)})_C\ge E_{n-1}(K_r)_L.
 $$
 С этой целью построим функцию $f^*\in W_1^{(r)},$ для которой
 $E_{n-1}(f^*)_C=E_{n-1}(K_r)_L.$
 Возьмем в {\eqref{f19-1}} наилучший в среднем полином
 $t_{n-1}^*$ для $K_r.$ Убедимся, что функция $\varphi^*=\sign (K_r-t_{n-1}^*)$
 является производной порядка $r\ge 1$ некоторой функции из
 $W_1^{(r)}.$

 Ограниченная измеримая $2\pi$-периодическая функция $\varphi$
 является  производной порядка $r\ge 1$ некоторой функции из
  $W^{(r)}$ в том и только  том случае, если  среднее значение функции
  $\varphi$ равно нулю. В нашем
 случае, как мы знаем, $\sign (K_r-t_{n-1}^*)=\sign \sin(nx+\alpha)$
 для соответствующего значения $\alpha,$ а потому $\displaystyle\int_0^{2\pi}\sign \{
 K_r-t_{n-1}^*\}\, dx=0.$ Значит, функция $\varphi^*=\sign(K_r-t^*_{n-1})$
 есть производная порядка $r$  некоторой функции $f^*\in W_1^{(r)};$
 эту функцию  можно восстановить по формуле (см.~(\ref{f18-12}) с учетом
 договоренности о символе $K_r$)
 $$
 f^*(x)=\int_0^{2\pi}\, K_r(\theta-\pi)\,
 \varphi^*(x+\theta)\, d\theta.
 %\eqno(\star)
 $$


 Ввиду свойств функции  $\varphi^*(x)=\sign \sin(nx+\alpha),$ справедливо соотношение
 $$
 f^*\left( x+\frac{\pi}{n}\right)=-f^*(x),\qquad x\in(-\infty,\infty);
 $$
а значит, функция $f^*$ имеет на $[0,2\pi)$ чебышевский $2n$-точечный
альтернанс. Из чего следует, что полином наилучшего равномерного приближения
функции $f^*$ есть тождественный нуль и потому
 $$
 E_{n-1}(f^*)_C=\|f^*\|_C=\int_0^{2\pi}
 |K_r(t)-t_{n-1}^*(t)|\, dt=E_{n-1}(K_r)_L.
 $$
 Следовательно, $E_{n-1}({W_1^{(r)}})_C=E_{n-1}(K_r)_L.$
 Теорема доказана.

 Итак, доказано, что
 $$
 E_{n-1}({W_1^{(r)}})_C=E_{n-1}(K_r)_L=\frac{M_r}{n^r}.
 $$
 Константы $M_r$ впервые были вычислены Фаваром и называются
 константами Фавара.  Имеем
 $$
 M_2=\frac{\pi}{8}\le M_r\le M_1=\frac{\pi}{2},\qquad r\ge 1;
 \qquad \lim_{r\to +\infty}M_r=\frac{4}{\pi}.
 $$





 Для любой функции $f\in C^{(r)}_{2\pi}$
 теперь можем написать
 \begin{equation}\label{f19-4}
 E_{n-1}(f)_C \le \frac{M_r}{n^r} \|f^{(r)}\|_C. %\eqno(4)
 \end{equation}
 Это неравенство называют неравенством Фавара.

 Применим неравенство Фавара к функции $f-t_{n-1},$ где $t_{n-1}$
 -- любой полином порядка $n-1;$ получим
 $$
 E_{n-1}(f)_C \le \frac{M_r}{n^r} \|f^{(r)}-t_{n-1}^{(r)}\|_C.
 $$
 Величина $\|f^{(r)}-t_{n-1}^{(r)}\|,$ вообще говоря, для любого $t_{n-1}$
 больше, чем $E_{n-1}(f^{(r)})_C,$
 так как $t_{n-1}^{(r)}$ имеет нулевое  среднее значение. Подбирая
 $t_{n-1}^{(r)}$ наилучшим образом, можно оценить
 \mbox{$\|f^{(r)}-t_{n-1}^{(r)}\|_C$} только через $2E_{n-1}(f^{(r)})_C,$
 так как свободный член в полиноме наилучшего приближения производной $f^{(r)}$
 оценивается величиной  $E_{n-1}(f^{(r)})_C.$ {Однако более}
 {аккуратные оценки позволяют избавиться от лишней двойки.}

 {Действительно,} для любой функции $f\in C^{(r)}_{2\pi}$ и произвольного
 тригонометрического полинома $\tau_{n-1}$ справедливо  представление
 $$
 f(x)-{t}_{n-1}(x) =\int_0^{2\pi} \{
 K_r(t)-t_{n-1}^*\}\{\varphi(x+t)-\tau_{n-1}(x+t)\}\, dt,\qquad \varphi=f^{(r)},
 $$
 в котором {$t_{n-1}$} -- некоторый тригонометрический полином порядка $n-1$,
 {соответствующий полиному $\tau_{n-1}$}.
 Выбрав в качестве  $\tau_{n-1}$ полином наилучшего равномерного приближения функции
 \mbox{$\varphi=f^{(r)}$} в $C_{2\pi},$
 получим полином {$t_{n-1},$} для которого
 $$
 \|f-{t}_{n-1}\|_C\le \frac{M_r}{n^r} E_{n-1} (f^{(r)})_C.
 $$
 Следовательно, имеет место неравенство
 \begin{equation}\label{f19-5}
 E_{n-1}(f)_C\le \frac{M_r}{n^r} E_{n-1}(f^{(r)})_C. %\eqno(5)
 \end{equation}
 %Так что изучение наилучших приближений функций $f\in C^{(r)}$  сводится к изучению наилучших приближений  недифференцируемых функций.

 Неравенство \eqref{f19-5} справедливо и для других {(классических)}
 пространств, но ни для одного из пространств {$L_{2\pi}^p$~ $(1\le p<\infty)$}
 константа $M_r$ не является точной.

 Рассмотрим далее приложения неравенства Фавара.

 %\section{Приложения неравенства Фавара}

 \section{Обобщение неравенства Бернштейна \\ на
 дифференцируемые функции}

 %\task %%%Задача.
 Пусть {$f\in C^{(r)}_{2\pi}.$} Оценим норму $\|f^{(r)}\|_C$ через $\|f\|_C.$

 Если $f=t_n,$ то имеет место неравенство Бернштейна
 $$
 \|f^{(r)}\|_C\le n^r \|f\|_C,
 $$
 которое обращается в равенство, например, на функции $f(x)=\sin nx.$
 Если $f\ne t_n,$ то неравенство Бернштейна уже не имеет места.
  На самом деле
 имеет место следующее утверждение.

 \begin{teo}\label{t-o-bern} При $r\ge 1$ для функций {$f\in C^{(r)}_{2\pi}$}
 имеет место обобщенное неравенство Бернштейна
 \begin{equation}\label{f19-6}
 \|f^{(r)}\|_C\le n^r\|f\|_C+A_rE_n(f^{(r)})_C, %\eqno(6)
 \end{equation}
 где $A_r$ есть некоторая константа, зависящая только от $r.$
 \end{teo}



 Если $r$ фиксировано, а $n\to \infty,$ то $E_n(f^{(r)})_C\to 0$ и
 $A_rE_n(f^{(r)})_C$ мало при больших $n.$

 %Д\;о\;к\;а\;з\;а\;т\;е\;л\;ь\;с\;т\;в\;о\quad неравенства {\eqref{f19-6}} проведем в 2 этапа.

Предварительно докажем имеющую самостоятельный интерес теорему об
одновременном приближении функции и ее производных.

\begin{teo}\label{t19-3}
Для любой функции $f\in C_{2\pi}^{(r)},\ r\in \mathbb N,$ и
ее полинома наилучшего приближения $t_n=t^*(f)$ справедливы
неравенства
\begin{equation}\label{f19-7}
\|f^{(k)}-{t}_n^{(k)}\|_C\le C_r E_{n}(f^{(k)})_C \qquad
k=1,2,\ldots,r
\end{equation}
с константой $C_r,$ зависящей только от $r.$
\end{teo}


% \begin{equation}\label{f19-7}
% \exists\ C_r\qquad \forall\ f\in C_{{2\pi}}^{(r)}\qquad \exists\ {t}_n\qquad
% \|f^{(k)}-{t}_n^{(k)}\|_C\le C_r E_{n-1}(f^{(k)})_C  %\eqno(7)
% \end{equation}
%для всех  $0\le k\le r.$ Если $r=0,$ то $C_r=1.$

% Пусть {$t_n=t_n^*$} -- полином наилучшего приближения функции  $f$ в $C.$

 Для доказательства рассмотрим  сумму Валле Пуссена
 $\sigma(f)=\sigma_{n+p,n}(f)$ при  $p=\left[
 \dfrac{n}{r}\right].$ Используя неравенства Бернштейна, Лебега,
 Фавара и теорему Никольского, при $0\le k\le r$  получим
 $$
 \begin{aligned}
 \|f^{(k)}-{t}_n^{(k)}\|_C &\le
 \|f^{(k)}-\sigma(f)^{(k)}\|_C+
 \|(\sigma(f)-{t}_n)^{(k)}\|_C\le\\
 &\le \|f^{(k)}-\sigma(f)^{(k)}\|_C+
 (n+p)^k \|\sigma(f)-{t}_n\|_C=\\
 &=\|f^{(k)}-\sigma(f^{(k)})\|_C+
 (n+p)^k \|\sigma(f-{t}_n)\|_C\le\\
 &\le (\|\sigma\|+1) E_{n}(f^{(k)})_C+(n+p)^k
 \|\sigma\|E_{n}(f)_C\le \\
 &\le (\|\sigma\|+1) \Big\{ E_{n}(f^{(k)})_C+(n+p)^k E_n(f)_C\Big\}\le\\
 &\le (\|\sigma\|+1) \Big\{ E_{n}(f^{(k)})_C+ M_k\left( \frac{n+p}{n+1}
 \right)^k E_n(f^{(k)})\Big\},\\
 \end{aligned}
 $$
 {где $\|\sigma\|=\|\sigma_{n+p,n}\|_C^C$.}
  Поскольку по выбору  $p,$~ $p\le \dfrac{n}{r}$ и $0\le k\le r$ и
  $M_k\le \dfrac{\pi}{2},$ то имеем
 $$
\|f^{(k)}-{t}_n^{(k)}\|_C\le A(\|\sigma\|+1) E(f^{(k)})_C,
 $$
где $A$ -- некоторая абсолютная константа. Помимо того, в наших предположениях
$$
\|\sigma\|= O\left(\ln \frac{n+p}{n+1}\right)=O(\ln(r+1)).
$$
Следовательно для $t_n=t_n^*(f)$ справедлива оценка
 $$
 \|f^{(k)}-{t}_n^{(k)}\|_C\le O(\ln(r+1)) E_n(f^{(k)})_C
 $$
 с абсолютной константой, скрытой под знаком $O.$ Теорема~\ref{t19-3} доказана.


 Д\;о\;к\;а\;з\;а\;т\;е\;л\;ь\;с\;т\;в\;о\quad теоремы~\ref{t-o-bern}.
  Предположим, что полином {$t_n$} осуществляет одновременное
 приближение функции и ее производных,
 а точнее, обладает свойством {\eqref{f19-7}}.
 Тогда, используя   {\eqref{f19-7}}, неравенства Бернштейна
 и Фавара, получаем
 $$
 \begin{aligned}
 \|f^{(r)}\|_C &\le \|f^{(r)}-{t}_n^{(r)}\|_C+\|{t}_n^{(r)}\|_C\le\\
 &\le C_rE_n(f^{(r)})_C+n^r\|{t}_n\|_C\le \\
 &\le C_rE_n(f^{(r)})_C+n^r\|f\|_C+n^r\|f-{t}_n\|_C\le \\
 &\le n^r\|f\|_C+C_rE_n(f^{(r)})_C+n^r C_r E_n(f)_C \le \\
 &\le n^r\|f\|_C+(1+{M_r})C_rE_n(f^{(r)})_C,
 \end{aligned}
 $$
 где {$M_r$} -- константа {Фавара}. Обобщенное неравенство
 Бернштейна доказано.

 \section{Приложение неравенства Фавара \\ к оценке
 нормы интеграла}



 \begin{teo}\label{t-Favard} %%%Теорема.
 Пусть  $f\in C_{{2\pi}}^{(r)}$ и  $f\perp {t}_{n-1}$ для любого
 {$t_{n-1}\in {\cal T}_{n-1},$}  т.\,е. спектр функции {$f$} начинается с номера  $n$
 $($спектр отделен от нуля$)$. Тогда имеет место неравенство
 $$
 \|f\|_C\le \frac{M_r}{n^r} \|f^{(r)}\|_C.
 $$
 \end{teo}

Действительно, условие $f\perp {t}_{n-1},\ t_{n-1}\in {\cal T}_{n-1},$
влечет, что и $f^{(r)}\perp {t}_{n-1}.$ Поэтому
 $$
 f(x)=\frac{1}{\pi}\int_0^{2\pi}\{ K_r(t)-t_{n-1}^*(t)\} f^{(r)}(x+t)\, dt,
 $$
 откуда следует нужная оценка.

 \section{Неравенство Колмогорова}

 Сравним нормы $\|f\|_C,$~ $\|f^{(k)}\|_C,$~ $\|f^{(n)}\|_C$~ $(0<k<n)$
 для любой   функции $f\in C^{(r)}.$
 {Имеет место} следующее утверждение.

 \begin{teo}[неравенство Колмогорова]\label{t19-4}
 При любых $0<k<n$ существуют константы $K_{n,k}$ такие, что
 $$
 \|f^{(k)}\|_C\le K_{n,k}\|f\|^{\frac{n-k}{n}}_C\cdot
 \|f^{(n)}\|^{\frac{k}{n}}_C,\qquad f\in C_{{2\pi}}^{(n)}.
 $$
 \end{teo}

З\;а\;м\;е\;ч\;а\;н\;и\;е.\quad На самом деле $K_{n,k}$ равномерно ограничены
относительно параметров $n$ и $k$.

 Д\;о\;к\;а\;з\;а\;т\;е\;л\;ь\;с\;т\;в\;о\quad теоремы~\ref{t19-4}.
 Рассмотрим  суммы Валле Пуссена $\sigma=\sigma_{m,p}(f)$ функции $f\in C^{(n)}_{2\pi},$
 которые определены формулой~{(\ref{f16-1}).}
 Имеем
 $$
 \|f^{(k)}\|_C\le \|f^{(k)}-\sigma(f)^{(k)}\|_C+\|\sigma^{(k)}(f)\|_C.
 $$
 Применяя неравенство Лебега для сумм Валле Пуссена и неравенство Фавара
 для $k$-й производной, получим
 $$
 \|f^{(k)}-\sigma^{(k)}(f)\|_C= \|f^{(k)}-\sigma(f^{(k)})\|_C\le (\|\sigma\|+1)
 E_{m-p}(f^{(k)})_C\le
 $$
 $$
 \le (\|\sigma\|+1) {\frac{M_{n-k}}{(m-p+1)^{n-k}}} \|f^{(n)}\|_C,
 \qquad {(\|\sigma\|=\|\sigma\|_C^C).}
 $$
 По неравенству Бернштейна
 $$
 \|\sigma^{(k)}(f)\|_C\le m^k \|\sigma(f)\|_C\le m^k \|\sigma\|\cdot
 \|f\|_C.
 $$
 Следовательно,
 $$
 \|f^{(k)}\|_C\le (\|\sigma\|+1) M_{{n-k}} \left\{ \frac{1}{(m-p+1)^{n-k}}
 \|f^{(n)}\|_C+m^k \|f\|_C \right\}.
 $$
 Для правильного выбора параметра $m$ (считая, например, $p\le \dfrac{m}{2}$)
 по сути, нужно минимизировать  величину
 $$
 \min_{X}\left( X^{-(n-k)} \|f^{(n)}\|_C+X^k\|f\|_C\right).
 $$
 Здесь первое слагаемое убывает, второе возрастает;
 значит, $X$ целесообразно выбирать из условия
 $$
 X^{-(n-k)}\|f^{(n)}\|_C=X^k\|f\|_C,
 $$
 т.\,е. целесообразно взять значение $m$ равным
 $$
 X=\left( \frac{\|f^{(n)}\|_C}{\|f\|_C} \right)^{\frac{1}{n}}.
 $$
 Если выбрать такое значение $m$  и взять  $p=\dfrac{m}{2},$ то все будет доказано. Но $m$ (да и $p$)
 должно быть целым. Выбираем $m$ из условия $m\le X\le m+1.$
 Если $X<1,$ то положим $m=0,$~ $p=0.$
 Если $X\ge 1,$ то, к примеру, при $p=\Big[\dfrac{m}{2}\Big]$ будем иметь нужный порядок;
 окончательно получаем
 $$
 \|f^{(k)}\|_C\le K_{n,k} \|f\|_C^{\frac{n-k}{n}}\cdot
 \|f^{(n)}\|_C^{\frac{k}{n}},
 $$
 что и требовалось доказать.

 Справедливость замечания вытекает из неравенства
 А.\,Н.\,Колмогорова на числовой оси, доказанного им в 1939 г. с точной,
 ограниченной по всем параметрам константой.\footnote{Колмогоров\,А.\,Н.
 Избранные труды. Математика, механика. М.: Наука, 1985. С.~252--263.}
