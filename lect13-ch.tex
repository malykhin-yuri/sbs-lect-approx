% Лекции Сергея Борисовича Стечкина
% ??? Внесены исправления С.В. Конягина и И.Г. Царькова, версия 24.02.2009
% Внесены исправления Н.И. Черныха, версия 16.07.2009
% Внесена грамматическая и ТеХ-правка М.Дейкаловой, версия 05.08.09

\chapter{Теоремы единственности}

\section{Чебышевские ранги подпространств}

Пусть $X$ -- банахово пространство, $M$ -- конечномерное
подпространство в $X$ и $x\in X$. Рассмотрим $Y(x)$~-- множество
наилучших элементов в $M$ для $x$. Это выпуклое замкнутое множество.

Зафиксируем $y_0\in Y(x)$ и рассмотрим множество $\{y-y_0\}$,~ $y\in Y(x)$.
Число $r(x)$ линейно независимых элементов среди $\{y-y_0\}$
называется \textit{размерностью множества наилучших элементов}. Ясно, что
$$ 0\le r(x)\le n = \dim\ M. $$
Для чебышевских подпространств $r(x)=0$ для любого $x\in X$. Это есть
характеристика чебышевских подпространств.

Можно ввести числовые характеристики для всего
подпространства в целом, например,
$$
\sup_{x\in X} r(x)=\max_{x\in X} r(x)=R(M).
$$
Ясно, что $0\le R(M)\le n$. Число $R(M)$ называется чебышевским рангом подпространства
$M$. Для чебышевских подпространств $R(M)=0$.

Нижняя грань
$\inf\limits_{x\in X} r(x)$ всегда равна нулю (достигается на $x\in M$), так
что такую характеристику подпространства вводить не имеет смысла.

Оказывается, в $C[a,b]$ существуют подпространства любого чебышевского ранга.

В $L[a,b]$ ситуация иная. А.\,Л.\,Гаркави в 1964 г.
доказал, что в $L_{\mu}[a,b]$ в случае безатомной меры
$\mu$ ранг любого конечномерного подпространства совпадает
с его размерностью.

Можно изучать чебышевский ранг подпространства $M\subset X$ не относительно
всего пространства $X$, а относительно какого-то множества
$\mathfrak M \subset X$;~ $\mathfrak M$, вообще говоря,~-- не подпространство.

Чебышевским рангом подпространства $M$ относительно множества
$\mathfrak M$ в пространстве $X$ называется
$$
\sup_{x\in \mathfrak M} r(x)=R(M,\mathfrak M)_X\le R(M)_X.
$$

В классических пространствах $C$ и $L^1$ существуют $\mathfrak M$ и $M$, для которых
$$ R(M,\mathfrak M)_X < R(M,X). $$
Если разумно выбрать $\mathfrak M$, то содержательным может оказаться
понятие чебышевского подпространства в $X$ относительно $\mathfrak
M.$ Так будет, если

$$
0=R(M,\mathfrak M)_X < R(M,X)_X.
$$
Например, если в $L[a,b]$ взять $\mathfrak M=C[a,b]$, то относительно
$\mathfrak M$ существуют конечномерные чебышевские подпространства
(теорема Джексона, будет доказана).

\section{Периодический случай}

Пусть $C_{2\pi}$ -- пространство $2\pi$-периодических непрерывных функций.
Для аппроксимации аппарат многочленов здесь совершенно не приспособлен, но можно
говорить о приближении таких функций тригонометрическими полиномами порядка $n$:
$$
t_n(x)=a_0+\sum\limits_{k=0}^n (a_k \cos kx+b_k \sin kx).
$$
При правильном подсчете нулей (на $[0,2\pi)$) это~-- {подпространство}
Чебышева порядка $2n+1$ и справедлив аналог теоремы Чебышева:
тригонометрический полином порядка $n$ будет наилучшим {для функции}
тогда и только тогда, когда {у его} {уклонения от функции} имеется
чебышевский альтернанс длины $2n+2$ (на любом полуинтервале вида
$[\alpha,\alpha+2\pi)$; наличие альтернанса не зависит от выбора $\alpha$).

\section{Лакунарные тригонометрические ряды}

Утверждения в этом разделе даются без доказательств.

Пусть имеется ряд вида
\begin{equation}\label{l13-lakuna}
a_0+\sum\limits_{k=1}^{\infty} (a_{n_k}\cos n_k x+b_{n_k}\sin n_k
x).
\end{equation}
Будем считать, что $1\le n_1<n_2<\ldots<n_k<\ldots\,.$ Если $\{n_k\}$~--
достаточно редкая последовательность, то говорят, что ряд лакунарный.

\textit{Лакунарность по Адамару}: существует $\lambda>1$ такое, что
$$
\frac{n_{k+1}}{n_k}\ge \lambda>1\qquad \forall\ k,
$$
т.\,е. $n_k$ встречается по крайней мере не чаще, чем геометрическая
прогрессия со знаменателем $\lambda$:
$$
n_{k+1}\ge \lambda n_k,\qquad n_k\ge \lambda^{k-1} n_1\ge \lambda^{k-1}.
$$
Лакунарные тригонометрические ряды обладают рядом свойств, которыми не
обладают обыкновенные тригонометрические ряды.

\textit{Пусть {$f\in C_{2\pi}$} и пусть ее ряд Фурье имеет вид
\eqref{l13-lakuna} с адамаровскими лакунами, $\dfrac{n_{k+1}}{n_k}\ge
\lambda>1.$ Тогда этот ряд Фурье равномерно сходится к $f$ {$($А.\,Н.\,Колмогоров$)$;}
этот ряд не только равномерно, но и абсолютно сходится
{$($С.\,Сидон$)$.}}

Пусть $\rho_k=\sqrt{a_{n_k}^2+b_{n_k}^2}$. Оказывается, для непрерывных
функций $f$ с рядом Фурье вида~(\ref{l13-lakuna})
$$
\sum\limits_{k=0}^{\infty}\rho_k\le C(\lambda)\|f\|_C
$$
и при фиксированном $\lambda>1$
\begin{equation}\label{l13-rho}
\|f\|_C \asymp \sum\limits_{k=0}^{\infty} \rho_k.
\end{equation}

\begin{ex}
Возьмем для рассматриваемых $f$ симметричную разность с шагом $2h$
$$
\Delta_h f(x)=f(x+h)-f(x-h) = 2\sum\limits_{k=1}^{\infty} \sin
{ n_k h(-a_{n_k}\sin n_k x+b_{n_k}\cos n_k x)}.
$$
Применяя~(\ref{l13-rho}), оценить норму $\|\Delta_hf\|_C$ и модуль непрерывности
$\omega(f,\delta)=\sup\limits_{|h|\le \delta} \|\Delta_h f\|_C$
через коэффициенты ряда Фурье функции $f$.
\end{ex}


\section{О наилучших приближениях класса функций,\\
представимых лакунарными рядами}

Рассмотрим ряд Вейерштрасса
$$
\sum\limits_{k=1}^{\infty} a^k\cos b^k x,
$$
где $b\in\mathbb N$,~ $b>1$,~ $0<a<1$. При надлежащем соотношении $a$ и
$b$ этот ряд равномерно и абсолютно сходится к функции Вейерштрасса
$f\in C_{2\pi}$, которая нигде не дифференцируема.

\begin{teo}[С.\,Н.\,Бернштейн]
Если $b$ нечетно, то
$$
s_n(x)=\sum\limits_{k:\,b^k\le n} a^k\cos b^k x
$$
для любого $n$ является тригонометрическим полиномом, наилучшим
образом приближающим $f$ в метрике $C$ среди полиномов порядка не выше
$n$.
\end{teo}

\begin{proof}
Имеем
$$
f(x)-s_n(x)=\sum\limits_{b^k>n} a^k \cos b^k x.
$$
При $x=0$ эта функция достигает максимума
\begin{equation}\label{l13-bernshtein}
\|f-s_n\|=\sum\limits_{b^k>n} a^k=\sum\limits_{k=k_0}^{\infty}a^k,
\end{equation}
где $b^{k_0}>n\ge b^{k_0-1}.$ Выясним, сколько раз достигается максимум.
Период всех функций $a^k\cos b^kx$ при $b^k>n$ равен
$\dfrac{2\pi}{b^{k_0}}$. В точке $x=\dfrac{\pi}{b^{k_0}}$ все они равны $-1$
{($b$ нечетно)} и, значит,
$$
f\left(\frac{\pi}{b^{k_0}}\right) - s_n\left(\frac{\pi}{b^{k_0}}\right)=-\|f-s_n\|.
$$
Таким образом, $f-s_n$ имеет на $[0,2\pi)$ чебышевский альтернанс длины $2b^{k_0}\ge 2n+2$
{(см. рис.~\ref{r13-1}})

 \bigskip
\begin{figure}[ht]
\begin{center}
\includegraphics{pict13-1.eps}
\end{center}
 \bigskip
 \refstepcounter{ris}\label{r13-1}

 \centerline{Рис.~\theris}
 \bigskip
\end{figure}

и, следовательно, $s_n$ есть полином, наименее уклоняющийся от $f$.
\end{proof}

\begin{Remark}
Если составить ряд из синусов, то
$$
f(x)-s_n(x)=\sum\limits_{b^k>n} a^k \sin b^k x
$$
и формула (\ref{l13-bernshtein}) неверна. Если прибавить к $x$ число
$\dfrac{\pi}{b^{k_0}}$, то все синусы изменят знак и, хотя формула
(\ref{l13-bernshtein}) неверна, все равно на $[0,2\pi)$ есть
$2b^{k_0}$-точечный чебышевский
альтернанс, так что теорема остается верной и в этом случае.
\end{Remark}

\begin{teo}[С.\,Б.\,Стечкин]
Пусть функция {$f\in C_{2\pi}$} разлагается в лакунарный ряд Фурье~{$(\ref{l13-lakuna}),$}
$\dfrac{n_{k+1}}{n_k}\ge \lambda>1.$ Тогда
$E_n(f)_C\asymp \sum\limits_{n_k>n}\rho_k,$ точнее,
$$
C(\lambda)\sum\limits_{n_k>n}\rho_k \le E_n(f)_C\le {C_1(\lambda)}
\sum\limits_{n_k>n}\rho_k,
$$
где числа $C(\lambda),\ C_1(\lambda)$ зависят только от $\lambda,$ $0<C(\lambda)\le
C_1(\lambda)<\infty$.
\end{teo}

\begin{proof}
Отметим одно свойство коэффициентов Фурье. Так как $\cos kx$ ортогонален
любому тригонометрическому полиному $t_n$ порядка $n$,~ $n<k$, то
$$
a_k=\frac{1}{\pi}\int_0^{2\pi} f(x)\cos kx\, dx=\frac{1}{\pi} \int_0^{2\pi}
\{ f-t_n^*\} \cos kx \, dx,
$$
где $t_n^*$~-- полином, наименее уклоняющийся от $f$ в метрике $C$. Откуда получаем
$$
|a_k|\le E_n(f)_C\cdot \frac{4}{\pi}, \qquad {k>n}.
$$
Аналогично, для $b_k$
$$
|b_k|\le E_n(f)_C\cdot \frac{4}{\pi}, \qquad {k>n}.
$$
Заметим также, что если $\sum A_{n_k}$~-- лакунарный ряд с $\lambda>1,$
то между $n$ и $2n$
имеется ограниченное число членов, которое зависит только от $\lambda$
(порядка $\ln 2/ \ln \lambda$).

Рассмотрим разность $f-\sigma_{2n,n}$, где $\sigma_{2n,n}=\dfrac{1}{n+1}
\sum\limits_{k=n}^{2n} s_k$~-- суммы Валле Пуссена (см. лекцию~6). Как мы знаем,
$\|f-\sigma_{2n,n}\|\le 4E_n(f)_C$. Подсчитаем, как отличается сумма Валле
Пуссена от $s_n$ {для любого ряда с общим членом $A_k(x)$. Имеем
{(см. рис.~\ref{r13-2})}}

 \bigskip
\begin{figure}[ht]
\begin{center}
\includegraphics{pict13-2.eps}
\end{center}
 \bigskip
 \refstepcounter{ris}\label{r13-2}

 \centerline{Рис.~\theris}
 \bigskip
\end{figure}


$$
\sigma_{2n,n}-s_n=\frac{1}{n}\sum\limits_{k=n+1}^{2n}
s_k-s_n=\sum\limits_{k=0}^{2n}\lambda_k^{(n)} A_k(x)-\sum\limits_{k=0}^n A_k(x)=
{\sum\limits_{k=n+1}^{2n}\lambda_k^{(n)} A_k(x)},
$$
откуда, так как в $\sum\limits_{k=n+1}^{2n}$ {в случае лакунарного ряда} число
слагаемых зависит только от~$\lambda,$
$$
|\sigma_{2n,n}-s_n|\le \sum\limits_{k=n+1}^{2n} |A_k(x)|\le C(\lambda)
E_n(f)_C.
$$
Таким образом, получаем, что
$$
E_n(f)_C\le \| f-s_n\|\le
\|f-\sigma_{2n,n}\|+\|\sigma_{2n,n}-s_n\|\le C_1(\lambda) E_n(f)_C,
$$
т.\,е. для лакунарных рядов, если $\lambda>1,$ то
$$
\|f-s_n\|\asymp E_n(f)_C.
$$
Применим к $f-s_n$ теорему Сидона и неравенство~(\ref{l13-rho}),
получим {соотношение}
$$
E_n(f)_C\asymp \sum\limits_{n_k>n}\rho_k,
$$
верное для любой функции, ряд Фурье которой лакунарный с $\lambda>1$.
\end{proof}

\begin{Example}
Функция, для которой суммы Фурье по порядку дают наилучшие приближения:
$$
f=\sum\limits_{n=1}^{\infty} a_n\cos nx,\qquad \left| \frac{a_{n+1}}{a_n}\right|<q<1.
$$
Имеем
$$
\|\sigma_{2n,n}-s_n\|\le \sum\limits_{k=n+1}^{2n} |a_k|\le C(q)|a_{n+1}|\le
C(q) E_n(f)_C.
$$
Учитывая, что $\|f-\sigma_{2n,n}\|\le 4E_n(f)_C,$
получаем
$$
\|f-s_n\|\le \|f-\sigma_{2n,n}\|+\|\sigma_{2n,n}-s_n\|\le
4E_n(f)_C+C(q)E_n(f)_C=C_1(q)E_n(f)_C.
$$
\end{Example}

\begin{Remark}
Для функций многих переменных задача очень трудна.
\end{Remark}


\section{Приближение функций посредством\\ конечномерных подпространств в метрике $L[a,b]$}

Пусть $L=L[a,b]$~-- пространство интегрируемых по Лебегу функций на отрезке $[a,b]$.

Мы знаем, что одномерное подпространство из констант в $L$ не является чебышевским
(был пример).

\begin{ex}
Доказать, что в $L[a,b]$ никакое нетривиальное конечномерное подпространство
не является чебышевским.
\end{ex}

Пространство $C[a,b]$ непрерывных на отрезке $[a,b]$ функций является в $L$
всюду плотным линейным многообразием, $(\overline C[a,b])=L$.

\begin{teo}[Д.\,Джексон]
Пусть $(\varphi) = \{\varphi_1,\varphi_2,\ldots,\varphi_n\}$~--
чебышевская система непрерывных на $[a,b]$ функций, $L_n$~-- натянутое на
$\varphi$ $n$-мерное подпространство. Тогда для любой
функции $f\in C[a,b]$ найдется единственный полином $\varphi^*\in L_n$ такой, что
$\|f-\varphi^*\|_L=E(f,L_n)_L.$
%, т.\,е. {чебышевский ранг подпространства $L_n$}
%{относительно $C[a,b]$ в $L[a,b]$}~ {$R(L_n,C)_L=0$.}
\end{teo}

Предварительно докажем несколько лемм.

\begin{lemma}\label{l13-1}
Если $\psi_1$, $\psi_2$~-- полиномы наилучшего приближения {в $L[a,b]$ для}
{$f \in C[a,b]$}, то $(f(x)-\psi_1(x))(f(x)-\psi_2(x))\ge0$
для любой точки $x\in[a,b]$.
\end{lemma}

\begin{proof}
Пусть $\psi_1$ и $\psi_2$~-- наилучшие полиномы для $f$. Тогда
$\psi=\dfrac12(\psi_1+\psi_2)$~-- тоже наилучший полином для $f$ и
$$
\int_a^b |f-\psi|\,dt = \frac12\left\{ \int_a^b |f-\psi_1|\,dt + \int_a^b
|f-\psi_2|\,dt \right\},
$$
т.\,е.
$$
\int_a^b |f-\psi_1+f-\psi_2|\,dt =
\int_a^b |f-\psi_1|\,dt+ \int_a^b |f-\psi_2|\,dt.
$$
Последнее равенство {для непрерывных функций} имеет место только
в том случае, когда разности $f-\psi_1$ и $f-\psi_2$ имеют одинаковые знаки.
\end{proof}

\begin{lemma}\label{J-sign-change}
Пусть $\psi_1$, $\psi_2$~-- различные наилучшие {в $L[a,b]$} полиномы для
{$f \in C[a,b]$} по системе Чебышева, $\alpha\in (0,1),\ \varphi_{\alpha}(t)=
\alpha\psi_1(t)+(1-\alpha)\psi_2(t)$. Тогда разность
$f-\psi_{\alpha}$ имеет не более $(n-1)$ нулей.
\end{lemma}

\begin{proof}
Пусть $f(t)-\varphi_{\alpha}(t)=0.$ В силу
леммы~\ref{l13-1}, мы имеем также $f(t)-\psi_1(t)=f(t)-\psi_2(t)=0.$
Таким образом, все нули функции $f-\varphi_{\alpha}$
являются нулями $\psi_1-\psi_2.$ Но последняя функция имеет
не более $(n-1)$ нулей, так как $(\varphi)$ -- чебышевская
система.
%Если непрерывная функция в точке $a$ меняет знак, то в этой точке она равна нулю.
%Следовательно, во всех точках, где $f-\psi_1,$
%а, значит, где $f-\psi_2$, меняет знак, имеет место равенство $\psi_1-\psi_2=0$.
%Следовательно, таких точек может быть не более $n-1$, так как
% система $(\varphi)$~-- чебышевская.
\end{proof}

\begin{lemma}\label{J-integral}
Пусть на $[a,b]$ задана система Чебышева $(\varphi) = \{\varphi_1, \varphi_2, \ldots, \varphi_n\}$. Рассмотрим полиномы
$\varphi(t)=\sum\limits_{k=1}^n a_k \varphi_k(t)$ по этой системе, для которых
$$
\|\varphi\|_L=\int_a^b |\varphi(t)|\, dt=\mathcal D
$$
для некоторого $\mathcal D$. Тогда для любого измеримого множества $E$ из $[a,b]$
$$
J(E)=\int_E |\varphi(t)|\, dt \le K\mathcal D\,\mes E,
$$
где $K$ зависит только от $(\varphi)$.
\end{lemma}

\begin{proof}
На сфере $\sum\limits_{k=1}^n a_k^2=1$ функция $\ds\int_a^b |\varphi(t)|\, dt$ достигает минимума как непрерывная функция от $a_k$,~
$k=1,\ldots,n$; но этот минимум не может равняться нулю, так как система $(\varphi)$~-- чебышевская. Следовательно,
$$
\int_a^b |\varphi(t)|\, dt \ge C>0\qquad \forall \ (a_1,\ldots,a_n)\colon \ \sum a_k^2=1.
$$
Теперь если $l=\sqrt{\sum\limits_{k=1}^n a_k^2}\ne 0$ и
$\ds\int_a^b |\varphi(t)|\,dt=\mathcal D$, то
$\ds\int_a^b \frac{|\varphi(t)|}{l}\,dt\ge C>0,$
т.\,е. $\mathcal D = \ds\int_a^b |\varphi(t)|\, dt\ge Cl$.
Тогда $|a_k|\le l\le \dfrac{\mathcal D}{C}.$
В силу этой оценки для $|a_k|,$ имеем
$$
\int_E |\varphi(t)|\, dt\le \sum\limits_{k=1}^n |a_k| \int_E
|\varphi_k(t)|\,dt\le K\mathcal D \,\mes E,
$$
где $K$~-- некоторая константа, зависящая только от $\varphi_1,\varphi_2,\ldots,\varphi_n.$
\end{proof}

%\begin{proof}
Д\;о\;к\;а\;з\;а\;т\;е\;л\;ь\;с\;т\;в\;о~ теоремы Джексона. Надо доказать,
что если $f\in C[a,b]$, $\{\varphi_1(t),\ldots,\varphi_n(t)\}$~--
{система Чебышева}, то полином $\varphi(t)=\sum\limits_{k=1}^n a_k
\varphi_k(t)$, наименее уклоняющийся от $f$ в метрике $L$, единственен.

Из леммы~\ref{J-sign-change} следует, что если $R(t)=f(t)-\varphi(t)$
имеет не менее $n$ перемен знака,
то $\varphi$~-- единственный наименее уклоняющийся от $f$ полином.

Допустим, что существуют два полинома $\psi_1$ и $\psi_2$, наименее
уклоняющихся от $f.$
%(так, что каждый из них имеет $q\le n-1$ перемен знака).
Обозначим $R(t)=f(t)-\varphi_{\alpha}(t),$ где $\varphi_{\alpha}(t)$
~-- произвольный полином вида
$$
{\varphi_\alpha(t)}=\alpha\psi_1(t)+(1-\alpha)\psi_2(t),\qquad \alpha\in [0,1].
$$
{Покажем что} разность $R(t)$ обращается в нуль на множестве
положительной меры, не зависящей от $\alpha$, и это приведет к противоречию с
леммой~\ref{J-sign-change}.

{Перенумеруем} точки перемен знака разности $R(t)$:
$$
a<t_1<\ldots<t_q<b,\qquad q\le n-1.
$$
Дополним, если нужно, эти точки до $n-1$ штук на отрезке $[b-\delta,b]$, где $b-\delta>t_q$:
$$ t_1<\ldots<t_q<t_{q+1}<\ldots<t_{n-1}<b. $$
Так как $(\varphi)$~-- чебышевская система, то по доказанному в п.~11.4
существует полином $F(t)$ по этой системе, который обращается в
нуль только в этих $n-1$ точках, причем меняет в них знак.
Можно считать, что $\|F\|_C=1$. Например, возьмем $F(t)=
\widetilde F(t)/\|\widetilde F\|_C$ или $F(t)=
-\widetilde F(t)/\|\widetilde F\|_C,$ где
$$
\widetilde F(t)=\left|
\begin{array}{ccc}
\varphi_1(t) & \ldots &  \varphi_n(t)\\
\varphi_1(t_1) & \ldots &  \varphi_n(t_1)\\
\ldots & \ldots &  \ldots\\
\varphi_1(t_{n-1}) & \ldots &  \varphi_n(t_{n-1})
\end{array}
\right| .
$$
Знак при $F(t)$ выбирается так, что $\sign F(t)=\sign R(t)$
на отрезке $[a,b-\delta]$ (см. рис.~\ref{r13-3})

 \bigskip
\begin{figure}[ht]
\begin{center}
\includegraphics{pict13-3.eps}
\end{center}
 \bigskip
 \refstepcounter{ris}\label{r13-3}

 \centerline{Рис.~\theris}
 \bigskip
\end{figure}


Рассмотрим разность $R(t)-\eps F(t)$,~ $\eps>0$. Определим на $[a,b]$ три множества
$$
\begin{array}{ll}
r:\ & |R(t)|> \eps,\quad \sign R=\sign F,\\[2ex]
s:\ & |R(t)|\le \eps,\quad \sign R=\sign F,\\[2ex]
u:\ & \sign R(t)\ne \sign F.
\end{array}
$$
По построению функции $F$: $u\subset [b-\delta,b]$ и
$$
R(t)-\eps F(t)=f(t)-\widetilde\varphi(t),
$$
где $\widetilde\varphi$~-- некоторый полином по системе $(\varphi)$. Следовательно,
$$ \int_a^b |R(t)-\eps F(t)|\, dt\ge \int_a^b |R(t)|\,dt, $$
так как $\ds\int_a^b |R(t)|\,dt=E(f,(\varphi))_L$. Далее,
имеем
$$
\int_r |R(t)-\eps F(t)|\, dt=\int_r |R(t)|\, dt-\eps \int_r |F(t)|\,dt,
$$
так как $\sign R(t)=\sign F(t)$ и $|R(t)|> \varepsilon$,~ $|F(t)|\le 1$ при $t\in r;$
и
$$
\int_s |R(t)-\eps F(t)|\, dt\le \int_s |R(t)|\, dt+\eps \int_s |F(t)|\, dt.
$$
Тогда
$$
\int_a^b |R(t)|\, dt\le\int_a^b |R(t)-\eps F(t)|\, dt= \int_r+\int_s+\int_u\le
$$
$$
\le \int_r |R(t)|\, dt-\eps\int_r|F(t)|\, dt+\int_s|R(t)|\,
dt+ \eps \int_s |F(t)|\, dt+\int_u |R(t)|\, dt+\eps \int_u |F(t)|\, dt=
$$
$$
 =\int_a^b |R(t)|\, dt-\eps\int_r |F(t)|\, dt+\eps \int_{s\cup u}
|F(t)|\, dt.
$$
Значит,
$$
\int_r |F(t)|\, dt \le \int_{s\cup u} |F(t)|\,dt.
$$
Прибавим к обеим частям $\ds\int_{s\cup u} |F(t)|\,dt$. Используя
лемму~\ref{J-integral}, получим
$$
\mathcal D = \int_a^b |F(t)|\,dt\le 2 \int_{s\cup u} |F(t)|\,dt\le 2K\mathcal D\,
\mes(s\cup u),\qquad K=K\big((\varphi)\big).
$$
Отсюда
$$ \mes(s\cup u)\ge c>0, $$
где $c$ от $\mathcal D$ не зависит. Выбрав $\delta<\dfrac{c}{2}$, будем иметь
$$ \mes s\ge \frac{c}{2}>0, $$
но при $\eps\to 0$ множество $s$ стремится к множеству, на котором $R(t)=0$.
Таким образом, для любого $\alpha\in [0,1]$ разность
$R_{\alpha}=f(t)-\varphi_{\alpha}(t)$ обращается в нуль на множестве
$s_{\alpha}$, $\mes s_{\alpha}\ge \dfrac{c}{2}>0$.
Но это противоречит лемме~13.2.

%Теперь выберем столько
%$\alpha_1,\ldots,\alpha_{\mu}$, чтобы соответствующие {какие-нибудь}
%множества $s_{\alpha_1}$ и $s_{\alpha_2}$ пересекались по множеству
%$\sigma$ положительной меры {$\left(\dfrac{c}{2}\,\mu>(b-a)\right);$} тогда
%$$ R_{\alpha_1}(t)=f(t)-\varphi_{\alpha_1}(t)=0\qquad \forall\  t\in \sigma, $$
%$$ R_{\alpha_2}(t)=f(t)-\varphi_{\alpha_2}(t)=0\qquad \forall\  t\in \sigma,
%\qquad \mes \sigma>0. $$
%Тогда на множестве $\sigma$,~ $\mes \sigma>0$,~ $\varphi_{\alpha_1} -
%\varphi_{\alpha_2}=0$, т.\,е. некоторый полином
%$\varphi=\varphi_{\alpha_1}-\varphi_{\alpha_2}$ по системе $(\varphi)$
%обращается в нуль на множестве положительной меры. Это противоречит тому,
%что $(\varphi)$~-- система Чебышева.
%\end{proof}

\begin{ex}
Где в процессе доказательства использовалась непрерывность функций
$\varphi_k$~ $(k=1,\ldots,n)$ и $f$? В случае $n=1$ получается, что
$q\le 0$, т.\,е. $q=0$. Как здесь быть?
\end{ex}

\begin{Remark}
Доказательство теоремы не проходит для произвольной меры, в доказательстве
рассматривалась мера Лебега.
\end{Remark}
