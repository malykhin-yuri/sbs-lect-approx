% Лекции Сергея Борисовича Стечкина
% ??? Внесены исправления В.И.Бердышева
% Внесены исправления Н.И.Черныха, версия 23.07.2009
% Внесена грамматическая и ТеХ-правка М.Дейкаловой, версия 05.08.09


%%%%%%%%%%%%%%%%%%%%%%%%%%%%%
\chapter{Общие линейные задачи теории приближений}
%%%        Лекция 9.

\section{Выпуклость пространства $L^p$}

Продолжаем рассматривать пространства ${L^p},$~ $1\le p<\infty.$ При $p=2$ пространство
 ${L^2}$ гильбертово. Далее считаем, что $\mu$ -- мера
 Лебега.

Имеет место следующая, приводимая здесь без доказательства
 \begin{teo}    %%% Теорема.
 Полное нормированное пространство {$H$ гильбертово} тогда и только тогда, когда все его
 конечномерные подпространства, включая его, если оно конечномерно,
 евклидовы.
 \end{teo}


 \begin{Corollary} %%% Следствие.
 В гильбертовом пространстве верна плоская геометрия.
 \end{Corollary}

 \begin{Example} %%% Пример.
 Параллелограмм обладает свойством:
  сумма квадратов диагоналей равна сумме квадратов его четырех сторон,
  т.\,е.
 $$
 \|x+y\|^2+\|x-y\|^2=2(\|x\|^2+\|y\|^2)
 $$
 (закон параллелограмма). В курсе анализа
  доказывается, что уже это свойство характеризует гильбертово
  пространство.
 \end{Example}

 При $p\ne 2,$~ $p>1,$ пространство ${L^p}$ (если его размерность больше 1)
 не является гильбертовым,
 но является строго выпуклым и рефлексивным.

 В любом банаховом пространстве норма удовлетворяет  неравенству
 треугольника
 $$
 \|x+y\|\le \|x\|+\|y\|.
 $$
 В ${L^p}$ получаем
 $$
 \left\{ \int_Q {|x+y|}^p\, dt \right\}^{1/p} \le
 \left\{ \int_Q |x|^p\, dt \right\}^{1/p}+
 \left\{ \int_Q |y|^p\, dt \right\}^{1/p}
 $$
 -- классическое неравенство Минковского. Здесь
  равенство (в вещественном случае) будет тогда и только тогда, когда
 $x$ и $y$ положительно пропорциональны, т.\,е. существуют
 $\alpha,\beta \ge 0$ и $\alpha^2+\beta^2>0$ такие, что $\alpha
 x=\beta y,$ т.\,е. $x$ и $y$ лежат на одном луче, исходящем из начала
 (см. условный рис.~9.1).

 Это свойство называется свойством {\it строгой
  нормированности}; оно эквивалентно строгой выпуклости пространства.

  \begin{figure}[ht]
\begin{center}
\includegraphics{pict09-1.eps}
\end{center}
 \bigskip
 \refstepcounter{ris}\label{r9-1}

 \centerline{Рис.~\theris}
 \bigskip
\end{figure}


 Следовательно, для любого $p>1$ пространство ${L^p}$ строго выпукло. Пространство ${L^1}$
 не является строго выпуклым; для обоснования этого построим в ${L^1}[0,1]$
  два элемента $x$ и $y$ таких, что в неравенстве Минковского
  {выполняется равенство}, но $x$ и $y$ не будут положительно пропорциональны:
  {достаточно положить}
  $$
    {
    x(t) = \begin{cases}
      1, & t \in [0,1/2)  \\[7pt]
      0, & t \in [1/2,1]
    \end{cases},\qquad
    y(t) = \begin{cases}
      0, & t \in [0,1/2)  \\[7pt]
      1, & t \in [1/2,1]
    \end{cases}.}
  $$

 % \vspace{1cm}
 %%%%%%%%%%%%%%%%%%%%%%%%%%%%%%%%%%%%%%%%%%%%%%%%%%%%%%
 %\hbox to 0.5cm {}{\special{em:graph pict1.pcx}}
 %\vspace{6cm}
 %%%%%%%%%%%%%%%%%%%%%%%%%%%%%%%%%%%%%%%%%%%%%%%%%%%%%
 На рис.~\ref{r9-2} представлен еще один вариант функций
 $x$ и $y.$

 \bigskip
\begin{figure}[ht]
\begin{center}
\includegraphics{pict09-2.eps}
\end{center}
 \bigskip
 \refstepcounter{ris}\label{r9-2}

 \centerline{Рис.~\theris}
 \bigskip
\end{figure}

 Пример легко распространить на пространства $L_{\mu}.$

 %\bigskip

 \section{Равномерная выпуклость}

 Возьмем единичный шар $O_1$ в банаховом пространстве, проведем гиперплоскость
  на расстоянии $h<1$ от $\Theta_X,$ <<отрежем ломтик>> $l$ от шара $O_1$ (см. рис.~\ref{r9-3}).
   Будем брать $d(l)$
  -- диаметр <<ломтика>>. Теперь устремим $h$ к $1$ и рассмотрим
 $\lim\limits_{h\to 1} d(l).$

 \bigskip
\begin{figure}[ht]
\begin{center}
\includegraphics{pict09-3.eps}
\end{center}
 \bigskip
 \refstepcounter{ris}\label{r9-3}

 \centerline{Рис.~\theris}
 \bigskip
\end{figure}


\noindent В евклидовом пространстве $d(l)\to 0$~ $(h\to 1).$

 Пространство называется {\it равномерно выпуклым}, если
  диаметр $d(l) \rightrightarrows 0$~ $(h\to 1)$
  равномерно по всем гиперплоскостям {(по всем <<ломтикам>>)}.
  Ясно, что равномерно выпуклое пространство -- строго
  выпуклое.

 {Имеет место} следующая теорема, которую также приводим
 без доказательства.
 \begin{teo}[Кларксон] %%% Теорема
 Всякое равномерно выпуклое банахово пространство рефлексивно.
 \end{teo}

 Доказать равномерную выпуклость пространства ${L^p}$ при $p>1,$
 просто, следовательно, ${L^p},$~ $p>1$,  рефлексивно. ${L^1},$
 вообще говоря, не рефлексивно, за исключением
 вырожденных случаев, когда мера сосредоточена в
 конечном числе точек.

 \section{Общие линейные задачи теории приближений}

 {\bf Постановка задачи.} Пусть $X$ -- банахово пространство,
 $L$ -- собственное подпространство $(L\ne X)$ ($L$ замкнуто).

 Рассмотрим задачу о наилучшем приближении
 любого элемента $x\in X$ при помощи элементов $y\in L,$ т.\,е. рассмотрим
 задачу
 $$
 \inf_{y\in L} \|x-y\|_X=E(x,L)_X.
 $$

 \vspace{3mm}
 {\bf 1.~Проблема единственности}
 \vspace{3mm}

 Для любого $x\in X$ составим множество $Y(x)$
 (может быть, и пустое):
 $$
 Y(x)=\{ y^*\in L : \|x-y^*\|_X=E(x,L)_X\}
 $$
 -- множество (или многогранник) наилучших элементов;
 $$
 x\longmapsto Y(x)\subset L.
 $$

Далее исследуется следующая

 \task %%% Задача.
 При каких условиях на $X$ для любого $x\in X$ и любого $L$
 мощность $\mbox{card}\ Y(x)\le 1?$ Т.\,е. для каких $X$
 для любого $x\in X$ в любом подпространстве $L$
 имеется не более одного наилучшего элемента?

\begin{defi}
 Если для любого подпространства $L$ и для любого элемента $x\in X$
 найдется только один или {не найдется} ни одного наилучшего элемента, то будем
 говорить, что $X$ обладает {\it свойством единственности} $(U).$
 \end{defi}

 \begin{teo} %% Теорема.
 Для того чтобы банахово пространство обладало свойством $(U)$,
 необходимо и достаточно, чтобы оно было строго выпуклым.
 \end{teo}

 \begin{proof} %%% Доказательство.
 Д\;о\;с\;т\;а\;т\;о\;ч\;н\;о\;с\;т\;ь.~ Пусть $X$ строго выпукло. Предположим, что
 $(U)$ не выполняется, т.\,е. существуют $L\subset X,$ $x\in
 X$ и $y_1,y_2$ такие, что
 $$
 \{y_1,y_2\}\subset Y_L(x),\qquad y_1\ne y_2.
 $$
 Тогда
 $$
 \inf_{y\in L} \|x-y\|=\|x-y_1\|=\|x-y_2\|=E(x,L)_X>0.
 $$
 Рассмотрим элемент $y=\dfrac{1}{2}(y_1+y_2).$ Подсчитаем уклонение $x$
 от $y$:
 $$
 \|x-y\|=\left\| \frac{1}{2} (x-y_1)+\frac{1}{2}(x-y_2) \right\|.
 $$
 Элемент $x-y$ есть середина отрезка $[x-y_1,x-y_2],$ концы
 которого лежат на сфере $S$ радиуса $\rho=E(x,L)_X.$
 Тогда в силу строгой выпуклости пространства элемент $x-y$
 лежит строго внутри шара ${O}_{\rho},$ значит, имеет строго меньшую норму, т.\,е.
 $$
 E(x,L)_X \le \|x-y\|=\left\| \frac{1}{2}(x-y_1)+\frac{1}{2}(x-y_2)
 \right\|<\|x-y_1\|=\|x-y_2\|=E(x,L)_X
 $$
 -- противоречие.



 {Н\;е\;о\;б\;х\;о\;д\;и\;м\;о\;с\;т\;ь.}~ Пусть $X$ обладает свойством $(U),$
 надо доказать, что тогда $X$ строго выпукло. Предположим, что $X$
 не строго выпукло, тогда существует гиперплоскость $L,$
 расстояние от которой до {начала} $\theta_X$ равно 1 и которая
 касается единичной сферы по крайней мере в двух точках $s_1,s_2.$
 <<Сдвинем>> гиперплоскость так, чтобы $L\to L_1,$~ $\theta_X\to x_0,$~ $L_1$
 -- подпространство, т.\,е. $L_1\ni \theta_X,$ тогда $s_1\to y_1,$~ $s_2\to y_2.$

 \bigskip
\begin{figure}[ht]
\begin{center}
\includegraphics{pict09-4.eps}
\end{center}
 \bigskip
 \refstepcounter{ris}\label{r9-4}

 \centerline{Рис.~\theris}
 \bigskip
\end{figure}


 Рассмотрим элемент $x_0$ (см. рис.~\ref{r9-4}). Для него в подпространстве $L_1$
 есть по крайней мере два элемента наилучшего  приближения $y_1$
 и $y_2.$ Противоречие. Теорема доказана.
 \end{proof}

\begin{Remark} %%% Замечание.
  { Мы фактически доказали, что если $y_1\in Y(x),$~ $y_2\in Y(x),$
 т.\,е.
 $$
 E(x,L)=\|x-y_1\|=\|x-y_2\|,
 $$
 то $[y_1,y_2]\subset Y(x).$}
 \end{Remark}

{ Действительно, любой элемент $y$ из $[y_1,y_2]$
 представляется в виде $y=ty_1+(1-t)y_2$ при некотором $t\in [0,1].$
 Тогда
 $$
{E(x,L)\le} \|x-y\|=\|t(x-y_1)+(1-t)(x-y_2)\|\le t\|x-y_1\|+(1-t)\|x-y_2\|=E(x,L).
 $$}

  \begin{Corollary} %%% Следствие.
 Многогранник наилучших приближений -- всегда выпуклое множество $($и, значит, для
 строго выпуклого пространства он либо пуст, либо состоит
 только из одного элемента$)$.
 \end{Corollary}

 Из классических пространств пространства $C$ и ${L^1}$ не являются строго выпуклыми,
 следовательно, $C$ и ${L^1}$ не обладают свойством единственности.
 {Пространства} ${L^p},$~ $p>1,$ строго выпуклы, следовательно, обладают свойством
 единственности.

 В общем случае все подпространства $\{L\}$ пространства $X$
 делятся на обладающие и  не обладающие свойством единственности.
 {На {рис.~\ref{r9-5}} изображены сфера и два}
 {подпространства пространства на плоскости с нормой,
 определяемой этой сферой.

 \begin{figure}[ht]
\begin{center}
\includegraphics{pict09-5.eps}
\end{center}
 \bigskip
 \refstepcounter{ris}\label{r9-5}

 \centerline{Рис.~\theris}
 \bigskip
\end{figure}


 \vspace{3mm}
 {\bf 2.~Проблема существования}
  \vspace{3mm}

 Если для  любого $x\in X$ в любом подпространстве $L$
 найдется  хотя бы один наилучший элемент, то будем говорить, что $X$
 обладает свойством существования $(E).$

Будем говорить, что гиперплоскость {$L_1$} касается сферы
$S_1$, если существует элемент $y\in S_1=\{ z\in X:\ \|z\|=1\}$
такой, что $\inf\limits_{x\in {L_1}} \|x-y\|=0.$ Заметим, что касание не предполагает наличие точки касания.

Для банаховых пространств {имеет место}
 \begin{teo}[Джеймс] %%% Теорема
 Для того чтобы банахово пространство было рефлексивным, необходимо и
 достаточно, чтобы всякая гиперплоскость этого пространства,
 касающаяся единичной сферы, имела $($хотя бы одну$)$ точку касания
 {или, что то же самое, всякая гиперплоскость $L_x=\{y : f(y)=f(x)\}$ имела бы точку}
 {касания со сферой $S_E=\{z\,:\,\|z\|=E(x,L)\},$ где $L$~-- подпростанство $\{y:f(y)=0\}.$}
 \end{teo}

% \begin{proof} %%% Доказательство.
 Таким образом, пространство $X$
  рефлексивно тогда и только тогда, когда для любого $f\in X^*$
 найдется $x\in X$ такой, что $\|x\|=1$ и $|f(x)|=\|f\|$
 (т.\,е. на элементе $x$ достигается норма функционала).
 %Не ограничивая общности,
 %можно считать, что $\|f\|=1.$ Тогда множество $\{ y : f(y)=1\}$
 %-- гиперплоскость и должен существовать элемент $x,$~ $\|x\|=1$,
 %такой, что $f(x)=1.$
 %\end{proof}

 \begin{Example} %%% Пример.
 В $C[0,1]$ рассмотрим функционал $f(x)=\ds\int_0^1 \sign\sin 2\pi t\cdot
 x(t)\,dt.$ Норма этого функционала
 $$
 \|f\|=\int_0^1 |\sign\sin 2\pi t|\, dt=1,
 $$
 но она не достигается в $C,$ так как функция $x(t)=\mbox{sign} \sin
 2\pi t$ не принадлежит $C[0,1].$


 \noindent В этом примере точки касания гиперплоскости $f(x)=1$ с единичной сферой
 $S_1$ в $C[0,1]$ нет (как легко видеть, для любой $x(t)\in C[0,1],~ \|x\|_C=1,~
 \Longrightarrow |f(x)|<1$).
 \end{Example}

 \begin{teo} %%% Теорема.
 Банахово пространство обладает свойством $(E)$
 в том и только том случае, когда оно рефлексивно.
 \end{teo}

 \begin{proof} %%%Доказательство.
 1) Пусть $X$ не рефлексивно. Тогда по теореме Джеймса существует
 гиперплоскость $\{ f(x)=1\},$ {$\|f\|=1,$} которая не имеет точки касания с
 единичной сферой. Рассмотрим подпространство $L=\{{y}:f(y)=0\}.$
 Тогда для любого элемента $x$ такого, что $f(x)\ne 0$ в $L$
 нет ближайшего элемента (если бы был, то после
 <<сдвига>> получили бы точку касания для гиперплоскости $L_x$ {(см. рис.~9.6)}).

 \vspace{10mm}
 %%%%%%%%%%%%%%%%%%%%%%%%%%%%%%%%%%%%%%%%%%%%%%%%%%%%%%
 %\hbox to 0.5cm {}{\special{em:graph pict1.pcx}}
 %\vspace{6cm}
 %%%%%%%%%%%%%%%%%%%%%%%%%%%%%%%%%%%%%%%%%%%%%%%%%%%%%%%%%%
 %\noindent \hskip3.0cm {рис.}
 %\bigskip
\begin{figure}[ht]
\begin{center}
\includegraphics{pict09-6.eps}
\end{center}
 %\bigskip
 \refstepcounter{ris}\label{r9-6}

 \centerline{Рис.~\theris}
% \bigskip
\end{figure}

 2) Пусть $X$ рефлексивно. Докажем, что тогда оно обладает свойством
 $(E).$ Прежде всего заметим, что если $\{ x_n\}\in L$ и $x_n \dashrightarrow x$
 (слабо сходится к $x$), то $x\in L$ и
 $$
 \|x\|\le d =
 \mathrel{\mathop{\underline{\lim}}\limits_{n\to \infty}} \|x_n\|.
 $$

 Пусть теперь $L$ -- любое подпространство и $x$ -- любой элемент из
 $X,$ {$x\not \in L.$} Построим шары $O_{d+\varepsilon_n}=O_{d+\varepsilon_n}(x)$ (см. рис.~9.7) с центром в $x$
 и радиусом $d+\varepsilon_n,$ где $d=E(x,L)$ и $\varepsilon_n
 \downarrow 0.$ Рассмотрим множества $K_n=O_{d+\varepsilon_n}\cap L.$
 $\{K_n\}$ -- вложенная последовательность непустых ограниченных
 замкнутых множеств.  В рефлексивном пространстве такая последовательность
 имеет непустое пересечение. Действительно, возьмем
 $$
 x_n\in K_n,\qquad \|x-x_n\| \longrightarrow d.
 $$
 Последовательность $\{x-x_n\}$~-- слабо компактная. Известно, что из нее можно
 выбрать подпоследовательность $\{x-x_{n_k}\},$
 слабо сходящуюся к некоторому элементу $x-x_0,\ x_0\in \cap K_n.$ Следовательно,
 $x_0\in L$ и $\|x-x_0\|\le
 d.$ Так как $d=E(x,L),$ то строгого неравенства не может быть, так что, на самом деле
 $\|x-x_0\|=d$ и, значит, $x_0$ -- наилучший элемент. Теорема доказана.
 \end{proof}

 \bigskip
\begin{figure}[ht]
\begin{center}
\includegraphics{pict09-7.eps}
\end{center}
 \bigskip
 \refstepcounter{ris}\label{r9-7}

 \centerline{Рис.~\theris}
 \end{figure}



 \begin{Corollary} %%% Следствие.
 Всякое конечномерное подпространство есть множество существования.
 \end{Corollary}

 \begin{Remark} %%% Замечание.
 Всякое ограниченно компактное
 множество, т.\,е. множество, пересечение которого с любым
 шаром $O_d(x)$ компактно, есть множество существования.
 \end{Remark}

 \begin{Example} %%% Пример
 {(конечно-параметрического множества} не являющегося ограниченно компактным).
 В $C[0,1]$ рассмотрим множество рациональных дробей вида $R_1=\dfrac{a}{b+ct}\in C[0,1].$
 $R_1$ зависит от трех параметров $a,b,c;$ это множество не компактно в $C[0,1]$:
 последовательность $\left\{ \dfrac{1}{1+ct}\right\}$
 {при $c \to 0$}
 сходится на $(0,1]$
 к $0,$ а в точке {$t=0$} принимает значение, равное $1.$

 Пространство ${L^p},~ p>1,$ является и рефлексивным, и строго выпуклым,
 следовательно, в $L_p,~ p>1,$ любое подпространство есть и
 подпространство $(U)$, и подпространство~$(E).$

 Такие подпространства, являющиеся и
 подпространствами единственности, и множествами
 существования, называются {\it чебышевскими}
 подпространствами.
 \end{Example}
