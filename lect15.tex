% Лекции Сергея Борисовича Стечкина
% ??? Внесены исправления В.А. Юдина, версия 27.01.2009
% Внесены исправления Н.И. Черныха, версия 19.07.2009
% Внесена грамматическая и ТеХ-правка М.Дейкаловой, версия 05.08.09


\chapter{{Дифференцируемость и аппроксимации в \boldmath $L^2$}}

\section{Доказательство второй теоремы Джексона в ${L^2}$}

Пусть {$f\in L_{2\pi}^2,$~ $f(t)= \dfrac{a_0}{2}+\sum\limits_{k=1}^{\infty}(a_k \cos kt+b_k\sin kt),$ где знак равенства означает совпадение}
{связываемых им функций как элементов из $L_{2\pi}^2$ (ряд здесь сходится в $
L_{2\pi}^2$ и его сумма} {принадлежит $L_{2\pi}^2$).}

Дадим определение производной, связанное со структурой {$L_{2\pi}^2$}. {Ясно, что}
$$
\dfrac{\Delta_h f(t)}{h}\in L_{2\pi}^2,\qquad h>0,
$$
$$
\dfrac{\Delta_h f(t)}{h}=\sum\limits_{k=1}^{\infty}\dfrac{2\sin kh/2}{h}(-a_k\sin kt+b_k\cos kt).
$$
Если существует {$\varphi\in L_{2\pi}^2$ такая, что}
 $$
 \lim\limits_{{L_2}\atop {h\to 0}} \dfrac{\Delta_h f}{h}=\varphi,
 $$
  т.\,е. если
 $$
\left\| \dfrac{\Delta_h f}{h}-\varphi\right\|_{L_2}\to 0\qquad (h\to 0),
$$
 то говорим, что $f$ имеет производную {в смысле $L^2$, равную} $\varphi,$ {которую будем} {обозначать также через $f'$~
$(f'=\varphi).$ Отметим, что если функция $f\in L_{2\pi}^2$ абсолютно} {непрерывна, то в качестве $f'$
в смысле $L^2$ можно брать обычную
производную, если она интегрируема с квадратом.}


{При условии существования производной $f'$ в смысле $L^2$} законно почленное дифференцирование ряда Фурье функции $f$,
{так как тогда
$a_k(\varphi)=\lim\limits_{h\to 0} a_k(\Delta_h f/h)=kb_k$ и, аналогично,} {$b_k(\varphi)=-ka_k$, так что ряд Фурье для $\varphi$ можно
получить формальным} {дифференцированием под знаком суммы ряда Фурье функции $f$.}

Аналогично {$f'$} определяем вторую и следующие
производные {в смысле $L^2$}. Итак, если существует $f^{(r)}$ {в смысле $L^2$,}
то в $L_{2\pi}^2$
$$
f^{(r)}(t)=\sum\limits_{k=1}^{\infty} k^r \left\{ a_k \cos \left( kt+\frac{r\pi}{2}\right)+
b_k \sin \left( kt+\frac{r\pi}{2}\right)\right\},
$$
$$
\left\| f^{(r)}\right\|_{{L_{2\pi}^2}}^2=\sum\limits_{k=1}^{\infty}
k^{2r}(a_k^2+b_k^2)=\sum\limits_{k=1}^{\infty} k^{2r}\rho_k^2.
$$

Рассмотрим
 $$
 f(t)-s_n(t,f)=\sum\limits_{k=n+1}^{\infty} A_k(t),\qquad {A_k(t)=A_k(t,f)=a_k\cos kt+b_k\sin kt}.
 $$
 Имеем
$$
E_n^2(f)_{{L^2}}=\|f-s_n\|_{{L_{2\pi}^2}}^2=\sum\limits_{k=n+1}^{\infty}\rho_k^2,\qquad {\rho_k^2=a_k^2+b_k^2,}
$$
$$
E_n^2(f^{(r)})_{{L_{2\pi}^2}}=\sum\limits_{k=n+1}^{\infty} k^{2r}\rho_k^2 \ge
(n+1)^{2r} \sum\limits_{k=n+1}^{\infty} \rho_k^2=(n+1)^{2r} E_n^2(f)_{{L_{2\pi}^2}},
$$
т.\,е.
\begin{equation}\label{l15-E_n(f)}
E_n(f)_{{L_{2\pi}^2}}\le \frac{1}{(n+1)^{{r}}} E_n (f^{(r)})_{{L_{2\pi}^2}}
\end{equation}
или
$$
\|f-s_n\|_{{L_{2\pi}^2}}\le \frac{1}{(n+1)^r}\|f^{(r)}-s_n^{(r)}\|_{{L_{2\pi}^2}}.
$$
Перепишем последнее неравенство иначе, заметив что
$$
{\{\varphi \perp t_n\quad \forall\ t_n\in \tau_n\}\qquad\Longleftrightarrow\qquad s_n(\varphi)\equiv 0.}
$$
Таким образом, если $\varphi\perp t_n,$ то получаем неравенство (неравенство Фавара или Бэра\,--\,Фавара)
\begin{equation}\label{l15-phi-norm}
\|\varphi\|_{{L^2}}\le\frac{1}{(n+1)^r} \|\varphi^{(r)}\|_{{L^2}},
\end{equation}
т.\,е. если спектр функции достаточно удален от нуля, то норма функции {в $L_{2\pi}^2$} {мала по сравнению с нормой производной в смысле
$L^2$.}

\begin{Remark}
В других метриках, где наилучшее приближение {как правило} достигается {не на} суммах
Фурье,~(\ref{l15-E_n(f)}) и~(\ref{l15-phi-norm})~-- разные неравенства.
\end{Remark}

Ранее было доказано неравенство Бернштейна, из которого следует
\begin{equation}\label{l15-t_n-norm}
\|t_n^{(r)}\|_{{L^2}}\le n^r \|t_n\|_{{L^2}}
\end{equation}
т.\,е. неравенство~(\ref{l15-t_n-norm}) имеет место для функций, спектр
которых отделен от $\infty.$

В неравенстве~(\ref{l15-t_n-norm}) равенство может быть только в том
случае, когда
$$ t_n(t)=A_n(t)=a_n\cos nt+b_n\sin nt.$$
В неравенстве Бэра\,--\,Фавара равенство может быть тогда и только тогда, когда $\varphi=A_{n+1}(t).$ {До конца лекции, говоря о производной
$f^{(r)}$ порядка $r$ будем, как и выше,} {считать эту производную в смысле $L^2$; или считать, что $f^{(r-1)}$ абсолютно непрерывна} {и
$f^{(r)}\in L_{2\pi}^2$.}

По неравенству Джексона
$$E_n(f^{(r)})_{{L^2}}\le C\omega\left( \frac{\pi}{n},f^{(r)}\right)_{{L^2}}.$$
{Отсюда получаем второе неравенство Джексона --} оценку для наилучшего приближения $r$
раз дифференцируемой функции
$$
E_n(f)_{{L^2}}\le \frac{C}{(n+1)^r}\, \omega\left( \frac{\pi}{n},f^{(r)}\right)_{{L^2}}.
$$
 Используя оценку
$$
E_n(f^{(r)})_{{L^2}}\le C_k \omega_k \left(
\frac{\pi}{n},f^{(r)}\right)_{{L^2}},
$$
для наилучших приближений~$r$
раз дифференцируемой функции получим оценку:
$$
E_n(f)_{{L^2}}\le \frac{C_k}{(n+1)^r}\, \omega_k\left( \frac{\pi}{n},f^{(r)}\right)_{{L^2}}.
$$

Пусть для $f$ существует $f^{(r)}$ и
$$
t_n(t)=\frac{\alpha_0}{2}+\sum\limits_{k=1}^n {(}\alpha_k\cos kt+\beta_k \sin kt{)}
$$
-- некоторый приближающий полином. Оценим $\|f^{(r)}-t_n^{(r)}\|$
через $\|f-t_n\|$ и $f^{(r)}.$ Имеем
$$
\|f-t_n\|_{{L^2}}^2=\frac{(a_0-\alpha_0)^2}{2}+\sum\limits_{k=1}^n
\{(a_k-\alpha_k)^2+(b_k-\beta_k)^2\}+\sum\limits_{k=n+1}^{\infty}(a_k^2+b_k^2),
$$
$$
\|f^{(r)}-t_n^{(r)}\|_{{L^2}}^2=\sum\limits_{k=1}^n k^{2r} \{
(a_k-\alpha_k)^2+(b_k-\beta_k)^2\}+\sum\limits_{k=n+1}^{\infty}k^{2r}\rho_k^2
\le
$$
$$
\le n^{2r}\|f-t_n\|_{{L^2}}^2+E_n^2(f^{(r)})_{{L^2}}\le
\left( n^{r}\|f-t_n\|_{{L^2}}+E_n(f^{(r)})_{{L^2}}\right)^2,
$$
т.\,е.
\begin{equation}\label{l15-rth}
\|f^{(r)}-t_n^{(r)}\|_{{L^2}}\le n^r \|f-t_n\|_{{L^2}}+E_n
(f^{(r)})_{{L^2}}.
\end{equation}

\begin{Remark}
Неравенство~(\ref{l15-rth}) переносится на другие метрики {$L_{2\pi}^p$~ $(1<p<\infty)$}.
\end{Remark}

Рассмотрим теперь случай, когда полином хорошо
приближает функцию {в $L_{2\pi}^2$, в} {том смысле, что} он дает приближение порядка
наилучшего:
\begin{equation}\label{l15-tk1}
\|f-t_n\|_{{L^2}}\le AE_n(f)_{{L^2}}.
\end{equation}
Тогда по неравенству Фавара
$$
\|f-t_n\|_{{L^2}}\le \frac{A}{(n+1)^r}E_n(f^{(r)})_{{L^2}}
$$
и из~(\ref{l15-rth}) следует
\begin{equation}\label{l15-rth2}
\|f^{(r)}-t_n^{(r)}\|_{{L^2}}\le
(A+1)^r E_n(f^{(r)})_{{L^2}},
\end{equation}
т.\,е. производная от хорошо приближающего полинома дает для производной функции тоже порядок наилучшего приближения.


\section{Дифференциальные свойства функций \\ и свойства приближающих полиномов}

Из~(\ref{l15-rth2}) следует, что если {полиномы из некоторого множества хорошо приближают в}
{указанном выше смысле} $r$ раз
дифференцируемую функцию, то {нормы их} {производных порядка $r$ равномерно ограничены:}
$$
{\|t_n^{(r)}\|_{L_2}\le  C_r\|f^{(r)}\|_{L^2},}
$$
{где константа $C_r=C_r(A)$ зависит только от $r$ и константы $A$ в неравенстве~(\ref{l15-tk1}).}
Для наилучших {в $L_{2\pi}^2$ полиномов $s_n=s_n(f)$ здесь можно положить $C_r=1,$ так как в}
{силу тождества $s_n^{(r)}(f)\equiv s_n(f^{(r)})$ и равенства Парсеваля имеем равномерно по $n$ оценку}
$$
\|s_n^{(r)}\|_{{L^2}}\le \|f^{(r)}\|_{{L^2}}.
$$
В других пространствах $L_{2\pi}^p$~ $(1<p<\infty)$, учитывая ограниченность в них норм частных сумм Фурье,
для полиномов $t_n^*=t_n^*(f)$ наилучшего
приближения в $L_{2\pi}^p$ получаем
$$
\|t_n^{*(r)}\|_{L^p}\le C_r \|f^{(r)}\|_{L^p}.
$$

Пусть $f$ имеет модуль непрерывности
{$\omega(\delta,f)_{{L^2}}.$} Оценим $\omega(\delta,s_n)_{L^2}.$
Имеем
$$
\|\Delta_h s_n{(f)}\|_{{L^2}}^2=4\sum\limits_{k=1}^n \sin^2 \frac{kh}{2}
(a_{{k}}^2+b_k^2){=\|s_n(\Delta_h f)\|_{L^2}^2},
$$
{поэтому}
$$
\|\Delta_h s_n\|_{{L^2}}\le \|\Delta_h f\|_{{L^2}}
$$
и {для любого $n\in \mathbb{N}$}
$$
\omega(\delta,s_n)_{{L^2}}\le \omega(\delta,f)_{{L^2}},
$$
т.\,е. равномерно по $n$ наилучший полином обладает теми же
дифференциальными свойствами, что и функция.

{Аналогично, для любого $n$ для $r$ раз дифференцируемых (в смысле $L^2$) функций}
$$
\omega(\delta,s_n^{(r)})_{L^2}\le \omega(\delta,f^{(r)})_{L^2}.
$$
Рассмотрим {величины} $\|\Delta_h t_n\|_{{L^2}}^2,$~ $\|\Delta_h^{r}
t_n\|_{{L^2}}^2,$~ $\|t_n^{(r)}\|_{{L^2}}^2$ {для любого
полинома $t_n\in {\cal T}_n$}. Для конечной разности имеем оценку через производные
\begin{equation}\label{15-7}
\|\Delta_h^{r} t_n\|_{{L^2}}\le |h|^r \|t_n^{(r)}\|_{{L^2}}.
\end{equation}

%\begin{excercise}
\begin{ex}
Получить эту оценку. (Указание: записать {$t_n$} через кратный интеграл {от $t_n^{(r)}$}).
\end{ex}
%\end{excercise}

Теперь получим оценку производной через конечные
разности. Имеем
$$
\|\Delta_h^{r} t_n\|_{{L^2}}^2 = 2^{2r}\sum\limits_{k=1}^n
\sin^{2r}\frac{kh}{2}\rho_k^2,
$$
$$
\|t_n^{(r)}\|_{{L^2}}^2=\sum\limits_{k=1}^n k^{2r}\rho_k^2,
$$
{где $\rho_k^2=a_k^2(t_n)+b_k^2(t_n)$.} Заметим, что отсюда
немедленно вытекает оценка~(\ref{15-7}).
Хотим получить неравенство типа
$$
\|t_n^{(r)}\|_{L^2}\le c_n(h) \|\Delta_h^{r} t_n\|_{L^2}.
$$
Для этого достаточно найти константу $c_n(h)$ такую, что
$$
k^{2r}\le 2^{2r}\sin^{2r} \frac{kh}{2} c_n^2(h)\qquad \forall\ k=1,\ldots,n
$$
или
$$
\frac{k}{2}\le \widetilde{c_n}(h)\left| \sin\frac{kh}{2}\right|\qquad \forall\ k=1,\ldots,n.
$$
Ясно, что
$$
 \widetilde{c_n}(h)=\max_{k=1,2,\ldots,n}\frac{k/2}{|\sin kh/2|}.
$$
Для конечности $ \widetilde{c_n}(h)$ величина $\left|\sin \dfrac{kh}{2}\right|$ должна здесь
быть больше нуля для любого $k$, значит, должно быть $\dfrac{k|h|}{2}<\pi,$ т.\,е. $|h|< \dfrac{2\pi}{n}.$
{Учитывая, что функция $\dfrac{\sin x}{x}$ убывает на отрезке $[0,\pi]$,
имеем} при $\dfrac{k|h|}{2}<\pi,$
$$
\left| \frac{\sin kh/2}{kh/2}\right|={\frac{\sin kh/2}{kh/2}\ge\frac{\sin nh/2}{nh/2}\qquad
\left(k=1,2,\ldots,n;\quad h\le \frac{2\pi}{n}\right),}
$$
откуда получаем при $\dfrac{k|h|}{2}<\pi$
$$
\max_{k=1,\ldots,n}\frac{k/2}{|\sin kh/2|}=\widetilde{c_n}(h)=\frac{{n}}{2|\sin
{nh}/{2}|}.
$$
Поведение функции $\widetilde{c_n}(h),\
0<h<\dfrac{2\pi}{n}$ отражает график, представленный на
рис.~\ref{r15-1}.
Таким образом, если $|h|<\dfrac{2\pi}{n},$ то
$$
\|t_n^{(r)}\|_{{L^2}}\le \left( \frac{n}{2\sin {n|h|}/{2}}\right)^r
\left\| \Delta_h^{(r)}t_n\right\|_{{L^2}}
$$
-- неравенство Стечкина.

\begin{figure}[ht]
\begin{center}
\includegraphics[width=0.5\textwidth]{pict/pict15-1.eps}
\end{center}

 \refstepcounter{ris}\label{r15-1}
 \centerline{Рис.~\theris\ \ График функции $h\widetilde{c_n}(h)$}
\end{figure}


В частном случае при $h=\dfrac{\pi}{n}$
$$
\|t_n^{(r)}\|_{{L^2}}\le \left( \frac{n}{2}\right)^r \left\| \Delta_h^{r}t_n\right\|_{{L^2}}
$$
и, так как
$$
\left\| \Delta_h^{r} t_n\right\|_{{L^2}}\le 2^r \|t_n\|_{{L^2}},
$$
то это есть усиление неравенства Бернштейна из п.~14.2.


\section{Дифференциальные свойства приближающих полиномов}

Пусть  {$f\in L_{2\pi}^2$} и модуль непрерывности $\omega(\delta,f)$ {в $L_{2\pi}^2$} известен.
Пусть $t_n$~-- хорошо приближающий полином {в том смысле, что}
\begin{equation}\label{l15-tk2}
\|f-t_n\|_{{L^2}}\le A\omega\left(
\frac{\pi}{n},f\right)_{L^2}
\end{equation}
(такие полиномы по неравенству Джексона существуют).

Изучим $\omega(\delta,t_n)=\omega(\delta,t_n)_{L^2}.$ Имеем
$$
\|\Delta_h t_n\|_{{L^2}}\le \|\Delta_h f\|_{{L^2}}+\|\Delta_h
(f-t_n)\|_{L^2}\le \omega(h,f)+2\|f-t_n\|_{{L^2}}\le
\omega(h,f)+2A\omega\left(\frac{\pi}{n},f\right).
$$
Пусть $h\ge \dfrac{\pi}{n}.$ Тогда
$$
\|\Delta_h t_n\|_{{L^2}}\le (2A+1)\omega (h,f),
$$
так как $\omega\left( \dfrac{\pi}{n}\right) \le \omega(h)$ при $h\ge \dfrac{\pi}{n}.$

Если $h=\dfrac{\pi}{n},$ то
$$
\left\| \Delta_{\frac{\pi}{n}}t_n \right\|_{{L^2}}\le (2A+1)\omega\left(
\frac{\pi}{n},f\right).
$$
Оценим норму производной. В силу неравенства Стечкина
$$
\|t_n'\|_{{L^2}}\le \frac{n}{2}
\left\| \Delta_{\frac{\pi}{n}}t_n \right\|_{{L^2}}\le \frac12 n\omega \left(
\frac{\pi}{n},f\right)=o(n),
$$
а по неравенству Бернштейна получили бы только
$$
\|t_n'\|_{{L^2}}\le n\|t_n\|_{{L^2}}=O(n).
$$

Теперь оценим $\|\Delta_h t_n\|_{{L^2}}$ для всех $h,$~ $0<h<\dfrac{\pi}{n}.$

Докажем сначала неравенство
$$
\omega(\lambda\delta,f)\le (\lambda+1)\omega(\delta,f).
$$
Если $k$ целое, то
$$
\Delta_{kh}f={\sum\limits_{\nu=0}^{k-1} \Delta_h f\left(t+\nu h+\frac{1-k}{2}h\right)}
$$
и
$$
\omega(k\delta,f)\le k\omega(\delta,f).
$$
Теперь, если $k\le \lambda < k+1,$ то
$$
\omega(\lambda\delta,f)\le \omega((k+1)\delta,f)\le
(k+1)\omega(\delta,f)\le (\lambda+1)\omega(\delta,f).
$$
Тогда {при $0<h<\dfrac{\pi}{n}$}
$$
\|\Delta_h t_n\|_{{L^2}}\le h\|t_n'\|_{{L^2}}\le
\frac12 nh \omega\left( \frac{\pi}{n},f\right){=}
$$
$$
= \frac12 nh \omega\left( \frac{\pi}{nh}\cdot h,f\right)\le
\frac12 nh \left( \frac{\pi}{nh}+1\right)\omega(h,f)\le \pi\omega(h,f).
$$
Итак, для  любого {$h>0$}
$$
\|\Delta_h t_n\|_{{L^2}}\le C \omega(h,f)_{L^2},
$$
откуда
$$
\omega(h,t_n)_{L^2}\le C \omega(h,f)_{L^2},
$$
где $C=\max\{2A+1,\pi\}$ {зависит только от константы $A$ в~(\ref{l15-tk2})} и
не зависит ни от $n$, ни от $h,$ т.\,е. дифференциальные свойства хорошо
приближающих полиномов равномерно по $n$ такие же, как дифференциальные свойства функций.

Аналогично доказывается, что
$$
\omega_k(h,t_n)_{L^2}\le C_k \omega_k(h,f)_{L^2},\qquad {C_k=C_k(A).}
$$

Очевидно, что имеет место и обратное утверждение, так как
$$
\Delta_h f=\Delta_h(f-t_n)+\Delta_n(t_n)
$$
и {$\|f-t_n\|_{L^2} \le A \omega\left(\dfrac{\pi}{n},f\right).$}

Итак, для того чтобы {полиномы $t_n(t)$~ $(n\in N)$ с заданным модулем непрерывности}
$\omega(\delta)$ (т.\,е. $\omega(\delta,t_n)_{L^2}\le \omega(\delta)$)
{были хорошо приближающими функцию $f$ в смысле равномерной по
$n$ оценки~(\ref{l15-tk2}),} необходимо и достаточно, чтобы сама функция имела такой же модуль
непрерывности (точнее, $\omega(\delta,f)_{L^2}\le C(\omega(\delta)).$
