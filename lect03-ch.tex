% Лекции Сергея Борисовича Стечкина
% Внесены исправления Ю.Н.Субботина и Н.И.Черныха, версия 30.06.2009
% Внесены исправления Н.И.Черныха, версия 29.07.2009
% Внесена грамматическая и ТеХ-правка М.Дейкаловой, версия 05.08.09

\chapter{Многочлены Чебышева (продолжение).\\ {Интерполяция (приложения)}}  %%{Лекция 3.}

{\section{Теорема В.\,А.\,Маркова }}

{\bf 1. Третье экстремальное свойство многочленов Чебышева}

\ \

Рассмотрим задачу В.\,А.\,Маркова. При фиксированных {$m\in \bN$ и $n \ge m$}
среди всех многочленов степени
{$n$ с коэффициентом} $a_m=1,$ какой многочлен наименее
уклоняется от нуля {на $[-1,1]$}?


Для ее решения предварительно докажем следующую лемму.

\begin{lemma}[Лемма о нулях]\label{l2-3}
Пусть дан {отличный от тождественного нуля} $(N+1)$-членный полином
$\sum\limits_{k=0}^{N}A_kx^{\lambda_k},$ $0 \le \lambda_0 < \lambda_1 < \cdots <
\lambda_N.$ Тогда на $(0,\infty )$ он может иметь не более $N$ нулей.
\end{lemma}

 %\begin{proof}
Д\;о\;к\;а\;з\;а\;т\;е\;л\;ь\;с\;т\;в\;о\ \ индукцией по $N.$
При $N=0$ {и $A_0 \ne 0$} {полином} $A_0 x^{\lambda_0}$
не имеет нулей на $(0, \infty ).$ При $N=1$ {и $|A_0|+|A_1|
\ne 0$ полином} $A_0 x^{\lambda_0}+A_1
x^{\lambda_1}=x^{\lambda_0}
                                   (A_0+A_1 x^{\lambda_1-\lambda_0})$
может иметь не более одного нуля на~$(0,\infty).$ Пусть
{любой $(N+1)$-членный полином
$\sum\limits_{k=0}^{N}A_kx^{\lambda_k} \not \equiv 0$ } имеет
не более $N$ нулей на~$(0, \infty).$ Тогда
{$G(x)=\sum\limits_{k=0}^{N+1}A_kx^{\lambda_k}=
                        x^{\lambda_0}\sum\limits_{k=0}^{N+1}A_kx^{\lambda_k-\lambda_0}=
                        x^{\lambda_0}\cdot F(x)$}
имеет столько нулей на $(0, \infty ),$ сколько их имеет
$F(x)=\sum\limits_{k=0}^{N+1}A_kx^{\lambda_k-\lambda_0},$
при этом можно считать, что $A_{N+1} \ne 0$ {(иначе
$G(x)$~-- $(N+1)$-членный полином).} Но {тогда}
$F'(x)=\sum\limits_{k=1}^{N+1}B_kx^{\lambda_k-\lambda_0-1}$~-- {$(N+1)$}-членный
{не тривиальный} полином
{$(B_{N+1}=(\lambda_{N+1}-\lambda_0) A_{N+1}\ne 0)$}, который по
предположению индукции имеет не более $N$ нулей. Следовательно,
{по теореме Ролля} $F(x)$ имеет не более $N+1$ нулей. Лемма
доказана. %\end{proof}

Теперь вернемся к задаче В.\,А.\,Маркова.

Пусть $C=C[-1,1],$~ $p_n \in {\Cal P}_n,$~ $p_n(x)=\sum\limits_{k=0}^{n}a_kx^k,$~
${\Cal Q}^m_{n}=\{p_n(x):\ p_n\in {\Cal P}_n,\ a_{m}=1\}.$

%\begin{task}
Требуется найти $p_n^* \in \Cal P_n$
такой, что
\[
  \inf_{p_n \in {\Cal Q}^m_n} \|p_n(\cdot)\|_{C[-1,1]} =
  \min_{p_n \in {\Cal Q}^m_n} \|p_n(\cdot)\|_{C[-1,1]} =
  \|p^*_n(\cdot)\|_{C[-1,1]}.
\]
Эту задачу можно также рассматривать как задачу о
наилучшем приближении, именно
\[
  \inf_{p_n \in {\Cal Q}^m_n} \|p_n(\cdot)\|_{C[-1,1]} =
  \inf_{q \in {\Cal P}_n,\ a_m=0} \|x^m-q(x)\|_{C[-1,1]} =
  E(x^m,L_n^m)_{C[-1,1]},
\]
где $L_n^m=\{p\in {\Cal P}_n:\ a_m=0\}$ -- конечномерное линейное подпространство пространства
$C,$ которое обладает как и ${\Cal P}_n$ свойством: если возьмем
четную или нечетную часть какого-нибудь многочлена из этого
подпространства, то снова получим многочлен из этого подпространства.
%\end{task}

Заметим, что среди многочленов {с $a_m=0$}, которые приближают функцию $x^m$
наилучшим образом, {найдется наилучший} многочлен {той же
четности, что} $x^m$.

Действительно, пусть $m$ четное и $p_n(x)$~-- наилучший многочлен из нашего
подпространства. Тогда {$q_n(x)=\dfrac{1}{2}\{p_n(-x)+p_n(x)\}$}
тоже многочлен из того же подпространства, но уже четный, и приближает
{функцию $x^m$} не хуже:
\[
  \|x^m-q_n(x)\|_C=\Bigl\|\frac{1}{2}(x^m-p_n(x))+
                     \frac{1}{2}((-x)^m-p_n(-x))\Bigr\|_C \le
  \|x^m-p_n(x)\|_C.
\]
Следовательно, четные функции приближаются четными
многочленами наилучшим образом. Аналогичный вывод можно сделать для нечетных
функций и нечетных многочленов. Отсюда следует, в частности,
что если $m$ и $n$ разной четности, то наилучший многочлен
{$q_n^*$} той же четности, что и $m$, должен иметь степень
$n-1,$ и следовательно, {$q_n^*\equiv q_{n-1}^*.$}

 Введем в рассмотрение {следующее представление многочлена}
Чебышева:
\[
  T_n(x)=\cos{( n \arccos x)} =\sum\limits_{k=0}^n A_k^nx^k.
\]
\begin{teo}[{В.\,А.\,Марков}]
Если $m$ и $n$ имеют одинаковую четность, то среди всех многочленов степени $n$ с
коэффициентом $a_m=1$ {$(0\le m\le n)$} наименее уклоняется от
нуля многочлен $\dfrac{T_n(x)}{A_m^n},$ с уклонением
\[
  \Bigl\|\frac{T_n(\cdot )}{A_m^n}\Bigr\|_{C[-1,1]}=\frac{1}{|A_m^n|}.
\]
Если же $m$ и $n$ имеют разную четность, то среди всех многочленов степени $n$ имеющих
$a_m=1$ {$(0\le m\le n-1),$} наименее уклоняющимся от нуля является многочлен
$\dfrac{T_{n-1}(x)}{A_m^{n-1}},$ с уклонением
\[
  \Bigl\|\frac{T_{n-1}(\cdot )}{A_m^{n-1}}\Bigr\|_{C[-1,1]}
                                             =\frac{1}{|A_m^{n-1}|}.
\]
\end{teo}

\begin{proof}
Для доказательства надо рассмотреть 4 случая для $m$ и $n$ разных четностей. Докажем
{теорему только в одном случае}, когда $m$ и $n$ четные. Итак, надо доказать, что если
$p_n(x)=\sum\limits_{k=0}^n a_kx^k$~-- произвольный многочлен с $a_m=1$ {и четными
$n$ и $m,$} то
\[
  \| p_{n}(\cdot ) \|_{C[-1,1]} \ge \frac{1}{|A_m^{n}|}.
\]


Пусть это утверждение неверно, т.\,е. пусть существует многочлен
{$p_n(x)$} такой, что
\begin{equation}
\label{Macar}
  \| p_{n}(\cdot ) \|_{C[-1,1]} {<} \frac{1}{|A_m^{n}|}.
\end{equation}
{Многочлен} $p_n$ может {не} быть четным, но тогда {$\dfrac{1}{2}(p_n(x)+p_n(-x))
\in \Cal Q^m_n$} тоже удовлетворяет неравенству~(\ref{Macar}) и будет четным.
{Поэтому далее считаем, что в~(\ref{Macar}) $p_n$ --} {четный многочлен, отличный от
$\dfrac{T_n(x)}{A_m^n}$.} Многочлен Чебышева $T_n(x)$ тоже четный. Рассмотрим многочлен
\[
  R_{n}(x)=\frac{T_n(x)}{A_m^{n}}-p_n(x)=
  \sum\limits_{k=0}^{n/2}b_kx^{2k}{\not\equiv 0,}
\]
{где} $b_{m/2}=0.$ Таким образом, у $R_n(x)$ не более, чем $n/2=l$ не равных нулю
коэффициентов. По {лемме~\ref{l2-3}} о нулях $l$-членный полином
{$R_n(x)\not\equiv0$} может иметь на $(0,\infty)$ и, следовательно, на $(0,1)$ -- не
более $l-1$ нулей. У нас же $T_n(x)$ имеет на $[0,1]$ ровно $n/2+1$ точек
{$\widetilde{x}_k=\cos\dfrac{k\pi}{n} \
 \left(k=0,1,\ldots,\dfrac{n}{2} \right)$} максимального уклонения, в которых
$R_n(x)$ {в силу~(\ref{Macar})} имеет тот же знак, что и $\dfrac{T_n(x)}{A_m^{n}}$ и,
значит, между ними есть $n/2=l$ точек, в которых
$R_n(x)=0.$ Все эти $l$ нулей $R_n(x)$ лежат точно в {интервале} $(0,1),$
{что противоречит предыдущему выводу.} Следовательно, вместо~(\ref{Macar})
имеем противоположное неравенство, которое обращается в равенство при
$p_n(x)\equiv\dfrac{T_n(x)}{A_m^{n}}.$ В остальных случаях теорема
доказывается аналогично.
\end{proof}

\begin{Remark}
В общем случае многочлен наименьшего уклонения не
единственный. Например, при $n=2,$ $m=0$ и $0\le c\le 2$
имеем
{$$
\|1-cx^2\|_{C[-1,1]}=1=\left\|\dfrac{T_2(x)}{A_0^2}\right\|_{C[-1,1]}.
$$}
\end{Remark}



\begin{Corollary}[Оценки коэффициентов многочленов]
Пусть $p_n(x)=\sum\limits_{k=0}^n a_kx^k$ и известна норма
$\|p_n(\cdot )\|_{C[-1,1]}.$ Тогда для $m$ и $n$ одинаковой
четности
\[
  |a_m| \le |A_m^n|\cdot  \|p_n(\cdot )\|_{C[-1,1]},
\]
{а} для $m$ и $n$ разной четности
\[
  |a_m| \le |A_m^{n-1}|\cdot  \|p_n(\cdot )\|_{C[-1,1]}.
\]
\end{Corollary}

\begin{proof} Действительно, например, для $m$ и $n$ {одинаковой}
{четности и $a_m\neq 0$} из теоремы следует {оценка}
\[
  \Bigl\| \frac{p_{n}(\cdot )}{a_m} \Bigr\|_{C[-1,1]}
  \ge \frac{1}{|A_m^{n}|},
\]
а, значит, и требуемое неравенство. {При $a_m=0$ это неравенство тривиальное.}
Следовательно, у многочленов Чебышева коэффициенты одинаковой
со степенью многочлена четности самые большие
{среди многочленов $p_n$ той же степени с $\|p_n\|_{C[-1,1]}\le 1.$}

\begin{ex}
Провести доказательство теоремы Маркова для $m$ и $n$
разной четности.
\end{ex}
\end{proof}

\begin{Remark} Так как $a_m={\dfrac{p_n^{(m)}(0)}{p!}}$,
то {неравенства следствия можно переписать} {в виде}
\[
  |p_n^{(m)}(0)| \le m!\cdot |A_m^n|\cdot  \|p_n(\cdot )\|_{C[-1,1]},
\]
\[
{|p_n^{(m)}(0)| \le |T_n^{(m)}(0)|\cdot \|p_n(\cdot )\|_{C[-1,1]}\qquad (0\le m\le n)}
\]
для $m$ и $n$ одинаковой четности и
\[
  |p_n^{(m)}(0)| \le m!\cdot |A_m^{n-1}|\cdot  \|p_n(\cdot )\|_{C[-1,1]},
\]
\[
{|p_n^{(m)}(0)| \le |T_{n-1}^{(m)}(0)|\cdot \|p_n(\cdot )\|_{C[-1,1]} \qquad (0\le m\le n-1)}
\]
для
$m$ и $n$ разной четности.
\end{Remark}
%\end{proof}

\section{Экстремальная интерполяция на классе $W^{n+1}$}

{\bf 1. Четвертое экстремальное свойство}

\ \

Пусть $a \le x_0<x_1<\dots <x_n\le b$~-- узлы интерполяции на
$[a,b],$ $f \in \G M \subset C^{(n+1)}[a,b]$,~ $p_n(x,f)=\sum\limits_{k=0}^n
f(x_k)l_k(x)$~-- интерполяционный многочлен Лагранжа. Тогда, как мы знаем,
\[
  R_n(x,f)=f(x)-p_n(x,f)=\frac{f^{(n+1)}(\xi)}{(n+1)!} \omega (x),
\]
где $\xi \in [a,b]$ и $\omega(x)=(x-x_0)\cdots (x-x_n).$
Итак, задан класс функций $f \in \G M.$
Как подобрать узлы так, чтобы остаток интерполяционной формулы по всему классу был
наименьший?


Для класса $\G M$
рассмотрим величину
\begin{equation}
\label{estim1}
  \sup_{f \in \G M}\| R_{n}(\cdot ,f,\{x_k\}) \|_{C[-1,1]}
                      =F_n(\G M,\{x_k\}),
\end{equation}
{считая $[a,b]=[-1,1]$.} Задача ставится так: как выбрать узлы,
чтобы величина~(\ref{estim1}) была поменьше, т.\,е. надо найти
\[
  \inf_{\{x_k\}}F_n(\G M,\{x_k\})=\Phi_n(\G M).
\]
{Тогда} для любой функции $f \in \G M $ {и экстремальных узлов будем иметь}
$$
{\| R_{n}(\cdot ,f,\{x_k\}) \|_{C[-1,1]} \le \Phi_n(\G M).}
$$

Заметим, что если для заданной функции выбирать узлы так, чтобы величина
$\|f(\cdot )-p_n(\cdot ,f)\|_{C[-1,1]}$ была наименьшей, то это была бы
{\it другая задача}. Мы ее здесь не решаем.

Выберем в качестве $\G M$ класс
\[
 W^{(n+1)}=\{f{\in C^{n+1}[-1,1]}:\ \|f^{(n+1)}\|_{C[-1,1]}{\le 1}\}.
\]
Найдем
\[
  \inf_{\{x_k\}}\sup_{f \in W^{(n+1)}} \| R_{n}(\cdot ,f,\{x_k\}) \|_{C[-1,1]}.
\]
Зафиксируем $x \in [-1,1].$
Из формулы для остаточного члена в форме Коши
\[
   R_n(x,f)=\frac{f^{(n+1)}{(\xi )}}{(n+1)!} \,\,\omega (x)
\]
следует
\[
  \sup_{f \in \,W^{(n+1)}} | R_{n}(x ,f,\{x_k\}) | \le \frac{|\omega(x)|}{(n+1)!}.
\]
Так как существует функция, для которой $f^{(n+1)}(x) \equiv \pm 1,$ например,
если $f$ есть многочлен со старшим коэффициентом $\pm
\dfrac{1}{(n+1)!},$ то эта оценка обращается в равенство для любого $x \in [-1,1]$:
\[
  \sup_{f \in \,W^{(n+1)}} | R_{n}(x ,f,\{x_k\}) | =\frac{|\omega(x)|}{(n+1)!}.
\]
Значит, и
\[
  \sup_{f \in\, W^{(n+1)}} \| R_{n}(\cdot ,f,\{x_k\}) {\|}_{C[-1,1]} \le
                                \frac{\|\omega(\cdot )\|_{C[-1,1]}}{(n+1)!},
\]
причем {знак равенства будет} опять для той же функции.
Следовательно, задача свелась к нахождению
\[
  \inf_{\{x_k\}} \|\omega(\cdot )\|_{C[-1,1]}.
\]
Так как $\omega (x)=\prod\limits_{k=0}^n (x-x_k)=x^{n+1}+\cdots ,$ то
\[
  \inf_{\{x_k\}} \|\omega(\cdot )\|_{C[-1,1]} \ge
  \inf_{x^{n+1}+\cdots } \|p_{n+1}(\cdot )\|_{C[-1,1]}
  =\|\widetilde T_{n+1}(\cdot )\|_{C[-1,1]}=\frac{1}{2^{n}}.
\]
На самом деле здесь {равенство}, так как класс {всех рассматриваемых} $\omega (x)$~--
{это класс} многочленов со старшим коэффициентом,
равным $1$, и с нулями $x_k \in [-1,1]$\ $(k=0,\dots ,n),$ {а}
нормированный многочлен Чебышева {$\widetilde{T}_{n+1}$} лежит в этом
классе. Следовательно,
\[
  \inf_{\{x_k\}}\sup_{f \in W^{(n+1)}} \| R_{n}(\cdot ,f,\{x_k\}) \|_{C[-1,1]}
  =\frac{1}{(n+1)!}\cdot 2^{-n}
\]
и достигается для {узлов} $\{x_k\},$
являющихся нулями $T_{n+1}.$

\begin{task}
Найти
$$
\inf\limits_{\{x_k\}}\sup\limits_{f \in W^{(r)}}
             \| R_{n}(\cdot ,f,\{x_k\}) \|_{C[-1,1]}
$$
для всех $0 \le r \le n+1.$
\end{task}

Для $r=0$ задача {решена асимптотически,} для $r=n+1$ {решение задачи было} {изложено только
что,} для остальных $r$ задача не решена.

\section{Интерполяция в комплексной области}

Рассмотрим комплексную плоскость $\bC,$ область {$D \subset \bC,$} и пусть $w=f(z)$~--
комплекснозначная функция на $D.$ Пусть в области $D$
задана система различных точек $\{z_k\}$ $(k=0,\dots ,n).$

Построим многочлен $p_n(z,f)$
такой, чтобы
\[
  p_n(z_k)=f(z_k).
\]
Интерполяционная формула Лагранжа и здесь сохраняется:
\[
  p_n(z,f)=\sum\limits_{k=0}^n f(z_k)l_k(z),\qquad {l_k(z)=\frac{\omega(z)}{\omega'(z_k)
  (z-z_k)},\qquad
  \omega(z)=\prod_{m=0}^n(z-z_m)}.
\]
Пусть $f$~-- аналитическая (регулярная) функция в области $D,$ т.\,е. у
$f$ существует $f'$ в $D,$ и пусть $D$ односвязна. Найдем выражение для
остаточного члена. Рассмотрим в $D$ контур $C$ такой, что все точки $z_k$
лежат внутри $C.$ Функция $\dfrac{f(z)}{(z-z_0)\cdots (z-z_n)}$ внутри
$C$ имеет в качестве особенностей только точки $z_0,\dots ,z_n$ и имеет там
{устранимые особенности (если $f(z_k)=0$) или} простые полюса,
так как $z_k \ne z_l$ при $k \ne l.$ Тогда {по теореме о вычетах}
$$
  \frac{1}{2\pi i}\int_C\frac{f(t)\,dt}{(t-z_0)\cdots (t-z_n)}
  =
$$
$$
  =\sum\limits_{k=0}^n \frac{f(z_k)}{(z_k-z_0)\cdots
                    (z_k-z_{k-1})(z_k-z_{k+1})\cdots (z_k-z_n)}
 = \sum\limits_{k=0}^n \frac{f(z_k)}{\omega'(z_k)}.
$$
Зафиксируем точку $z$
внутри контура $C.$
Обозначим
\[
  J(z)=\frac{1}{2\pi i}\int_C \frac{f(t)\,dt}{(t-z)\prod_{k=0}^n(t-z_k)}.
\]
Тогда
\[
  J(z)=\frac{f(z)}{\prod_{k=0}^n(z-z_k)}-
  \sum\limits_{k=0}^n \frac{f(z_k)}{(z-z_k)\omega'(z_k)}
\]
или
%\begin{multline*}
\[
  J(z)\cdot \omega(z)=f(z)-\sum\limits_{k=0}^n
            \frac{f(z_k)\omega(z)}{(z-z_k)\omega'(z_k)}={f(z)-p_n(z,f)=}R_n(z,f)
 =\frac{1}{2\pi i}\int_C \frac{\omega(z)f(t)\,dt}{(t-z)\omega(t)}
\]
%\end{multline*}
для любого $z$ внутри $C.$ Представим $f(z)$ в виде интеграла Коши
\[
{f(z)=}\frac{1}{2\pi i}\int_C \frac{\omega(t)f(t)\,dt}{(t-z)\omega(t)}.
\]
Тогда интерполяционный многочлен
\[
  p_n(z,f)=f(z)-R_n(z,f)=
 \frac{1}{2\pi i}\int_C \frac{\{\omega(t)-\omega(z)\}f(t)dt}{(t-z)\omega(t)}.
\]

\section{Простейшие приложения\\ интерполяционной формулы
Лагранжа}\label{3.4}


Пусть {функция $f$} определена на $[a,b].$ В приложениях {часто} встречается
задача о вычислении значений функционалов и операторов на элементах $f$
функциональных пространств. Конечно, в общем случае здесь
можно говорить лишь о приближенных методах.

%\subsubsection{1. Вычисление {значений} функционала $F(f),$
%т.\,е.числа}
\vspace{3mm}
{\bf 1. Вычисление определенного интеграла}
\vspace{3mm}

Далее рассматривается эта задача на примере простого функционала
$\displaystyle\int_a^b f(x)\,dx.$ При этом для простейшего функционала~-- значения функции в
точке -- считается
известным точный или приближенный метод вычисления.

%\subsubsection{2. Вычисление {значений} оператора A(f){,
%т.\,е. функции $A(f)(x).$}}


%\subsection{Вычисление $\displaystyle\boldsymbol{\int_a^b f(x)dx}$}\label{3.2.1}

Пусть $f(x)\in C[a,b],$ $p_n(x,f)$~-- интерполяционный многочлен, $\{x_k\}$~--
узлы интерполяции, $\{f(x_k)\}$~-- значения функции в узлах.

Сводим интегрирование функции к интегрированию ее интерполяционного многочлена Лагранжа.
Для интеграла получится квадратурная формула.

Итак, заменяем вычисление $\ds\int_a^b f(x)\,dx$ вычислением
\begin{equation}\label{3-3.3}
  \int_a^b p_n(x,f)\,dx=\sum\limits_{k=0}^n f(x_k)\int_a^b l_k(x)\,dx=
  \sum\limits_{k=0}^n A_k f(x_k) \approx \int_a^b
  f(x)\,dx,
\end{equation}
где $l_k(x)=l_k(x;\{x_i\}_0^n),\ A_k=A_k(n).$
Эта формула называется квадратурной формулой {Котеса}, $\{x_k\}^n_{k=0}$~-- узлы
квадратурной формулы, $\{A_k^n\}^n_{k=0}$~-- коэффициенты квадратурной формулы или
коэффициенты Котеса.

Особенностью таких формул является то, что они точны для всех полиномов степени не выше $n,$
т.\,е. если $f$ есть многочлен степени не выше $n,$ то такая формула будет
точной, в {ее правой части} будет знак равенства.

Для того чтобы формула~(\ref{3-3.3})
была точна для любого многочлена степени не выше $n,$ она
должна быть точна для функций $x^p$\ $(p=0,1,\dots ,n).$

Значит, коэффициенты Котеса легко считать:
\[
  \sum\limits_{k=0}^n A_k x^p_k=\int_a^b x^p dx\qquad (p=0,1,\dots ,n).
\]
Это система с определителем Вандермонда, и~$A_k$ определяются однозначно.

Аналогичным образом можно использовать интерполяционные
многочлены Лагранжа для приближенного вычисления значений
других линейных функционалов. Оценки погрешности
соответствующих квадратурных формул выражаются через нормы
функционалов и оценки погрешности интерполяционных формул.

\begin{ex}
Выписать такие оценки для квадратурных формул, используя
результаты из предыдущих лекций.
\end{ex}

Можно строить кубатурные формулы, даже если
интерполяционные формулы написать нельзя. Например, для функций нескольких переменных
на кубе, где вообще нет непрерывных интерполяционных
систем, можно построить кубатурные формулы точные на многочленах заданной степени.
\vspace{3mm}

{\bf 2. Вычисление {значений} оператора ${A(f)},$
{т.\,е. функции ${A(f)(x)}$}}
\vspace{3mm}

Аналогичный прием можно использовать для приближенного
вычисления значений линейного оператора, если умеем
вычислять~$A(l_k)(x).$ Для неограниченных операторов это не
сработает. Например, возьмем
довольно простой аддитивный и однородный оператор~-- оператор дифференцирования.
Задача вычисления~$f'(x)$ по конечному числу точек {без
дополнительной информации о} {функции и ее производных} не имеет смысла,
так как оператор дифференцирования не ограничен в классе непрерывных
функций, {и значит не удастся указать} {гарантированную погрешность
приближенной формулы для~$f'(x).$} Оператор дифференцирования
можно сделать непрерывным на подпространстве $C^{(r)}[a,b]\subset C[a,b]$ функции с
непрерывной производной порядка $r,\ r\ge 2$, если
ввести в $C^{(r)}[a,b]$ норму соболевского типа,
полагая
$$
\|f\|=\|f\|_{C[a,b]}+\|f^{(r)}\|_{C[a,b]}.
$$
Ограниченность операторов дифференцирования порядка
$1,2,\ldots,r-1$ легко получить, используя, например, формулу Тейлора
с остаточным членом в форме Коши и следствия из неравенства братьев Марковых
для алгебраических многочленов $p_n(x)$
$$
\|p_n^{(k)}\|_{C[a,b]}\le \dfrac{2^{k}}{(b-a)^k}  n^{2k}\|p_n\|_{C[a,b]}.
$$
В случае периодических
функций и функций, определенных на $\mathbb R,$
ограниченность этих дифференциальных операторов на
пространствах $C^{(r)}_{2\pi}$ и $C^{(r)}(\mathbb R)$ с
соболевской нормой вытекает из соответствующих неравенств Колмогорова, которые
будут обсуждаться в дальнейшем\footnote{В лекциях 19 (п.~19.4) и 20 (п.~20.1).}.
